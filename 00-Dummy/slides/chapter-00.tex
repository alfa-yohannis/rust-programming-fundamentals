\documentclass[aspectratio=169, table]{beamer}

%\usepackage[beamertheme=./praditatheme]{Pradita}
\usepackage[utf8]{inputenc}
\usepackage{listings} 

\usetheme{Pradita}

\subtitle{IF120203-Programming Fundamentals}

\title{Session-04:\\\LARGE{
		Functions in Rust}}
\date[Serial]{\scriptsize {PRU/SPMI/FR-BM-18/0222}}
\author[Pradita]{\small{\textbf{Alfa Yohannis}}}

\lstdefinelanguage{Rust} {
	keywords={MAX, std, print, fn, let, mut, println, true, false, u8, u16, u32, u64, u128, i8, i16, i32, i64, i128, f32, f64, char, bool, if, else, for, while, loop, match, return, Vec},
	basicstyle=\ttfamily\small,
	keywordstyle=\color{blue}\bfseries,
	ndkeywords={self, String, Option, Some, None, Result, Ok, Err},
	ndkeywordstyle=\color{purple}\bfseries,
	sensitive=true,
	commentstyle=\color{gray},
	stringstyle=\color{red},
	numbers=left,
	numberstyle=\tiny\color{gray},
	breaklines=true,
	frame=lines,
	backgroundcolor=\color{lightgray!10},
	tabsize=2,
	comment=[l]{//},
	morecomment=[s]{/*}{*/},
	commentstyle=\color{gray}\ttfamily,
	stringstyle=\color{purple}\ttfamily,
	%morestring=[b]',
	%morestring=[b]"
	showstringspaces=false
}

\begin{document}
	
	\frame{\titlepage}
	
	\begin{frame}[fragile]
		\frametitle{Basic Function}
		\begin{lstlisting}[language=Rust]
			fn call_basic_function() {
				println!("Executing a basic function");
			}
		\end{lstlisting}
		\begin{itemize}
			\item A basic function without parameters or return values.
			\item Demonstrates the basic structure of a function in Rust.
		\end{itemize}
	\end{frame}
	
	\begin{frame}[fragile]
		\frametitle{Function with Inputs}
		\begin{lstlisting}[language=Rust]
			// Function with inputs (string and integer)
			fn display_employee_info(name: &str, salary: i32) {
				println!("Employee name: {} | Salary: {}", name, salary);
			}
		\end{lstlisting}
		\begin{itemize}
			\item Shows how to define a function with input parameters.
			\item Accepts a string and an integer as inputs.
		\end{itemize}
	\end{frame}
	
	\begin{frame}[fragile]
		\frametitle{Function with Inputs and Output}
		\begin{lstlisting}[language=Rust]
			// Function with inputs and output (multiplication of two integers)
			fn multiply_numbers(num1: i32, num2: i32) -> i32 {
				num1 * num2
			}
		\end{lstlisting}
		\begin{itemize}
			\item Demonstrates a function with inputs and a single output.
			\item Multiplies two integers and returns the result.
		\end{itemize}
	\end{frame}
	
	\begin{frame}[fragile]
		\frametitle{Function with Multiple Outputs}
		\begin{lstlisting}[language=Rust]
			// Function with inputs and multiple outputs (multiplication, addition, subtraction)
			fn perform_operations(num1: i32, num2: i32) -> (i32, i32, i32) {
				let multiplication = num1 * num2;
				let addition = num1 + num2;
				let subtraction = num1 - num2;
				(multiplication, addition, subtraction)
			}
		\end{lstlisting}
		\begin{itemize}
			\item Shows a function with inputs and multiple outputs.
			\item Performs and returns the results of multiplication, addition, and subtraction.
		\end{itemize}
	\end{frame}
	
	\begin{frame}[fragile]
		\frametitle{Using Functions in Main}
		\begin{lstlisting}[language=Rust]
			fn main() {
				call_basic_function(); 
				display_employee_info("John", 40_000); 
				
				let employee_name = "Alice";
				let employee_salary = 50_000;
				display_employee_info(employee_name, employee_salary); 
				
				let multiplication_result = multiply_numbers(10, 15);
				println!("The result of multiplication is {}", multiplication_result);
				
				let (multiply_result, add_result, subtract_result) = perform_operations(10, 15);
				println!("Multiplication = {}, Addition = {}, Subtraction = {}", multiply_result, add_result, subtract_result);
			}
		\end{lstlisting}
		\begin{itemize}
			\item Shows invoking different functions from the main function.
			\item Demonstrates passing literal values and variables as inputs.
			\item Displays the use of functions with both single and multiple return values.
		\end{itemize}
	\end{frame}
	
	\begin{frame}[fragile]
		\frametitle{Code Blocks in Rust}
		\begin{lstlisting}[language=Rust]
			// Example of a code block
			let full_name = {
				let first_name = "Bob";
				let last_name = "Smith";
				format!("{} {}", first_name, last_name)
			};
			println!("Full name: {}", full_name);
		\end{lstlisting}
		\begin{itemize}
			\item Demonstrates the use of code blocks in Rust.
			\item Creates and formats a full name within a block.
		\end{itemize}
	\end{frame}
	
	\begin{frame}[fragile]
		\frametitle{Reading and Parsing Input}
		\begin{lstlisting}[language=Rust]
			// Reading input from stdin and parsing it as a float
			let mut input_string = String::new();
			std::io::stdin()
			.read_line(&mut input_string)
			.expect("Failed to read input.");
			let parsed_number: f64 = input_string.trim().parse().expect("Invalid input");
			println!("Parsed number: {:?}", parsed_number);
		\end{lstlisting}
		\begin{itemize}
			\item Shows how to read input from the standard input.
			\item Parses the input string as a float and handles possible errors.
		\end{itemize}
	\end{frame}
	
\end{document}
