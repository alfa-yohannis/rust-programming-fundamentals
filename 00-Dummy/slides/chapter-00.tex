\documentclass[aspectratio=169, table]{beamer}

\usepackage[utf8]{inputenc}
\usepackage{listings} 

\usetheme{Pradita}

\subtitle{IF120203-Programming Fundamentals}

\title{Session-10:\\\LARGE{Database in Rust}\\ \vspace{15pt}}
\date[Serial]{\scriptsize {PRU/SPMI/FR-BM-18/0222}}
\author[Pradita]{\small{\textbf{Alfa Yohannis}}}


\lstdefinelanguage{Rust} {
keywords={::, use, const, struct, fn, let, mut, move, pub, str, i32, if, else, true, false, enum, match, impl, for, while, loop, return, mod, crate, super, Self, self, break, continue, extern, ref, static, type, unsafe, as, async, await, dyn, macro_rules},
basicstyle=\ttfamily\footnotesize,
keywordstyle=\color{blue}\bfseries,
ndkeywords={Result, Option, Ok, None, Item, Vec, Error, Some, app, button, draw, enums, frame, input, prelude, table, window, Window, Button, Input, SecretInput, Table, Frame, Lazy, Mutex, new},
ndkeywordstyle=\color{purple}\bfseries,
sensitive=true,
commentstyle=\color{gray},
stringstyle=\color{red},
numbers=left,
numberstyle=\tiny\color{gray},
breaklines=true,
frame=lines,
backgroundcolor=\color{lightgray!10},
tabsize=2,
comment=[l]{//},
morecomment=[s]{/*}{*/},
commentstyle=\color{gray}\ttfamily,
string=[s]{'}{'},
morestring=[s]{"}{"},
stringstyle=\color{teal}\ttfamily,
showstringspaces=false
}

\lstdefinelanguage{Sql} {
keywords={DROP, SELECT, FROM, UPDATE, DELETE, ALTER, TABLE, IF, EXISTS, CREATE, SERIAL, PRIMARY, KEY, VARCHAR, DEFAULT, NOT, NULL, INT, FLOAT4, INSERT, INTO, VALUES, REFERENCES, UNIQUE, WITH, TIME, ZONE},
basicstyle=\ttfamily\footnotesize,
keywordstyle=\color{blue}\bfseries,
ndkeywords={FLOOR, CAST, LPAD, CURRENT_TIMESTAMP},
ndkeywordstyle=\color{purple}\bfseries,
sensitive=true,
numbers=left,
numberstyle=\tiny\color{gray},
breaklines=true,
frame=lines,
backgroundcolor=\color{lightgray!10},
tabsize=2,
comment=[l]{--},
morecomment=[s]{/*}{*/},
commentstyle=\color{gray}\ttfamily,
string=[s]{'}{'},
morestring=[s]{"}{"},
stringstyle=\color{teal}\ttfamily,
showstringspaces=false
}

\begin{document}

\frame{\titlepage}

% Add table of contents slide
\begin{frame}{Contents}
	\vspace{15pt}
	\begin{columns}[t]
		\begin{column}{.5\textwidth}
			\tableofcontents[sections={1-12}]
		\end{column}
		\begin{column}{.5\textwidth}
			\tableofcontents[sections={13-24}]
		\end{column}
	\end{columns}
\end{frame}


\section{Database}
\begin{frame}[fragile]{Dropping Existing Tables}

\begin{lstlisting}[language=Sql]
DROP TABLE IF EXISTS item;
DROP TABLE IF EXISTS sales_order_details;
DROP TABLE IF EXISTS sales_orders;
\end{lstlisting}

\begin{itemize}
\item \texttt{DROP TABLE IF EXISTS} removes the specified table if it exists.
\item Used here to ensure that tables are recreated without errors.
\end{itemize}
\end{frame}

\section{PostgreSQL: Item Table}
\begin{frame}[fragile]{Creating the \texttt{item} Table (Part 1)}
\vspace{15pt}
\begin{lstlisting}[language=Sql]
CREATE TABLE item (
id SERIAL PRIMARY KEY,
code VARCHAR(10) DEFAULT LPAD(CAST(FLOOR(1 + random() * 9)::INT * 1000000000 + FLOOR(random() * 1000000000)::INT AS VARCHAR), 10, '0') NOT NULL UNIQUE,
\end{lstlisting}

\begin{itemize}
\item \texttt{id} is a unique identifier for each item.
\item \texttt{code} is a 10-character string, generated with leading zeros and uniqueness enforced.
\end{itemize}
\end{frame}

\begin{frame}[fragile]{Creating the \texttt{item} Table (Part 2)}

\begin{lstlisting}[language=Sql]
name VARCHAR NOT NULL,
currency VARCHAR(3) NOT NULL,
price FLOAT4 NOT NULL,
quantity FLOAT4 NOT NULL DEFAULT 0,
unit VARCHAR(50) NULL
);
\end{lstlisting}

\begin{itemize}
\item \texttt{name} is the item's name.
\item \texttt{currency} is a 3-character code for currency.
\item \texttt{price} is the cost of the item.
\item \texttt{quantity} is the available stock, defaulting to 0 if not specified.
\item \texttt{unit} is an optional field describing the measurement unit.
\end{itemize}
\end{frame}

\begin{frame}[fragile]{Inserting Data into \texttt{item}}

\begin{lstlisting}[language=Sql]
INSERT INTO item (code, name, currency, price, quantity, unit)
VALUES
('0000000010', 'Organic Almonds', 'USD', 15.49, 50.00, 'kg'),
('0000000020', 'Stainless Steel Water Bottle', 'USD', 23.99, 150.00, 'pcs'),
('0000000030', 'Bluetooth Headphones', 'USD', 89.99, 75.00, 'pcs');
\end{lstlisting}

\begin{itemize}
\item Inserts three rows into the \texttt{item} table.
\item Each row provides values for \texttt{code}, \texttt{name}, \texttt{currency}, \texttt{price}, \texttt{quantity}, and \texttt{unit}.
\end{itemize}
\end{frame}

\section{PostgreSQL: Sales Order Tables}
\begin{frame}[fragile]{Creating the \texttt{sales\_orders} Table}
\vspace{15pt}
\begin{lstlisting}[language=Sql]
CREATE TABLE sales_orders (
id SERIAL PRIMARY KEY,
code VARCHAR(10) NOT NULL UNIQUE,
order_date TIMESTAMP WITH TIME ZONE DEFAULT CURRENT_TIMESTAMP NOT NULL,
note VARCHAR(256) NULL
);
\end{lstlisting}

\begin{itemize}
\item \texttt{id} is a unique identifier for each order.
\item \texttt{code} is a unique identifier for the order.
\item \texttt{order\_date} records when the order was placed.
\item \texttt{note} is an optional field for additional information.
\end{itemize}
\end{frame}

\begin{frame}[fragile]{Creating the \texttt{sales\_order\_details} Table}
\vspace{15pt}
\begin{lstlisting}[language=Sql]
CREATE TABLE sales_order_details (
id SERIAL PRIMARY KEY,
order_code VARCHAR(10) NOT NULL REFERENCES sales_orders(code),
line_num INTEGER NOT NULL,
item_id INTEGER NOT NULL,
quantity REAL NOT NULL DEFAULT 0,
unit VARCHAR(50) NULL,
unit_price REAL NOT NULL
);
\end{lstlisting}
\vspace{-20pt}
\begin{columns}[t]
\begin{column}{.5\pagewidth}
\begin{itemize}
\item \texttt{id} is a unique identifier for each detail record.
\item \texttt{order\_code} links to the \texttt{sales\_orders} table.
\item \texttt{line\_num} indicates the item line number in the order.
\end{itemize}
\end{column}
\begin{column}{.5\pagewidth}
\begin{itemize}
\item \texttt{item\_id} references the \texttt{item} table.
\item \texttt{quantity} is the amount of the item ordered.
\item \texttt{unit} describes the measurement unit for the item.
\item \texttt{unit\_price} is the price per unit of the item.
\end{itemize}
\end{column}
\end{columns}
\end{frame}

\section{Rust: Publish Modules}
\begin{frame}[fragile]{Module Declarations in \texttt{src/lib.rs}}
\begin{lstlisting}[language=Rust]
pub mod item;
pub mod sales_order;
pub mod sales_order_detail;
\end{lstlisting}

\textbf{Explanation:}
\begin{itemize}
\item \texttt{pub mod item;} declares a public module named \texttt{item}.
\item \texttt{pub mod sales\_order;} declares a public module named \texttt{sales\_order}.
\item \texttt{pub mod sales\_order\_detail;} declares a public module named \texttt{sales\_order\_detail}.
\item Each module is expected to be defined in a corresponding file, such as \texttt{item.rs}, \texttt{sales\_order.rs}, and \texttt{sales\_order\_detail.rs}.
\end{itemize}
\end{frame}


\section{Rust: Item}
\begin{frame}[fragile]{Rust Code Overview: Item Module}
\vspace{15pt}
\begin{lstlisting}[language=Rust]
use tokio_postgres::{Client, Error};

#[derive(Debug)]
	pub struct Item {
	pub code: String,
	pub name: String,
	pub currency: String,
	pub price: f32,
	pub quantity: f32,
	pub unit: Option<String>,
}
\end{lstlisting}

\begin{itemize}
\item \texttt{Item} struct represents an item with attributes: \texttt{code}, \texttt{name}, \texttt{currency}, \texttt{price}, \texttt{quantity}, and \texttt{unit}.
\item \texttt{unit} is an optional field.
\end{itemize}
\end{frame}

\section{Rust: Item Operations}
\begin{frame}[fragile]{Adding an Item}
\vspace{15pt}
\begin{lstlisting}[language=Rust]
pub async fn add(client: &Client, item: &Item) -> Result<(), Error> {
	
let stmt = client.prepare("INSERT INTO item (code, name, currency, price, quantity, unit) VALUES ($1, $2, $3, $4, $5, $6)",).await?;

client.execute(&stmt, &[&item.code, &item.name, &item.currency, &item.price, &item.quantity, &item.unit,],).await?;

Ok(())
}
\end{lstlisting}

\begin{itemize}
\item \texttt{add} function inserts a new item into the \texttt{item} table.
\item It prepares and executes an \texttt{INSERT} SQL statement.
\end{itemize}
\end{frame}

\begin{frame}[fragile]{Deleting an Item}
\vspace{15pt}
\begin{lstlisting}[language=Rust]
pub async fn delete(client: &Client, code: &str) -> Result<(), Error> {

	let stmt = client.prepare("DELETE FROM item WHERE code = $1").await?;
	
	client.execute(&stmt, &[&code]).await?;
	
	Ok(())
}
\end{lstlisting}

\begin{itemize}
\item \texttt{delete} function removes an item from the \texttt{item} table based on the \texttt{code}.
\item It prepares and executes a \texttt{DELETE} SQL statement.
\end{itemize}
\end{frame}


\begin{frame}[fragile]{Updating an Item}
\vspace{15pt}
\begin{lstlisting}[language=Rust]
pub async fn update(client: &Client, item: &Item) -> Result<(), Error> {
	
	let stmt = client.prepare("UPDATE item SET name = $1, currency = $2, price = $3, quantity = $4, unit = $5 WHERE code = $6",).await?;
	
	client.execute(&stmt, &[&item.name, &item.currency, &item.price, &item.quantity, &item.unit, &item.code,],).await?;
	
	Ok(())
}
\end{lstlisting}

\begin{itemize}
\item \texttt{update} function updates the details of an existing item in the \texttt{item} table.
\item It prepares and executes an \texttt{UPDATE} SQL statement based on the \texttt{code}.
\end{itemize}
\end{frame}

\begin{frame}[fragile]{Listing All Items}
\vspace{15pt}
\begin{lstlisting}[language=Rust]
pub async fn list(client: &Client) -> Result<Vec<Item>, Error> {
	let rows = client
	.query("SELECT code, name, currency, price, quantity, unit FROM item ORDER BY code",&[],).await?;	
	let items = rows.iter().map(|row| Item {
		code: row.get("code"), name: row.get("name"),
		currency: row.get("currency"), price: row.get("price"),
		quantity: row.get("quantity"), unit: row.get("unit"),
	}).collect();
	Ok(items)
}
\end{lstlisting}

\begin{itemize}
\item \texttt{list} function retrieves all items from the \texttt{item} table.
\item It executes a \texttt{SELECT} SQL statement and maps the result to a vector of \texttt{Item} structs.
\end{itemize}
\end{frame}

\begin{frame}[fragile]{Getting a Specific Item}
\vspace{15pt}
\begin{lstlisting}[language=Rust]
pub async fn get(client: &Client, code: &str) -> Result<Option<Item>, Error> {
	let stmt = client.prepare("SELECT code, name, currency, price, quantity, unit FROM item WHERE code = $1").await?;
	let row = client.query_opt(&stmt, &[&code]).await?;
	if let Some(row) = row {
		Ok(Some(Item {
			code: row.get("code"), name: row.get("name"),
			currency: row.get("currency"), price: row.get("price"),
			quantity: row.get("quantity"), unit: row.get("unit"),
		}))
	} else { Ok(None) } }
\end{lstlisting}

\begin{itemize}
\item \texttt{get} function retrieves a specific item based on its \texttt{code}.
\item It executes a \texttt{SELECT} SQL statement and returns an \texttt{Option<Item>} where \texttt{None} indicates no item was found.
\end{itemize}
\end{frame}

\section{Rust: Sales Order}
\begin{frame}[fragile]{Rust Code Overview: Sales Order Module}
	\begin{lstlisting}[language=Rust]
		use chrono::{DateTime, Utc};
		use tokio_postgres::{Client, Error};
		
		#[derive(Debug, Clone)]
		pub struct SalesOrder {
			pub code: String,
			pub order_date: DateTime<Utc>,
			pub note: Option<String>,
		}
	\end{lstlisting}
	
	\begin{itemize}
		\item \texttt{SalesOrder} struct represents a sales order with attributes: \texttt{code}, \texttt{order\_date}, and \texttt{note}.
		\item \texttt{order\_date} is a timestamp with timezone.
		\item \texttt{note} is an optional field.
	\end{itemize}
\end{frame}

\section{Rust: Sales Order Operations}
\begin{frame}[fragile]{Adding a Sales Order}
	\begin{lstlisting}[language=Rust]
		pub async fn add_sales_order(client: &Client, order: &SalesOrder) -> Result<(), Error> {
			let stmt = client
			.prepare(
			"INSERT INTO sales_orders (code, order_date, note)
			VALUES ($1, $2, $3)",
			)
			.await?;
			
			client
			.execute(&stmt, &[&order.code, &order.order_date, &order.note])
			.await?;
			
			Ok(())
		}
	\end{lstlisting}
	
	\begin{itemize}
		\item \texttt{add\_sales\_order} function inserts a new sales order into the \texttt{sales\_orders} table.
		\item It prepares and executes an \texttt{INSERT} SQL statement.
	\end{itemize}
\end{frame}

\begin{frame}[fragile]{Getting a Sales Order}
	\begin{lstlisting}[language=Rust]
		pub async fn get_sales_order(client: &Client, code: &str) -> Result<Option<SalesOrder>, Error> {
			let stmt = client
			.prepare("SELECT code, order_date, note FROM sales_orders WHERE code = $1")
			.await?;
			
			let row_opt = client.query_opt(&stmt, &[&code]).await?;
			
			if let Some(row) = row_opt {
				Ok(Some(SalesOrder {
					code: row.get("code"),
					order_date: row.get("order_date"),
					note: row.get("note"),
				}))
			} else {
				Ok(None)
			}
		}
	\end{lstlisting}
	
	\begin{itemize}
		\item \texttt{get\_sales\_order} function retrieves a specific sales order based on its \texttt{code}.
		\item It executes a \texttt{SELECT} SQL statement and returns an \texttt{Option<SalesOrder>} where \texttt{None} indicates no order was found.
	\end{itemize}
\end{frame}

\begin{frame}[fragile]{Updating a Sales Order}
	\begin{lstlisting}[language=Rust]
		pub async fn update_sales_order(client: &Client, order: &SalesOrder) -> Result<(), Error> {
			let stmt = client
			.prepare(
			"UPDATE sales_orders
			SET order_date = $1, note = $2
			WHERE code = $3",
			)
			.await?;
			
			client
			.execute(&stmt, &[&order.order_date, &order.note, &order.code])
			.await?;
			
			Ok(())
		}
	\end{lstlisting}
	
	\begin{itemize}
		\item \texttt{update\_sales\_order} function updates an existing sales order in the \texttt{sales\_orders} table.
		\item It prepares and executes an \texttt{UPDATE} SQL statement based on the \texttt{code}.
	\end{itemize}
\end{frame}

\begin{frame}[fragile]{Deleting a Sales Order}
	\begin{lstlisting}[language=Rust]
		pub async fn delete_sales_order(client: &Client, code: &str) -> Result<(), Error> {
			let stmt = client.prepare("DELETE FROM sales_orders WHERE code = $1").await?;
			client.execute(&stmt, &[&code]).await?;
			Ok(())
		}
	\end{lstlisting}
	
	\begin{itemize}
		\item \texttt{delete\_sales\_order} function removes a sales order from the \texttt{sales\_orders} table based on the \texttt{code}.
		\item It prepares and executes a \texttt{DELETE} SQL statement.
	\end{itemize}
\end{frame}

\begin{frame}[fragile]{Listing All Sales Orders}
	\begin{lstlisting}[language=Rust]
		pub async fn list_sales_orders(client: &Client) -> Result<Vec<SalesOrder>, Error> {
			let rows = client.query("SELECT code, order_date, note FROM sales_orders ORDER BY code", &[]).await?;
			
			let orders = rows
			.iter()
			.map(|row| SalesOrder {
				code: row.get("code"),
				order_date: row.get("order_date"),
				note: row.get("note"),
			})
			.collect();
			
			Ok(orders)
		}
	\end{lstlisting}
	
	\begin{itemize}
		\item \texttt{list\_sales\_orders} function retrieves all sales orders from the \texttt{sales\_orders} table.
		\item It executes a \texttt{SELECT} SQL statement and maps the result to a vector of \texttt{SalesOrder} structs.
	\end{itemize}
\end{frame}

\section{Rust: Sales Order Detail}

\begin{frame}[fragile]{SalesOrderDetail Struct}
	\begin{lstlisting}[language=Rust]
		use tokio_postgres::{Client, Error};
		
		#[derive(Debug, Clone)]
		pub struct SalesOrderDetail {
			pub id: i32,
			pub order_code: String,
			pub line_num: i32,
			pub item_id: i32,
			pub quantity: f32,
			pub unit: Option<String>,
			pub unit_price: f32,
		}
	\end{lstlisting}
	\begin{itemize}
		\item \texttt{id} is the unique identifier for each detail record.
		\item \texttt{order\_code} is the reference to the order.
		\item \texttt{line\_num} is the line number of the item in the order.
		\item \texttt{item\_id} is the identifier of the item.
		\item \texttt{quantity} is the amount of the item ordered.
		\item \texttt{unit} describes the measurement unit.
		\item \texttt{unit\_price} is the price per unit of the item.
	\end{itemize}
\end{frame}

\section{Rust: Sales Order Detail Operations}

\begin{frame}[fragile]{Adding Sales Order Detail}
	\begin{lstlisting}[language=Rust]
		pub async fn add_sales_order_detail(
		client: &Client,
		detail: &SalesOrderDetail,
		) -> Result<(), Error> {
			let stmt = client
			.prepare(
			"INSERT INTO sales_order_details (order_code, line_num, item_id, quantity, unit, unit_price)
			VALUES ($1, $2, $3, $4, $5, $6)",
			)
			.await?;
			
			client
			.execute(
			&stmt,
			&[
			&detail.order_code,
			&detail.line_num,
			&detail.item_id,
			&detail.quantity,
			&detail.unit,
			&detail.unit_price,
			],
			)
			.await?;
			
			Ok(())
		}
	\end{lstlisting}
	\begin{itemize}
		\item Prepares an \texttt{INSERT} statement for adding a sales order detail.
		\item Executes the statement with the detail's attributes.
	\end{itemize}
\end{frame}

\begin{frame}[fragile]{Getting Sales Order Detail by ID}
	\begin{lstlisting}[language=Rust]
		pub async fn get_sales_order_detail(
		client: &Client,
		id: i32,
		) -> Result<Option<SalesOrderDetail>, Error> {
			let stmt = client
			.prepare(
			"SELECT id, order_code, line_num, item_id, quantity, unit, unit_price
			FROM sales_order_details WHERE id = $1",
			)
			.await?;
			
			let row_opt = client.query_opt(&stmt, &[&id]).await?;
			
			if let Some(row) = row_opt {
				Ok(Some(SalesOrderDetail {
					id: row.get("id"),
					order_code: row.get("order_code"),
					line_num: row.get("line_num"),
					item_id: row.get("item_id"),
					quantity: row.get("quantity"),
					unit: row.get("unit"),
					unit_price: row.get("unit_price"),
				}))
			} else {
				Ok(None)
			}
		}
	\end{lstlisting}
	\begin{itemize}
		\item Prepares a \texttt{SELECT} statement to fetch a sales order detail by \texttt{id}.
		\item Returns \texttt{Option<SalesOrderDetail>} based on the query result.
	\end{itemize}
\end{frame}

\begin{frame}[fragile]{Updating Sales Order Detail}
	\begin{lstlisting}[language=Rust]
		pub async fn update_sales_order_detail(
		client: &Client,
		detail: &SalesOrderDetail,
		) -> Result<(), Error> {
			let stmt = client
			.prepare(
			"UPDATE sales_order_details
			SET quantity = $1, unit = $2, unit_price = $3
			WHERE order_code = $4 and line_num = $5",
			)
			.await?;
			
			client
			.execute(
			&stmt,
			&[
			&detail.quantity,
			&detail.unit,
			&detail.unit_price,
			&detail.order_code,
			&detail.line_num,
			],
			)
			.await?;
			
			Ok(())
		}
	\end{lstlisting}
	\begin{itemize}
		\item Prepares an \texttt{UPDATE} statement to modify existing sales order details.
		\item Updates \texttt{quantity}, \texttt{unit}, and \texttt{unit\_price} based on \texttt{order\_code} and \texttt{line\_num}.
	\end{itemize}
\end{frame}

\begin{frame}[fragile]{Deleting Sales Order Detail}
	\begin{lstlisting}[language=Rust]
		pub async fn delete_sales_order_detail(
		client: &Client,
		detail: &SalesOrderDetail,
		) -> Result<(), Error> {
			let stmt = client
			.prepare("DELETE FROM sales_order_details WHERE order_code = $1 and line_num = $2")
			.await?;
			client
			.execute(&stmt, &[&detail.order_code, &detail.line_num])
			.await?;
			Ok(())
		}
	\end{lstlisting}
	\begin{itemize}
		\item Prepares a \texttt{DELETE} statement to remove sales order details based on \texttt{order\_code} and \texttt{line\_num}.
	\end{itemize}
\end{frame}

\begin{frame}[fragile]{Deleting Sales Order Detail by Code}
	\begin{lstlisting}[language=Rust]
		pub async fn delete_sales_order_detail_by_code(client: &Client, code: &str) -> Result<(), Error> {
			let stmt = client
			.prepare("DELETE FROM sales_order_details WHERE order_code = $1")
			.await?;
			client.execute(&stmt, &[&code]).await?;
			Ok(())
		}
	\end{lstlisting}
	\begin{itemize}
		\item Prepares a \texttt{DELETE} statement to remove all details for a given \texttt{order\_code}.
	\end{itemize}
\end{frame}

\begin{frame}[fragile]{Listing All Sales Order Details}
	\begin{lstlisting}[language=Rust]
		pub async fn list_sales_order_details(client: &Client) -> Result<Vec<SalesOrderDetail>, Error> {
			let rows = client
			.query(
			"SELECT id, order_code, line_num, item_id, quantity, unit, unit_price
			FROM sales_order_details ORDER BY order_code, line_num",
			&[],
			)
			.await?;
			
			let details = rows
			.iter()
			.map(|row| SalesOrderDetail {
				id: row.get("id"),
				order_code: row.get("order_code"),
				line_num: row.get("line_num"),
				item_id: row.get("item_id"),
				quantity: row.get("quantity"),
				unit: row.get("unit"),
				unit_price: row.get("unit_price"),
			})
			.collect();
			
			Ok(details)
		}
	\end{lstlisting}
	\begin{itemize}
		\item Queries all sales order details ordered by \texttt{order\_code} and \texttt{line\_num}.
		\item Returns a \texttt{Vec<SalesOrderDetail>} containing all records.
	\end{itemize}
\end{frame}

\begin{frame}[fragile]{Getting Sales Order Details by Order Code}
	\begin{lstlisting}[language=Rust]
		pub async fn get_details_by_order_code(
		client: &Client,
		order_code: &str,
		) -> Result<Vec<SalesOrderDetail>, Error> {
			let stmt = client
			.prepare(
			"SELECT id, order_code, line_num, item_id, quantity, unit, unit_price
			FROM sales_order_details WHERE order_code = $1 ORDER BY line_num",
			)
			.await?;
			
			let rows = client.query(&stmt, &[&order_code]).await?;
			
			let details = rows
			.iter()
			.map(|row| SalesOrderDetail {
				id: row.get("id"),
				order_code: row.get("order_code"),
				line_num: row.get("line_num"),
				item_id: row.get("item_id"),
				quantity: row.get("quantity"),
				unit: row.get("unit"),
				unit_price: row.get("unit_price"),
			})
			.collect();
			
			Ok(details)
		}
	\end{lstlisting}
	\begin{itemize}
		\item Retrieves sales order details for a specific \texttt{order\_code}.
		\item Orders results by \texttt{line\_num}.
	\end{itemize}
\end{frame}

\end{document}