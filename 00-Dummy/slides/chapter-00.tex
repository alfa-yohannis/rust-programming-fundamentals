\documentclass[aspectratio=169, table]{beamer}

\usepackage[utf8]{inputenc}
\usepackage{listings}

\usetheme{Pradita}

\subtitle{IF120203-Programming Fundamentals}

\title{Session-21:\\\LARGE{Creating and Publishing a Rust Crate}\\ \vspace{10pt}}
\date[Serial]{\scriptsize {PRU/SPMI/FR-BM-18/0225}}
\author[Pradita]{\small{\textbf{Alfa Yohannis}}}

\lstdefinelanguage{Rust} {
	keywords={MAX, std, print, fn, let, mut, println, true, false, u8, u16, u32, u64, u128, i8, i16, i32, i64, i128, f32, f64, char, bool, if, else, for, while, loop, match, return},
	basicstyle=\ttfamily\small,
	keywordstyle=\color{blue}\bfseries,
	ndkeywords={self, String, HashMap, Entry, or_insert},
	ndkeywordstyle=\color{purple}\bfseries,
	sensitive=true,
	commentstyle=\color{gray},
	stringstyle=\color{red},
	numbers=left,
	numberstyle=\tiny\color{gray},
	breaklines=true,
	frame=lines,
	backgroundcolor=\color{lightgray!10},
	tabsize=2,
	comment=[l]{//},
	morecomment=[s]{/*}{*/},
	commentstyle=\color{gray}\ttfamily,
	stringstyle=\color{purple}\ttfamily,
	showstringspaces=false
}

\begin{document}
	
	\frame{\titlepage}
	
	% Add table of contents slide
	\begin{frame}
		\tableofcontents
	\end{frame}
	
	\section{Documentation Comments}
	\begin{frame}[fragile]
		\frametitle{Documentation Comments}
		\begin{lstlisting}[language=Rust]
			/* 
			These lines are not intended for documentation
			// Documentation comments begin with three slashes. Ensure your code is error-free and well-documented before publishing.
			
			// Command to generate documentation: cargo doc --open
			*/
		\end{lstlisting}
		\begin{itemize}
			\item Regular comments using \texttt{/* */} and \texttt{//} are not included in documentation.
			\item Documentation comments use \texttt{///}.
			\item Generate documentation with \texttt{cargo doc --open}.
		\end{itemize}
	\end{frame}
	
	\section{Module-Level Documentation}
	\begin{frame}[fragile]
		\frametitle{Module-Level Documentation}
		\begin{lstlisting}[language=Rust]
			// The following lines will be included in the documentation 
			// _______________________________________________
			
			//! # Fitness Tracker Crate
			//! 
			//! This crate provides utilities for basic fitness calculations and metrics.
		\end{lstlisting}
		\begin{itemize}
			\item Use \texttt{//!} for module-level documentation.
			\item Provides an overview of the crate and its purpose.
		\end{itemize}
	\end{frame}
	
	\section{Function Documentation}
	\begin{frame}[fragile]
		\frametitle{Function Documentation: \texttt{calculate\_bmi}}
		\begin{lstlisting}[language=Rust]
			/// Calculates the Body Mass Index (BMI) from weight and height
			/// 
			/// # Examples
			/// ```  
			/// let weight_kg = 70;
			/// let height_m = 1.75;
			/// let bmi = fitness_tracker::calculate_bmi(weight_kg, height_m);
			/// assert_eq!(22.86, bmi); 
			/// ``` 
			/// # Panics
			/// Panics if height is zero or negative.
			/// 
			/// # Additional Information
			/// BMI is calculated using the formula weight (kg) / (height (m) * height (m)).
			
			pub fn calculate_bmi(weight: f32, height: f32) -> f32 {
				weight / (height * height)
			} 
		\end{lstlisting}
		\begin{itemize}
			\item Documents the \texttt{calculate\_bmi} function.
			\item Includes examples, panic conditions, and additional information.
		\end{itemize}
	\end{frame}
	
	\section{Function Documentation}
	\begin{frame}[fragile]
		\frametitle{Function Documentation: \texttt{calculate\_calories}}
		\begin{lstlisting}[language=Rust]
			/// Calculates the daily calorie needs based on basal metabolic rate (BMR) and activity level
			/// 
			/// # Examples
			/// ``` 
			/// let bmr = 1500.0;
			/// let activity_factor = 1.2;
			/// let daily_calories = fitness_tracker::calculate_calories(bmr, activity_factor);
			/// assert_eq!(1800.0, daily_calories); 
			/// ```  
			pub fn calculate_calories(bmr: f32, activity_level: f32) -> f32 {
				bmr * activity_level
			}
		\end{lstlisting}
		\begin{itemize}
			\item Documents the \texttt{calculate\_calories} function.
			\item Includes examples to demonstrate usage.
		\end{itemize}
	\end{frame}
	
	\section{Publishing Your Crate}
	\begin{frame}[fragile]
		\frametitle{Publishing Your Crate}
		\begin{lstlisting}[language=Rust]
			// Steps for publishing your crate
			// 1. Create a GitHub account and log in to crates.io.
			// 2. Navigate to the crates.io dashboard, access API tokens, and create a new token.
			// 3. Copy the token.
			// 4. In the terminal, execute cargo login followed by the token from crates.io.
			// 5. Go back to crates.io, verify your email in Account Settings -> Profile -> Email -> Save.
			// 6. Ensure your Cargo.toml includes: version = "0.1.1", edition = "2021", authors = ["Your Name"], description = "Fitness calculation utilities", license = "MIT".
			// 7. Run cargo publish --allow-dirty to publish your crate.
			// 8. To update, modify the version number in Cargo.toml and run the publish command again.
			// 9. Your crate will appear on your dashboard.
			// 10. From your dashboard, you can yank or unyank versions to control downloads.
			// 11. Your crate will also be searchable.
		\end{lstlisting}
		\begin{itemize}
			\item Step-by-step guide for publishing your crate to \texttt{crates.io}.
			\item Important to ensure all fields in \texttt{Cargo.toml} are correctly filled out.
		\end{itemize}
	\end{frame}
	
	\section{Demonstrating Crate Usage}
	\begin{frame}[fragile]
		\frametitle{Demonstrating Crate Usage}
		\begin{lstlisting}[language=Rust]
			// To demonstrate usage, create a new project and include the following crate:
			use my_bmi_crate_01::calculate_bmi; 
			fn main() {
				println!("BMI for 70kg and 1.75m height: {}", calculate_bmi(70.0, 1.75));
				println!("Daily calories needed: {}", calculate_calories(1500.0, 1.2));
			}
		\end{lstlisting}
		\begin{itemize}
			\item Shows how to use the \texttt{calculate\_bmi} and \texttt{calculate\_calories} functions from the crate.
			\item Practical example of including and using a custom crate in a new project.
		\end{itemize}
	\end{frame}
	
\end{document}
