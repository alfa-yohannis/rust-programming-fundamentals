\documentclass[aspectratio=169, table]{beamer}

%\usepackage[beamertheme=./praditatheme]{Pradita}
\usepackage[utf8]{inputenc}
\usepackage{listings} 


\usetheme{Pradita}

\subtitle{IF120203-Programming Fundamentals}

\title{\LARGE{Session-01:\\Introduction}
\vspace{20pt}}
\date[Serial]{\scriptsize {PRU/SPMI/FR-BM-18/0222}}
\author[Pradita]{\small{\textbf{Alfa Yohannis}}}

%\lstdefinelanguage{Rust}{
%	keywords={!, fn, let, mut, println, true, false, u8, u16, u32, u64, u128, i8, i16, i32, i64, i128, f32, f64, char, bool, if, else, for, while, loop, match, return},
%	keywordstyle=\color{blue}\bfseries,
%	ndkeywords={self, String, Option, Some, None, Result, Ok, Err},
%	ndkeywordstyle=\color{purple}\bfseries,
%	identifierstyle=\color{black},
%	sensitive=true,
%	comment=[l]{//},
%	morecomment=[s]{/*}{*/},
%	commentstyle=\color{gray}\ttfamily,
%	stringstyle=\color{purple}\ttfamily,
%	morestring=[b]',
%	morestring=[b]"
%}

\lstdefinelanguage{Rust} {
keywords={MAX, std, print, fn, let, mut, println, true, false, u8, u16, u32, u64, u128, i8, i16, i32, i64, i128, f32, f64, char, bool, if, else, for, while, loop, match, return},
basicstyle=\ttfamily\small,
keywordstyle=\color{blue}\bfseries,
ndkeywords={self, String, Option, Some, None, Result, Ok, Err},
ndkeywordstyle=\color{purple}\bfseries,
sensitive=true,
commentstyle=\color{gray},
stringstyle=\color{red},
numbers=left,
numberstyle=\tiny\color{gray},
breaklines=true,
frame=lines,
backgroundcolor=\color{lightgray!10},
tabsize=2,
comment=[l]{//},
morecomment=[s]{/*}{*/},
	commentstyle=\color{gray}\ttfamily,
stringstyle=\color{purple}\ttfamily,
%morestring=[b]',
%morestring=[b]"
showstringspaces=false
}

\begin{document}

\frame{\titlepage}

\begin{frame}[fragile]
\frametitle{Rust: Hello, World!}
\begin{lstlisting}[language=Rust]
fn main() {
  println!("Hello, world!");
}
\end{lstlisting}
\begin{itemize}
\item This is a basic Rust program that prints "Hello, world!" to the console.
\item \texttt{fn main()} is the entry point of every Rust program.
\item \texttt{println!("Hello, world!");} outputs the text to the standard output (console).
\end{itemize}
\end{frame}

\begin{frame}[fragile]
\frametitle{Program Outputs and Comments}
\begin{lstlisting}[language=Rust]
fn main() {
	// This is the first program in this tutorial.
	// Here is another line of comment.
	
	/* 
	This comment spans 
	multiple lines 
	and demonstrates a block comment.
	*/
	
	// Printing a simple message.
	println!("Hello from the Rust program!");
\end{lstlisting}
\end{frame}

\begin{frame}
\frametitle{Program Outputs and Comments}
\begin{itemize}
	\item Comments (\texttt{//} and \texttt{/* ... */}) document and explain the code.
	\item Block comments (\texttt{/* ... */}) span multiple lines for longer explanations.
	\item \texttt{println!} macro prints "Hello from the Rust program!" to the console.
\end{itemize}
\end{frame}

\begin{frame}[fragile]
\frametitle{Printing and Inline Comments}
\begin{lstlisting}[language=Rust]
	// Printing with an inline comment affecting the command.
	print/*ln*/!("Hello, world without newline!");
	
	// Demonstrating basic output commands.
	println!("The value of the constant is {}", 100);
\end{lstlisting}
\end{frame}

\begin{frame}
\frametitle{Printing and Inline Comments}
\begin{itemize}
	\item Inline comment (\texttt{print/*ln*/!}) affects the behavior of the command.
	\item \texttt{println!} prints the value of a constant (100) within a formatted string.
\end{itemize}
\end{frame}

\begin{frame}[fragile]
\frametitle{Formatted Strings and Placeholders}
\begin{lstlisting}[language=Rust]
	// Printing a formatted string with placeholders.
	println!("My first name is {} and my last name is {}", "Jane", "Doe");
	
	// Using the `print!` command which does not add a newline.
	print!("This text is printed ");
	print!("on the same line.");
\end{lstlisting}
\end{frame}

\begin{frame}
\frametitle{Formatted Strings and Placeholders}
\begin{itemize}
	\item Formatted strings with placeholders (\texttt{\{\}}) insert variable values into the string.
	\item \texttt{print!} prints text without automatically adding a newline.
\end{itemize}
\end{frame}

\begin{frame}[fragile]
\frametitle{Printing Across Multiple Lines}
\begin{lstlisting}[language=Rust]
	// Printing text over multiple lines.
	print!("\nThis text is split
	across multiple 
	lines."); 
\end{lstlisting}
\end{frame}

\begin{frame}
\frametitle{Printing Across Multiple Lines}
\begin{itemize}
	\item Rust supports multiline strings within \texttt{print!} and \texttt{println!} macros.
	\item Facilitates clearer code formatting with line breaks.
\end{itemize}
\end{frame}

\begin{frame}[fragile]
\frametitle{Escape Sequences and Special Characters}
\begin{lstlisting}[language=Rust]
	// Using escape sequences for special formatting.
	println!("\n\\n\n This line starts after a double newline. \t This line has a tab.");
	
	// Demonstrating the use of backslashes.
	println!("Printing single quote \' and double quote \"");
	println!("Printing a single backslash \\");
	print!("This text is overwritten \r only this part is shown");
\end{lstlisting}
\end{frame}

\begin{frame}
\frametitle{Escape Sequences and Special Characters}
\begin{itemize}
	\item Escape sequences (\texttt{\textbackslash n}, \texttt{\textbackslash t}, \texttt{\textbackslash\textbackslash}, \texttt{\textbackslash r}) for special formatting.
	\item Print single quote (\texttt{\'}) and double quote (\texttt{\"}) using escape sequences.
\end{itemize}
\end{frame}

\begin{frame}[fragile]
\frametitle{Macro Arguments and Arithmetic Operations}
\begin{lstlisting}[language=Rust]
	// Using positional arguments in the print macro.
	println!("\nI have been {2} for {1} years and I {0} it", "enjoying", 15, "coding");
	
	// Using named arguments in the print macro.
	println!("Rust is a systems programming language that's great for development.");
	
	// Performing and printing basic arithmetic.
	println!("The result of 42 + 58 is {}", 42 + 58);
}
\end{lstlisting}
\end{frame}

\begin{frame}
\frametitle{Macro Arguments and Arithmetic Operations}
\begin{itemize}
\item Positional (\texttt{\{\}}) and named (\texttt{\{language\}}) arguments in \texttt{println!} macro for dynamic output.
\item Rust handles basic arithmetic operations (\texttt{42 + 58}) and prints the result.
\end{itemize}
\end{frame}



%----


\begin{frame}[fragile]{Variables in Rust}
\begin{lstlisting}[language=Rust]
fn main() {
	let mut a = 10;
	println!("The initial value of variable a = {}", a);
	
	a = 50;
	println!("The updated value of variable a = {}", a);
	
	let b = 6 * 6;
}
\end{lstlisting}
\end{frame}

\begin{frame}{Discussion}
\begin{itemize}
\item Introduced a mutable variable `a` initialized to 10, then updated to 50.
\item Declared an immutable variable `b` as the product of 6 * 6.
\end{itemize}
\end{frame}

\begin{frame}[fragile]{Data Types in Rust}
\begin{lstlisting}[language=Rust]
fn main() {
	// Integers
	println!("The Maximum value for i8 = {}", std::i8::MAX);
	println!("The Maximum value for u8 = {}", std::u8::MAX);
	
	// Floats
	let c = 2.75;
	println!("The sum of an integer and a float is {}", a as f64 + c);
	println!("The Maximum value for f32 = {}", std::f32::MAX);
}
\end{lstlisting}
\end{frame}

\begin{frame}{Discussion}
\begin{itemize}
\item Demonstrated integer types and their maximum values (i8 and u8).
\item Showcased floating-point type `f32` and `f64` with maximum value.
\end{itemize}
\end{frame}

\begin{frame}[fragile]{Boolean and Characters in Rust}
\begin{lstlisting}[language=Rust]
fn main() {
	// Boolean
	let is_active = true;
	println!("The values of our variables are {:?}", (a, b, c, is_active));
	
	let is_not_equal = 10 != 20;
	println!("Is 10 not equal to 20? {}", is_not_equal);
	
	// Characters
	let char1 = 'X';
	let char2 = '8';
	let char3 = '&';
	let char4 = '\u{03A9}'; // Unicode character for Omega (Ω)
	let char5 = '\'';
	
	println!("The value of char1 is {}, char2 is {}, char3 is {}, char4 is {} and char5 is {}", char1, char2, char3, char4, char5);
}
\end{lstlisting}
\end{frame}

\begin{frame}{Discussion}
\begin{itemize}
\item Introduced boolean type with `true` and `false`.
\item Used comparison operator (`!=`) to check inequality.
\item Demonstrated character types with various examples including Unicode.
\end{itemize}
\end{frame}

\end{document}
