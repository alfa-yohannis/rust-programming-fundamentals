\documentclass[aspectratio=169, table]{beamer}

%\usepackage[beamertheme=./praditatheme]{Pradita}
\usepackage[utf8]{inputenc}
\usepackage{listings} 

\usetheme{Pradita}

\subtitle{IF120203-Programming Fundamentals}

\title{Session-02:\\\LARGE{
Variables, Data Types, Functions,\\
Inputs}}
\date[Serial]{\scriptsize {PRU/SPMI/FR-BM-18/0222}}
\author[Pradita]{\small{\textbf{Alfa Yohannis}}}

\lstdefinelanguage{Rust} {
keywords={MAX, std, print, fn, let, mut, println, true, false, u8, u16, u32, u64, u128, i8, i16, i32, i64, i128, f32, f64, char, bool, if, else, for, while, loop, match, return},
basicstyle=\ttfamily\small,
keywordstyle=\color{blue}\bfseries,
ndkeywords={self, String, Option, Some, None, Result, Ok, Err},
ndkeywordstyle=\color{purple}\bfseries,
sensitive=true,
commentstyle=\color{gray},
stringstyle=\color{red},
numbers=left,
numberstyle=\tiny\color{gray},
breaklines=true,
frame=lines,
backgroundcolor=\color{lightgray!10},
tabsize=2,
comment=[l]{//},
morecomment=[s]{/*}{*/},
commentstyle=\color{gray}\ttfamily,
stringstyle=\color{purple}\ttfamily,
%morestring=[b]',
%morestring=[b]"
showstringspaces=false
}

\begin{document}

\frame{\titlepage}

\begin{frame}[fragile]
\frametitle{Initializing Multiple Variables}
\begin{lstlisting}[language=Rust]
fn main() {
// Initialize height and weight variables.
let (height, weight) = (175, 70.5);
println!("Height: {} cm, Weight: {} kg", height, weight);
\end{lstlisting}
\begin{itemize}
\item Demonstrates tuple assignment.
\item Initializes two variables simultaneously.
\end{itemize}
\end{frame}

\begin{frame}[fragile]
\frametitle{Readability of Large Numbers}
\begin{lstlisting}[language=Rust]
// Declare city population with underscores for readability.
let population = 8_000_000;
println!("City population: {}", population);
\end{lstlisting}
\begin{itemize}
\item Underscores improve readability for large numbers.
\end{itemize}
\end{frame}

\begin{frame}[fragile]
\frametitle{Integer Overflow}
\begin{lstlisting}[language=Rust]
// Uncommenting the following line would cause an overflow error for \texttt{u8}.
// let max_byte: u8 = 300;
\end{lstlisting}
\begin{itemize}
\item Demonstrates potential overflow error.
\item \texttt{u8} type can only hold values from 0 to 255.
\end{itemize}
\end{frame}

\begin{frame}[fragile]
\frametitle{Decimal Numbers in Different Formats}
\begin{lstlisting}[language=Rust]
// Declare a distance variable and print it in different numeral systems.
let distance = 123;
println!("Distance: Hexadecimal {:X}, Octal {:o}, Binary {:b}", distance, distance, distance);
\end{lstlisting}
\begin{itemize}
\item Shows decimal, hexadecimal, octal, and binary representations.
\end{itemize}
\end{frame}

\begin{frame}[fragile]
\frametitle{Snake Case Convention for Variables}
\begin{lstlisting}[language=Rust]
// Naming convention using snake_case for variables.
let student_count = 30; // Preferred over let StudentCount = 30;
\end{lstlisting}
\begin{itemize}
\item Emphasizes Rust's convention for variable naming.
\end{itemize}
\end{frame}

\begin{frame}[fragile]
\frametitle{Operations on Numbers in Different Formats}
\begin{lstlisting}[language=Rust]
// Demonstrate type casting and operations on variables.
let apples = 20;
let price_per_kg = 2.75;
let total_price = price_per_kg as i32 * apples;
println!("Total price for apples: {}", total_price);
\end{lstlisting}
\begin{itemize}
\item Demonstrates type casting and arithmetic operations.
\end{itemize}
\end{frame}

\begin{frame}[fragile]
\frametitle{Shadowing}
\begin{lstlisting}[language=Rust]
// Shadowing allows redeclaration of a variable.
// let rate = 10;
// let rate = rate * 2;
// println!("Updated rate: {}", rate);
\end{lstlisting}
\begin{itemize}
\item Shadowing allows the reuse of variable names.
\end{itemize}
\end{frame}

\begin{frame}[fragile]
\frametitle{Shadowing with \texttt{mut}}
\begin{lstlisting}[language=Rust]
// let mut temperature = 25;
// let temperature = temperature - 5;
// println!("Adjusted temperature: {}", temperature);
\end{lstlisting}
\begin{itemize}
\item Shadowing can also be used with mutable variables.
\end{itemize}
\end{frame}

\begin{frame}[fragile]
\frametitle{Changing the Type through Shadowing}
\begin{lstlisting}[language=Rust]
// let value = 100;
// println!("Value as integer: {}", value);
// let value = 'B';
// println!("Value as character: {}", value);
// let value = 22.9;
// println!("Value as float: {}", value);
\end{lstlisting}
\begin{itemize}
\item Demonstrates changing the type of a variable through shadowing.
\end{itemize}
\end{frame}

\begin{frame}[fragile]
\frametitle{Shadowing within Code Blocks}
\begin{lstlisting}[language=Rust]
let mut grade = 88;
{
	grade = 90; // Modifying variable inside a block.
	println!("Grade inside block: {}", grade);
}
println!("Grade outside block: {}", grade);
\end{lstlisting}
\begin{itemize}
\item Highlights the effect of shadowing within inner blocks.
\end{itemize}
\end{frame}

\begin{frame}[fragile]
\frametitle{Constants}
\begin{lstlisting}[language=Rust]
// Define a constant with uppercase and underscores.
const MAX_CAPACITY: u32 = 50_000;
println!("Maximum capacity: {}", MAX_CAPACITY);
}
\end{lstlisting}
\begin{itemize}
\item Constants are immutable and defined with \texttt{const} keyword.
\end{itemize}
\end{frame}

\begin{frame}[fragile]
\frametitle{Fixed Length Strings (\&str)}
\begin{lstlisting}[language=Rust]
// Declare a fixed-length string.
let fixed_str = "Immutable fixed length string";
println!("The text in the fixed string is \"{}\" ", fixed_str);
\end{lstlisting}
\begin{itemize}
\item Demonstrates declaration of a fixed-length string.
\item Immutable by nature.
\end{itemize}
\end{frame}

\begin{frame}[fragile]
\frametitle{Variable Length Strings (String)}
\begin{lstlisting}[language=Rust]
// Declare a mutable, growable string.
let mut dynamic_str = String::from("This string can expand");
println!("The initial text in the dynamic string is \"{}\" ", dynamic_str);
\end{lstlisting}
\begin{itemize}
\item Demonstrates declaration of a variable-length string.
\item Mutable and growable.
\end{itemize}
\end{frame}

\begin{frame}[fragile]
	\frametitle{Adding Characters to Strings}
	\begin{lstlisting}[language=Rust]
		// Add a character to the dynamic string.
		dynamic_str.push('!');
		println!("After adding a character, the string is \"{}\" ", dynamic_str);
		
		// Remove the last character from the dynamic string.
		dynamic_str.pop();
		println!("After removing the last character, the string is \"{}\" ", dynamic_str);
	\end{lstlisting}
	\begin{itemize}
		\item Shows adding a character.
		\item Demonstrates removing the last character.
	\end{itemize}
\end{frame}

\begin{frame}[fragile]
	\frametitle{Appending Substrings}
	\begin{lstlisting}[language=Rust]
		// Append a substring to the dynamic string.
		dynamic_str.push_str(" that can grow and shrink");
		println!("After adding more text, the string is \"{}\" ", dynamic_str);
	\end{lstlisting}
	\begin{itemize}
		\item Illustrates appending a substring.
	\end{itemize}
\end{frame}
		
\begin{frame}[fragile]
	\frametitle{String Functions}
	\begin{lstlisting}[language=Rust]
		// Demonstrate various string functions.
		println!(
		"String details:\n- Is the string empty? {}\n- Length of the string: {}\n- String capacity: {}\n- Does the string contain 'grow'? {}",
		dynamic_str.is_empty(),
		dynamic_str.len(),
		dynamic_str.capacity(),
		dynamic_str.contains("grow")
		);
	\end{lstlisting}
	\begin{itemize}
		\item Demonstrates string functions: \texttt{is\_empty}, \texttt{len}, \texttt{capacity}, \texttt{contains}.
	\end{itemize}
\end{frame}

\begin{frame}[fragile]
	\frametitle{Trimming Strings}
	\begin{lstlisting}[language=Rust]
		// Add spaces and demonstrate trimming.
		dynamic_str.push_str("   ");
		println!(
		"Length before trimming: {}, Length after trimming: {}",
		dynamic_str.len(),
		dynamic_str.trim().len()
		);
	\end{lstlisting}
	\begin{itemize}
		\item Illustrates trimming of spaces.
	\end{itemize}
\end{frame}


\begin{frame}[fragile]
	\frametitle{Number to String Conversion}
	\begin{lstlisting}[language=Rust]
		// Convert a number to a string and compare.
		let number = 42;
		println!("The number as a string: {}", number.to_string());
		println!("Does the string equal '42'? {}", number.to_string() == "42");
	\end{lstlisting}
	\begin{itemize}
		\item Shows conversion of number to string and comparison.
	\end{itemize}
\end{frame}

\begin{frame}[fragile]
	\frametitle{Character to String Conversion}
	\begin{lstlisting}[language=Rust]
		// Convert a character to a string and compare.
		let character = 'b';
		println!(
		"The character as a string: {}, Is it equal to 'b'? {}",
		character.to_string(),
		character.to_string() == "b"
		);
	\end{lstlisting}
	\begin{itemize}
		\item Demonstrates conversion of character to string and comparison.
	\end{itemize}
\end{frame}


\begin{frame}[fragile]
\frametitle{Creating a String from a Name}
\begin{lstlisting}[language=Rust]
// Create a string from a name.
let full_name = "John Doe".to_string();
println!("This string contains a name: {}", full_name);
\end{lstlisting}
\begin{itemize}
\item Demonstrates creating a string from a name.
\end{itemize}
\end{frame}

\begin{frame}[fragile]
\frametitle{Creating an Empty String}
\begin{lstlisting}[language=Rust]
// Create an empty string and check its length.
let empty_str = String::new();
println!("Length of the empty string: {}", empty_str.len());
\end{lstlisting}
\begin{itemize}
\item Shows how to create an empty string.
\item Checks the length of the empty string.
\end{itemize}
\end{frame}

\begin{frame}[fragile]
	\frametitle{Using \texttt{format!} to Combine Strings}
	\begin{lstlisting}[language=Rust]
		// Using \texttt{format!} to combine strings.
		let first_name = "John".to_string();
		let last_name = "Doe".to_string();
		let full_intro = format!("My first name is {}, and my last name is {}", first_name, last_name);
		println!("{}", full_intro);
	\end{lstlisting}
	\begin{itemize}
		\item Demonstrates using \texttt{format!} to combine strings.
	\end{itemize}
\end{frame}

\begin{frame}[fragile]
	\frametitle{Concatenating Strings with format!}
	\begin{lstlisting}[language=Rust]
		// Concatenating strings using \texttt{format!}.
		let part_1 = String::from("Hello");
		let part_2 = String::from(", World!");
		let combined_str = format!("{}{}", part_1, part_2);
		println!("The combined string is \"{}\"", combined_str);
	\end{lstlisting}
	\begin{itemize}
		\item Shows concatenating strings with \texttt{format!}.
	\end{itemize}
\end{frame}

\end{document}

