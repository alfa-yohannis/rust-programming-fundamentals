\documentclass[aspectratio=169, table]{beamer}

%\usepackage[beamertheme=./praditatheme]{Pradita}
\usepackage[utf8]{inputenc}
\usepackage{listings} 

\usetheme{Pradita}

\subtitle{IF120203-Programming Fundamentals}

\title{Session-03:\\\LARGE{Ownership in Rust\\}}
\date[Serial]{\scriptsize {PRU/SPMI/FR-BM-18/0222}}
\author[Pradita]{\small{\textbf{Alfa Yohannis}}}

\lstdefinelanguage{Rust} {
keywords={MAX, std, print, fn, let, mut, println, true, false, u8, u16, u32, u64, u128, i8, i16, i32, i64, i128, f32, f64, char, bool, if, else, for, while, loop, match, return},
basicstyle=\ttfamily\small,
keywordstyle=\color{blue}\bfseries,
ndkeywords={self, String, Option, Some, None, Result, Ok, Err},
ndkeywordstyle=\color{purple}\bfseries,
sensitive=true,
commentstyle=\color{gray},
stringstyle=\color{red},
numbers=left,
numberstyle=\tiny\color{gray},
breaklines=true,
frame=lines,
backgroundcolor=\color{lightgray!10},
tabsize=2,
comment=[l]{//},
morecomment=[s]{/*}{*/},
commentstyle=\color{gray}\ttfamily,
stringstyle=\color{purple}\ttfamily,
%morestring=[b]',
%morestring=[b]"
showstringspaces=false
}

\begin{document}

\frame{\titlepage}

\begin{frame}[fragile]
\frametitle{Rust Ownership}
\begin{itemize}
\item Every value in Rust has an associated variable known as its owner.
\item Only one owner exists at any given time.
\item When the owner exits the scope, the value is deallocated.
\end{itemize}
\end{frame}

\begin{frame}[fragile]
\frametitle{Basic Ownership Transfer}
\begin{lstlisting}[language=Rust]
// Ownership transfer example
let mut a = 45.8; 
let mut b = a;

let str1 = String::from("xyz"); 
let str2 = &str1; 
println!("The value of str1 = {} and str2 = {}", str1, str2);   
\end{lstlisting}
\begin{itemize}
\item Demonstrates variable ownership and transfer.
\item \texttt{a} is moved to \texttt{b}, and \texttt{str1} is borrowed by \texttt{str2}.
\end{itemize}
\end{frame}

\begin{frame}[fragile]
\frametitle{Ownership Transfer and Borrowing}
\vspace{12pt}
\begin{lstlisting}[language=Rust]
// Ownership and borrowing
let num_list1: Vec<i32> = vec![1, 2, 3, 4, 5];  
// let num_list2 = num_list1;                   // Ownership transfer
// println!("The first list is {:?} {:?}", num_list1, num_list2);

let num_list2 = &num_list1;                   // Borrowing
println!("The first list is {:?} {:?}", num_list1, num_list2);

let num_list2 = num_list1.clone();
println!("The first list is {:?} {:?}", num_list1, num_list2);
\end{lstlisting}
\begin{itemize}
\item Illustrates ownership transfer and borrowing.
\item \texttt{num\_list1} is either moved or borrowed.
\item \texttt{num\_list1} is cloned to \texttt{num\_list2} to demonstrate ownership duplication.
\end{itemize}
\end{frame}

\begin{frame}[fragile]
\frametitle{Scope and Deallocation}
\begin{lstlisting}[language=Rust]
// Variable scope and deallocation
{
	let user_name = String::from("John Doe"); 
} 
println!("User name is {}", user_name); 
\end{lstlisting}
\begin{itemize}
\item Shows the impact of variable scope on ownership.
\item \texttt{user\_name} is created and deallocated within a block.
\item Accessing \texttt{user\_name} outside its scope would result in an error.
\end{itemize}
\end{frame}

\begin{frame}[fragile]
\frametitle{Heap and Stack Overview}
\begin{itemize}
\item In Rust, variables can be stored either on the heap or the stack.
\item The stack is used for simple, fixed-size data, while the heap is used for dynamically-sized data.
\item Understanding heap and stack memory is crucial for efficient memory management.
\end{itemize}
\end{frame}

\begin{frame}[fragile]
\frametitle{Basic Code Example - Part 1}
\begin{lstlisting}[language=Rust]
const LIMIT:i32 = 50_000; 
fn main() {
	let (a, b) = (3, 7); 
	let total_value = compute_square_sum(a, b); 
	println!("The square of the sum is {}", total_value); 
} 
\end{lstlisting}
\begin{itemize}
\item Introduces a constant \texttt{LIMIT}.
\item \texttt{main} function shows variable assignment and function call.
\end{itemize}
\end{frame}

\begin{frame}[fragile]
\frametitle{Basic Code Example - Part 2}
\begin{lstlisting}[language=Rust]
fn compute_square_sum(val1: i32, val2: i32) -> i32 {
	let outcome = compute_square(val1 + val2); 
	outcome
}

fn compute_square(value: i32) -> i32 {
	value * value
}
\end{lstlisting}
\begin{itemize}
\item \texttt{compute\_square\_sum} calculates the square of the sum of two values.
\item \texttt{compute\_square} performs the squaring operation.
\end{itemize}
\end{frame}

\begin{frame}[fragile]
\frametitle{Heap and Stack in Action}
\begin{lstlisting}[language=Rust]
fn main() 
{
	let num: i32 = 10; 
	
	let str1 = String::from("another string");  
	let str2 = str1; 
	let str3 = &str2; 
	let str4 = str2.clone(); 
}
\end{lstlisting}
\begin{itemize}
\item Shows variable assignment and borrowing with heap and stack data.
\item \texttt{str1} is moved to \texttt{str2}, and \texttt{str2} is cloned to \texttt{str4}.
\item \texttt{str3} is a reference to \texttt{str2}.
\end{itemize}
\end{frame}

\begin{frame}[fragile]
\frametitle{Rust Ownership}
\begin{itemize}
\item Every value in Rust has a variable known as its owner.
\item Only one owner can exist at any given time.
\item When the owner exits the scope, the value will be deallocated.
\end{itemize}
\end{frame}

\begin{frame}[fragile]
\frametitle{Ownership and Functions - Part 1}
\begin{lstlisting}[language=Rust]
fn main() { 
	// Ownership and Functions
	/*
	let integer_value = 45;  
	let mut array_value = vec![7, 8, 9]; 
\end{lstlisting}
\begin{itemize}
	\item Introduces variables and functions related to ownership.
	\item \texttt{integer\_value} and \texttt{array\_value} are defined.
\end{itemize}
\end{frame}

\begin{frame}[fragile]
\frametitle{Ownership and Functions - Part 2}
\begin{lstlisting}[language=Rust]
	process_stack(integer_value); 
	println!("The stack variable was copied, and the original value was {}", integer_value);
	
	process_heap(&mut array_value);  
	println!("The variable after the function call is {:?}", array_value);  
	*/ 
}
\end{lstlisting}
\begin{itemize}
\item \texttt{process\_stack} shows copying of a variable.
\item \texttt{process\_heap} demonstrates borrowing a mutable reference.
\end{itemize}
\end{frame}


\begin{frame}[fragile]
\frametitle{Quiz}
\begin{lstlisting}[language=Rust]
// Quiz
let mut array_value = vec![7, 8, 9]; 
let reference1 = array_value;    
let reference2 = &reference1; 
\end{lstlisting}
\begin{itemize}
\item Illustrates variable assignment and referencing.
\item \texttt{reference1} takes ownership, and \texttt{reference2} borrows it immutably.
\end{itemize}
\end{frame}

\begin{frame}[fragile]
\frametitle{A Common Mistake}
\begin{lstlisting}[language=Rust]
// A Common Mistake
let mut array_value = vec![7, 8, 9]; 
let mut reference1 = &array_value; 

println!("Immutable references are {:?}", reference1); 
\end{lstlisting}
\begin{itemize}
\item Demonstrates a common mistake with mutable references.
\item Mutable references should not be used with immutable operations.
\end{itemize}
\end{frame}

\begin{frame}[fragile]
\frametitle{When References are Useful}
\begin{lstlisting}[language=Rust]
// When References are Useful
let long_text1 = String::from("This is the first lengthy string"); 
let long_text2 = String::from("This is the second lengthy string"); 

let combined_texts: Vec<&str> = vec![&long_text1, &long_text2];  
println!("The combined strings are {:?}", combined_texts);
\end{lstlisting}
\begin{itemize}
\item Shows the use of references to combine strings without ownership transfer.
\item \texttt{combined\_texts} holds references to \texttt{long\_text1} and \texttt{long\_text2}.
\end{itemize}
\end{frame}

\begin{frame}[fragile]
\frametitle{Function Definitions}
\vspace{12pt}
\begin{lstlisting}[language=Rust]
fn process_stack(mut value:i32)    
{   
	value = 78; 
	println!("The copied value has been updated to {}", value); 
}

fn process_heap(value:&mut Vec<i32>)    
{
	value.push(60);    
	println!("The vector value in the function is {:?}", value); 
}
\end{lstlisting}
\begin{itemize}
\item Defines functions for processing stack and heap values.
\item \texttt{process\_stack} works with a copied value, while \texttt{process\_heap} works with a mutable reference.
\end{itemize}
\end{frame}

\begin{frame}[fragile]
\frametitle{Rust Ownership}
\begin{itemize}
\item Every value in Rust has a variable known as its owner.
\item Only one owner can exist at any given time.
\item When the owner exits the scope, the value will be deallocated.
\end{itemize}
\end{frame}

\begin{frame}[fragile]
\frametitle{Ownership and Functions - Part 1}
\begin{lstlisting}[language=Rust]
// Ownership and Functions

let integer_value = 45;  
let mut array_value = vec![7, 8, 9]; 

process_stack(integer_value); 
println!("The stack variable was copied, and the original value was {}", integer_value);
\end{lstlisting}
\begin{itemize}
\item Introduces variables and functions related to ownership.
\item \texttt{integer\_value} and \texttt{array\_value} are defined.
\end{itemize}
\end{frame}

\begin{frame}[fragile]
\frametitle{Ownership and Functions - Part 2}
\begin{lstlisting}[language=Rust]
process_heap(&mut array_value);  
println!("The variable after the function call is {:?}", array_value);  

\end{lstlisting}
\begin{itemize}
\item \texttt{process\_stack} shows copying of a variable.
\item \texttt{process\_heap} demonstrates borrowing a mutable reference.
\end{itemize}
\end{frame}

\begin{frame}[fragile]
\frametitle{Quiz}
\begin{lstlisting}[language=Rust]
// Quiz
let mut array_value = vec![7, 8, 9]; 
let reference1 = array_value;    
let reference2 = &reference1; 
\end{lstlisting}
\begin{itemize}
\item Illustrates variable assignment and referencing.
\item \texttt{reference1} takes ownership, and \texttt{reference2} borrows it immutably.
\end{itemize}
\end{frame}

\begin{frame}[fragile]
\frametitle{A Common Mistake}
\begin{lstlisting}[language=Rust]
// A Common Mistake
let mut array_value = vec![7, 8, 9]; 
let mut reference1 = &array_value; 

println!("Immutable references are {:?}", reference1); 
\end{lstlisting}
\begin{itemize}
\item Demonstrates a common mistake with mutable references.
\item Mutable references should not be used with immutable operations.
\end{itemize}
\end{frame}

\begin{frame}[fragile]
\frametitle{When References are Useful}
\begin{lstlisting}[language=Rust]
// When References are Useful
let long_text1 = String::from("This is the first lengthy string"); 
let long_text2 = String::from("This is the second lengthy string"); 

let combined_texts: Vec<&str> = vec![&long_text1, &long_text2];  
println!("The combined strings are {:?}", combined_texts);
\end{lstlisting}
\begin{itemize}
\item Shows the use of references to combine strings without ownership transfer.
\item \texttt{combined\_texts} holds references to \texttt{long\_text1} and \texttt{long\_text2}.
\end{itemize}
\end{frame}

\begin{frame}[fragile]
\frametitle{Function Definitions - Part 1}
\begin{lstlisting}[language=Rust]
fn process_stack(mut value:i32)    
{   
	value = 78; 
	println!("The copied value of the variable has been updated to {}", value); 
}
\end{lstlisting}
\begin{itemize}
\item Defines \texttt{process\_stack} function for processing a copied stack value.
\end{itemize}
\end{frame}

\begin{frame}[fragile]
\frametitle{Function Definitions - Part 2}
\begin{lstlisting}[language=Rust]
fn process_heap(value:&mut Vec<i32>)    
{
	value.push(60);    
	println!("The value of the vector inside the function is {:?}", value); 
}
\end{lstlisting}
\begin{itemize}
\item Defines \texttt{process\_heap} function for processing a mutable reference to a heap value.
\end{itemize}
\end{frame}


\begin{frame}[fragile]
\frametitle{String Concatenation and Ownership}
\begin{itemize}
\item Concatenation in Rust can involve ownership transfer.
\item Understanding how ownership changes is crucial for managing strings.
\item Let's explore a few examples to see how ownership is affected.
\end{itemize}
\end{frame}

\begin{frame}[fragile]
\frametitle{Ownership Transfer Example 1}
\begin{lstlisting}[language=Rust]
let greeting = String::from("hi"); 
let target: &str = "there";

let combined = greeting + target;  // Ownership transferred here
println!("{}", combined);
\end{lstlisting}
\begin{itemize}
\item \texttt{greeting} is a \texttt{String} and \texttt{target} is a \texttt{\&str}.
\item The \texttt{+} operator transfers ownership of \texttt{greeting} to \texttt{combined}.
\item \texttt{target} is borrowed, but \texttt{greeting} is consumed.
\end{itemize}
\end{frame}

\begin{frame}[fragile]
\frametitle{Ownership Transfer Example 2}
\begin{lstlisting}[language=Rust]
let greeting = String::from("hi"); 
let place = String::from("there");

let combined = greeting + &place;  // Ownership of only greeting changed
println!("{} {}", combined, place); 
\end{lstlisting}
\begin{itemize}
\item \texttt{place} is a \texttt{String}, and \texttt{greeting} is moved into \texttt{combined}.
\item \texttt{place} remains valid because its ownership was not transferred.
\item \texttt{combined} contains the value of \texttt{greeting} concatenated with \texttt{place}.
\end{itemize}
\end{frame}

\begin{frame}[fragile]
\frametitle{Ownership Transfer Example 3}
\begin{lstlisting}[language=Rust]
let greeting = String::from("hi"); 
let place = String::from("there");
let addition = String::from(" from Rust");

let combined = greeting + \&place + &addition;  // Ownership of only greeting changed
println!("{} {} {}", combined, place, addition); 
\end{lstlisting}
\begin{itemize}
\item \texttt{greeting} is the only variable whose ownership is transferred.
\item \texttt{place} and \texttt{addition} are borrowed and remain valid.
\item \texttt{combined} contains the result of concatenating \texttt{greeting}, \texttt{place}, and \texttt{addition}.
\end{itemize}
\end{frame}


\end{document}
