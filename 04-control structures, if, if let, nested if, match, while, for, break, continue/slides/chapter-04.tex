\documentclass[aspectratio=169, table]{beamer}

\usepackage[utf8]{inputenc}
\usepackage{listings} 

\usetheme{Pradita}

\subtitle{IF120203-Programming Fundamentals}

\title{Session-04:\\\LARGE{Flow Control in Rust}\\ \vspace{10pt}}
\date[Serial]{\scriptsize {PRU/SPMI/FR-BM-18/0222}}
\author[Pradita]{\small{\textbf{Alfa Yohannis}}}

\lstdefinelanguage{Rust} {
keywords={MAX, std, print, fn, let, mut, println, true, false, u8, u16, u32, u64, u128, i8, i16, i32, i64, i128, f32, f64, char, bool, if, else, for, while, loop, match, return},
basicstyle=\ttfamily\small,
keywordstyle=\color{blue}\bfseries,
ndkeywords={self, String, Option, Some, None, Result, Ok, Err},
ndkeywordstyle=\color{purple}\bfseries,
sensitive=true,
commentstyle=\color{gray},
stringstyle=\color{red},
numbers=left,
numberstyle=\tiny\color{gray},
breaklines=true,
frame=lines,
backgroundcolor=\color{lightgray!10},
tabsize=2,
comment=[l]{//},
morecomment=[s]{/*}{*/},
commentstyle=\color{gray}\ttfamily,
stringstyle=\color{purple}\ttfamily,
showstringspaces=false
}

\begin{document}

\frame{\titlepage}

\begin{frame}[fragile]
\frametitle{Simple If Statement}
\begin{lstlisting}[language=Rust]
let value = 25;
if value < 30 {
println!("The value is less than 30");
}
println!("This line runs regardless of the above condition");
\end{lstlisting}
\begin{itemize}
\item Basic if statement checks if a condition is true.
\item The code block inside the if statement executes if the condition is true.
\item Code outside the if block runs regardless of the condition.
\end{itemize}
\end{frame}

\begin{frame}[fragile]
\frametitle{If with Compound Conditions}
\begin{lstlisting}[language=Rust]
let score = 75;
if score >= 70 && score <= 80 {
println!("The score is within the expected range");
}
\end{lstlisting}
\begin{itemize}
\item Using logical operators to combine conditions.
\item The code block executes if both conditions are true.
\item Helps in checking complex conditions.
\end{itemize}
\end{frame}

\begin{frame}[fragile]
\frametitle{If-Else Statement}
\begin{lstlisting}[language=Rust]
let score = 90;
if score > 60 {
println!("You have passed the test");
} else {
println!("You have failed the test");
}
\end{lstlisting}
\begin{itemize}
\item If-Else statement provides alternative execution paths.
\item The else block executes if the if condition is false.
\end{itemize}
\end{frame}

\begin{frame}[fragile]
\frametitle{If-Else If Ladder}
\vspace{25pt}
\begin{columns}[t,onlytextwidth]
\begin{column}{0.7\textwidth}
\begin{lstlisting}[language=Rust]
let test_score = 85;
let mut grade = 'U';
if test_score >= 80 {
	grade = 'A';
} else if test_score >= 70 {
	grade = 'B';
} else if test_score >= 60 {
	grade = 'C';
} else if test_score >= 50 {
	grade = 'D';
} else {
	grade = 'E';
}
println!("The assigned grade is {}", grade);
\end{lstlisting}
\end{column}
\begin{column}{0.3\textwidth}
\begin{itemize}
\item Checks multiple conditions sequentially.
\item Executes the block of the first true condition.
\item Optional else block runs if none are true.
\end{itemize}
\end{column}
\end{columns}
\end{frame}


\begin{frame}[fragile]
\frametitle{Nested If Statements}
\vspace{25pt}
\begin{columns}[t]
\begin{column}{0.25\textwidth}
\begin{itemize}
\item Allows checking multiple conditions.
\item Outer if checks the main condition.
\item Inner if-else handles detailed conditions.
\end{itemize}
\end{column}
\begin{column}{0.75\textwidth}
\begin{lstlisting}[language=Rust]
println!("Please input a number"); 
let mut input = String::new();
std::io::stdin().read_line(&mut input).expect("Error reading input.");
let input: i32 = input.trim().parse().expect("Invalid number");
if input != 0 {
	if input % 2 == 0 {
		println!("The number is even.");
	} else {
		println!("The number is odd.");
	}  
} else {
	println!("The number is zero, which is neither even nor odd.");
}
\end{lstlisting}
\end{column}
\end{columns}
\end{frame}


\begin{frame}[fragile]
\frametitle{If Let Statement}
\begin{columns}[t,onlytextwidth]
\begin{column}{0.6\textwidth}
\begin{lstlisting}[language=Rust]
let score = 75; 
let rank = if score >= 85 {  
	'A'
} else if score >= 75 {
	'B'
} else if score >= 65 { 
	'C'
} else if score >= 55 {
	'D'
} else {
	'F' // Else block is mandatory
};
println!("The achieved rank is {}", rank);
\end{lstlisting}
\end{column}
\begin{column}{0.4\textwidth}
\begin{itemize}
\item Simplifies variable assignments based on conditions.
\item All branches must return the same type.
\item The else block ensures the variable is always initialized.
\end{itemize}
\end{column}
\end{columns}
\end{frame}


\begin{frame}[fragile]
\frametitle{Simple Match Statement}
\begin{lstlisting}[language=Rust]
let value = 30; 
match value {
	1 | 2 => println!("Value is 1 or 2"),  
	3 | 4 => println!("Value is 3 or 4"), 
	5..=40 => println!("Value is between 5 and 40 inclusive"),       
	_ => println!("Value is greater than 40"), 
}
\end{lstlisting}
\begin{itemize}
\item Match statements compare a value against patterns.
\item Patterns can include single values, ranges, and multiple values.
\item The underscore (\_) acts as a default case.
\end{itemize}
\end{frame}

\begin{frame}[fragile]
\frametitle{Converting If-Else Ladder to Match}
\vspace{12pt}
\begin{lstlisting}[language=Rust]
let points = 72; 
let mut level = 'U'; 

match points {
	90..=100 => level = 'X', 
	80..=89  => level = 'Y', 
	70..=79  => level = 'Z',
	60..=69  => level = 'W',
	_ => level = 'V',
}
println!("Achieved level is {}", level);
\end{lstlisting}
\begin{itemize}
\item Match statements can simplify complex if-else ladders.
\item Each match arm assigns a value based on the pattern.
\item The default case handles any unmatched values.
\end{itemize}
\end{frame}

\begin{frame}[fragile]
\frametitle{Using Match in Place of If Let}
\begin{lstlisting}[language=Rust]
let points = 85; 
let level = match points {
90..=100 => 'X', 
80..=89  => 'Y', 
70..=79  => 'Z',
60..=69  => 'W',
_ => 'V',
};
println!("Achieved level is {}", level);	
\end{lstlisting}
\begin{itemize}
\item Matches can directly return values.
\item This allows for clean and concise code.
\item The returned value is based on the pattern matched.
\end{itemize}
\end{frame}

\begin{frame}[fragile]
\frametitle{While Loop Example 1}
\vspace{10pt}
\begin{columns}[t,onlytextwidth]
\begin{column}{0.7\textwidth}
\begin{lstlisting}[language=Rust]
let secret_num = 7;
println!("Guess the number (1-20)");
let mut correct = false;
while !correct {
	let mut input = String::new();
	std::io::stdin().read_line(&mut input).expect("Input error");
	let input: u8 = input.trim().parse().expect("Invalid input");
	if secret_num == input {
		println!("You guessed it!");
		correct = true;
	} else {
		println!("Try again!");
	}
}
\end{lstlisting}
\end{column}
\begin{column}{0.3\textwidth}
\begin{itemize}
\item Uses a while loop to ask for user input.
\item Checks if the input matches the secret number.
\item Continues until the correct guess is made.
\end{itemize}
\end{column}
\end{columns}
\end{frame}



\begin{frame}[fragile]
\frametitle{While Loop Example 2}
\vspace{10pt}
\begin{lstlisting}[language=Rust]
println!("Enter a number:");

let mut num = String::new();
std::io::stdin().read_line(&mut num).expect("Input error");
let mut num: u8 = num.trim().parse().expect("Invalid input");

num += 1;
while !(num % 2 == 0 && num % 5 == 0) {
	num += 1;
}
println!("Next divisible by 2 and 5: {}", num);
\end{lstlisting}
\begin{itemize}
\item Prompts the user for a number.
\item Finds the next number divisible by both 2 and 5.
\item Uses a while loop to increment the number until the condition is met.
\end{itemize}
\end{frame}

\begin{frame}[fragile]
\frametitle{For Loop with Index}
\begin{lstlisting}[language=Rust]
let numbers = vec![12, 24, 36, 48, 60, 72];

for index in 0..=5 {   // Range 0 to 5 inclusive
println!("Element {} at index {} is {}", index, index, numbers[index]);
}
\end{lstlisting}
\begin{itemize}
\item Iterates through indices of a vector.
\item Accesses elements using the index.
\item Useful for cases where both the index and value are needed.
\end{itemize}
\end{frame}

\begin{frame}[fragile]
\frametitle{For Loop with Ownership}
\begin{lstlisting}[language=Rust]
let values = vec![12, 24, 36, 48, 60, 72];
for value in values {
	println!("{}", value);
}
println!("{:?}", values);   // This line will cause an error as `values` is moved
\end{lstlisting}
\begin{itemize}
\item Iterates over vector elements with ownership.
\item After iteration, the vector is no longer available.
\item This example will fail due to the moved ownership of `values`.
\end{itemize}
\end{frame}

\begin{frame}[fragile]
\frametitle{For Loop with Immutable References}
\begin{lstlisting}[language=Rust]
let values = vec![12, 24, 36, 48, 60, 72];
for value in values.iter() {   // Iterate over immutable references
	println!("{}", value);
}
println!("{:?}", values);   // `values` remains valid here
\end{lstlisting}
\begin{itemize}
\item Uses `.iter()` to loop over immutable references.
\item Vector remains available after iteration.
\item Useful for read-only access to elements.
\end{itemize}
\end{frame}

\begin{frame}[fragile]
\frametitle{For Loop with Mutable References}
\begin{lstlisting}[language=Rust]
let mut numbers = vec![12, 24, 36, 48, 60, 72];
	for item in numbers.iter_mut() {   // Iterate over mutable references
	*item += 10;
	println!("{}", item);
}
println!("{:?}", numbers);
\end{lstlisting}
\begin{itemize}
\item Uses \texttt{.iter\_mut()} for mutable references.
\item Allows modifying elements in place.
\item After the loop, the vector reflects the changes.
\end{itemize}
\end{frame}

\begin{frame}[fragile]
\frametitle{Using Break to Exit a Loop}
\begin{lstlisting}[language=Rust]
let mut num = 100;
loop {
	num -= 1;
	if num % 13 == 0 {
		break;
	}
}
println!("The highest number less than the given number divisible by 13 is {}", num);
\end{lstlisting}
\begin{itemize}
\item Demonstrates using \texttt{break} to exit a loop early.
\item Stops when a condition is met (number divisible by 13).
\item Outputs the highest number below 100 that is divisible by 13.
\end{itemize}
\end{frame}

\begin{frame}[fragile]
\frametitle{Using Continue to Skip Iterations}
\vspace{10pt}
\begin{columns}
\column{.6\textwidth}
\begin{lstlisting}[language=Rust]
let mut num = 0;
let mut count = 0;
loop {
num += 1;
if num % 5 == 0 && 
num % 3 == 0 {
	println!("\nThe number divisible by both 3 and 5 is: {}\n", num);
	count += 1;
	if count == 3 {
		break;
	} else {
		continue;
	}
}
\end{lstlisting}

\column{.4\textwidth}
\begin{lstlisting}[language=Rust]
print!("{} ", num); // This line is skipped due to continue
}
\end{lstlisting}
\begin{itemize}
\item Uses \texttt{continue} to skip the rest of the loop iteration.
\item Prints numbers until the third divisible by both 3 and 5 is found.
\item Skips printing numbers that do not meet the condition.
\end{itemize}
\end{columns}
\end{frame}


\begin{frame}[fragile]
\frametitle{Returning Values with Loop}
\vspace{10pt}
\begin{columns}
\column{.6\textwidth}
\begin{lstlisting}[language=Rust]
let mut num = 0;
let mut count = 0;
let third_highest: i32 = loop {
	num += 1;
	if num % 5 == 0 && 
	num % 3 == 0 {
		println!("\nNumber divisible by 3 and 5: {}\n", num);
		count += 1;
		if count == 3 {
			break num;
		} else {
			continue;
		}
	}
\end{lstlisting}

\column{.4\textwidth}
\begin{lstlisting}[language=Rust]
	print!("{} ", num);
};
println!("The third highest number divisible by both 3 and 5 is {}", third_highest);
\end{lstlisting}

\begin{itemize}
\item Uses a loop to find numbers divisible by both 3 and 5.
\item Breaks the loop after finding the third such number.
\item Prints the third highest number found.
\end{itemize}
\end{columns}
\end{frame}


\begin{frame}[fragile]
\frametitle{Returning Values with Loop (Part 2)}
\begin{lstlisting}[language=Rust]
println!("The third highest number divisible by both 3 and 5 is {}", third_highest);
\end{lstlisting}
\begin{itemize}
\item Prints the third highest number found.
\end{itemize}
\end{frame}


\begin{frame}[fragile]
\frametitle{Collecting Student Marks (Part 1)}
\begin{lstlisting}[language=Rust]
let mut continue_input = true;
println!("Enter student marks:");
let mut student_grades = Vec::new(); 
\end{lstlisting}
\begin{itemize}
\item Initializes the loop for collecting user input.
\item Prepares a vector to store student grades.
\end{itemize}
\end{frame}

\begin{frame}[fragile]
\frametitle{Collecting Student Marks (Part 2)}
\begin{lstlisting}[language=Rust]
while continue_input {
	println!("Enter student's marks:");
	let mut input_marks = String::new();                                       
	std::io::stdin().read_line(&mut input_marks).unwrap();
	let marks: i32 = input_marks.trim().parse().unwrap(); 
	student_grades.push(marks); 
\end{lstlisting}
\begin{itemize}
\item Uses a loop to collect marks.
\item Reads and parses user input for student marks.
\end{itemize}
\end{frame}

\begin{frame}[fragile]
\frametitle{Collecting Student Marks (Part 3)}
\begin{lstlisting}[language=Rust]
	println!("Enter marks for another? [Y/N]"); 
	
	let user_choice: char = {
		let mut choice_input = String::new();                                
		std::io::stdin().read_line(&mut choice_input).unwrap();
		choice_input.trim().parse().unwrap()
	};
\end{lstlisting}
\begin{itemize}
\item Prompts the user to continue or stop.
\item Reads and parses the user's choice.
\end{itemize}
\end{frame}

\begin{frame}[fragile]
\frametitle{Collecting Student Marks (Part 4)}
\begin{lstlisting}[language=Rust]
	continue_input = user_choice == 'Y';
}

println!("Grades: {:?}", student_grades);
\end{lstlisting}
\begin{itemize}
\item Decides whether to continue or stop based on user input.
\item Displays the collected student grades.
\end{itemize}
\end{frame}

\begin{frame}[fragile]
\frametitle{Calculations with Sums and Squares (Part 1)}
\begin{lstlisting}[language=Rust]
// Read user input
let mut input = String::new();
std::io::stdin().read_line(&mut input).unwrap();
let n: i32 = input.trim().parse().unwrap();
\end{lstlisting}
\begin{itemize}
\item Reads a number from the user.
\end{itemize}
\end{frame}

\begin{frame}[fragile]
\frametitle{Calculations with Sums and Squares (Part 2)}
\begin{lstlisting}[language=Rust]
// Initialize variables
let (mut sum_of_numbers, mut sum_of_squares) = (0, 0);

// Loop through numbers
for i in 1..=n { 
	sum_of_numbers += i;   
	sum_of_squares += i.pow(2);  
}
\end{lstlisting}
\begin{itemize}
\item Initializes variables for calculations.
\item Computes the sum of numbers and their squares.
\end{itemize}
\end{frame}

\begin{frame}[fragile]
\frametitle{Calculations with Sums and Squares (Part 3)}
\begin{lstlisting}[language=Rust]
// Calculate difference
let difference = sum_of_numbers.pow(2) - sum_of_squares; 

// Print result
println!("Difference for N = {}: {}", n, difference);  
\end{lstlisting}
\begin{itemize}
\item Calculates the difference between the square of the sum and the sum of squares.
\item Outputs the result.
\end{itemize}
\end{frame}



\begin{frame}[fragile]
\frametitle{Handling Multiples and Sums (Part 1)}
\begin{lstlisting}[language=Rust]
// Read user input
let mut input = String::new();
std::io::stdin().read_line(&mut input).unwrap();
let n: i32 = input.trim().parse().unwrap();

// Initialize vectors
let mut multiples_of_3 = vec![0];
let mut multiples_of_5 = vec![0]; 
\end{lstlisting}
\begin{itemize}
\item Reads a number from the user.
\item Initializes vectors to store multiples.
\end{itemize}
\end{frame}

\begin{frame}[fragile]
\frametitle{Handling Multiples and Sums (Part 2)}
\begin{lstlisting}[language=Rust]
// Populate vectors
for i in 1..n {
	multiples_of_3.push((i % 3 == 0) as i32);
	multiples_of_5.push((i % 5 == 0) as i32);
}
// Combine multiples
let mut combined_list = vec![0]; 
for i in 1..n as usize {
	combined_list.push((multiples_of_3[i] | multiples_of_5[i]) as i32);
}	
\end{lstlisting}
\begin{itemize}
\item Populates vectors with multiples of 3 and 5.
\item Combines multiples of 3 and 5.
\end{itemize}
\end{frame}

\begin{frame}[fragile]
\frametitle{Handling Multiples and Sums (Part 3)}
\begin{lstlisting}[language=Rust]
// Compute values of multiples
let values_of_multiples: Vec<_> = (1..=n).map(|i| combined_list[i as usize] * i).collect();

// Print results
println!("Multiples: {:?}", values_of_multiples);  
println!("Sum: {:?}", values_of_multiples.iter().sum::<i32>());
\end{lstlisting}
\begin{itemize}
\item Computes values of multiples.
\item Prints the multiples.
\item Prints the sum of multiples.
\end{itemize}
\end{frame}

\begin{frame}[fragile]
\frametitle{Palindrome Check (Part 1)}
\begin{lstlisting}[language=Rust]
// Initialize string and flags
let input = "abbbbaa".to_string();
let mut is_palindrome = true;

// Check empty string case
if input.is_empty() {
	println!("\n\nPalindrome: {:?}", is_palindrome);
	return;
}
\end{lstlisting}
\begin{itemize}
\item Initialize string and palindrome flag.
\item Handle empty string separately.
\end{itemize}
\end{frame}

\begin{frame}[fragile]
\frametitle{Palindrome Check (Part 2)}
\begin{lstlisting}[language=Rust]
// Compare characters from both ends
let mut chars = input.chars().peekable();
while let (Some(front), Some(back)) = (chars.next(), chars.next_back()) {
	if front != back {
		is_palindrome = false;
		break;
	}
}
// Output result
println!("\n\nPalindrome: {:?}", is_palindrome);
\end{lstlisting}
\begin{itemize}
\item Compare characters from both ends of the string.
\item Output the result.
\end{itemize}
\end{frame}

\begin{frame}[fragile]
\frametitle{Finding Pythagorean Triplets}
\vspace{20pt}
\begin{columns}[t]
\begin{column}{0.3\textwidth}
\begin{itemize}
\item Searches for Pythagorean triplets (a, b, c) such that \(a^2 + b^2 = c^2\) and \(a + b + c = 1000\).
\item Uses nested loops to explore possible values for \(a\), \(b\), and \(c\).
\item Prints the triplet and terminates the program once found.
\end{itemize}
\end{column}

\begin{column}{0.7\textwidth}
\begin{lstlisting}[language=Rust]
// Iterate over possible values of a, b, and c
for a in 1..=1000 {
	for b in a + 1..1000 {
		for c in b + 1..1000 {
			if a * a + b * b == c * c && a + b + c == 1000 {
				println!("The Pythagorean triplet is ({}, {}, {})", a, b, c);
				return;
			}
		}
	}
}
\end{lstlisting}
\end{column}
\end{columns}
\end{frame}


\end{document}
