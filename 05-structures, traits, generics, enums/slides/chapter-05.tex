\documentclass[aspectratio=169, table]{beamer}

\usepackage[utf8]{inputenc}
\usepackage{listings} 

\usetheme{Pradita}

\subtitle{IF120203-Programming Fundamentals}

\title{Session-05:\\\LARGE{Structures in Rust}\\ \vspace{10pt}}
\date[Serial]{\scriptsize {PRU/SPMI/FR-BM-18/0222}}
\author[Pradita]{\small{\textbf{Alfa Yohannis}}}

\lstdefinelanguage{Rust} {
keywords={MAX, std, print, fn, let, mut, println, true, false, u8, u16, u32, u64, u128, i8, i16, i32, i64, i128, f32, f64, char, bool, if, else, for, while, loop, match, return},
basicstyle=\ttfamily\small,
keywordstyle=\color{blue}\bfseries,
ndkeywords={self, String, Option, Some, None, Result, Ok, Err},
ndkeywordstyle=\color{purple}\bfseries,
sensitive=true,
commentstyle=\color{gray},
stringstyle=\color{red},
numbers=left,
numberstyle=\tiny\color{gray},
breaklines=true,
frame=lines,
backgroundcolor=\color{lightgray!10},
tabsize=2,
comment=[l]{//},
morecomment=[s]{/*}{*/},
commentstyle=\color{gray}\ttfamily,
stringstyle=\color{purple}\ttfamily,
showstringspaces=false
}

\begin{document}

\frame{\titlepage}

\begin{frame}[fragile]
\frametitle{Structure Definition: Employee}
\begin{lstlisting}[language=Rust]
struct Employee {
	nationality: String,
	full_name: String,
	years: i32,
	sex: char,
	income: i32,
}
\end{lstlisting}
\begin{itemize}
\item Defines the structure of an \texttt{Employee}.
\item Includes fields for personal and financial details.
\end{itemize}
\end{frame}

\begin{frame}[fragile]
\frametitle{Implementation of Employee: Initialization}
\vspace{15pt}
\begin{lstlisting}[language=Rust]
impl Employee {
// Initialize an Employee with default values
	fn create_default() -> Employee {
		Employee {
			nationality: String::from("Canada"),
			full_name: String::from("John Doe"),
			years: 30,
			sex: 'F',
			income: 50_000,
		}
	}
\end{lstlisting}
\begin{itemize}
\item Implements methods for the \texttt{Employee} structure.
\item \texttt{create\_default} initializes with default values.
\end{itemize}
\end{frame}

\begin{frame}[fragile]
\frametitle{Implementation of Employee: Tax Calculation}
\begin{lstlisting}[language=Rust]
	// Calculate tax based on income
	fn calculate_tax(&self) -> f32 {
		(self.income as f32 / 4.0) * 0.4
	}
}
\end{lstlisting}
\begin{itemize}
\item \texttt{calculate\_tax} computes tax based on income.
\item Returns the calculated tax as a floating-point number.
\end{itemize}
\end{frame}

\begin{frame}[fragile]
\frametitle{Main Function: Employee Creation (Part 1)}
\vspace{20pt}
\begin{lstlisting}[language=Rust]
fn main() {
	// Create an Employee instance with specified values
	let employee1 = Employee {
		full_name: String::from("Alice Smith"),
		nationality: String::from("UK"),
		years: 35,
		sex: 'F',
		income: 60_000,
	};
\end{lstlisting}
\begin{itemize}
\item Creates an \texttt{Employee} instance with specific values.
\end{itemize}
\end{frame}

\begin{frame}[fragile]
\frametitle{Main Function: Employee Usage (Part 2)}
\begin{lstlisting}[language=Rust]
// Print employee details and calculated tax
	println!("Employee details: {} {} {}", employee1.nationality, employee1.years, employee1.sex);
	println!("Tax for {} is {}", employee1.full_name, employee1.calculate_tax());
}
\end{lstlisting}
\begin{itemize}
\item Prints details and tax for the \texttt{Employee}.
\end{itemize}
\end{frame}

\begin{frame}[fragile]
\frametitle{Main Function: Employee Creation (Part 1)}
\begin{lstlisting}[language=Rust]
// Create an Employee with default values
let employee2 = Employee::create_default();
println!("Default employee: Name {}, Nationality {}", employee2.full_name, employee2.nationality);
\end{lstlisting}
\begin{itemize}
\item Demonstrates creating an \texttt{Employee} with default values.
\item Shows how to use a method for default initialization.
\end{itemize}
\end{frame}

\begin{frame}[fragile]
\frametitle{Main Function: Employee Field Updates (2)}
\begin{lstlisting}[language=Rust]
// Create and modify an Employee instance using fields from another instance
let employee3 = Employee {
	years: 45,
	full_name: String::from("Robert Brown"),
	..employee1
};
println!("Updated employee: Name = {}, Salary = {}", employee3.full_name, employee3.income);
\end{lstlisting}
\begin{itemize}
\item Shows how to create an \texttt{Employee} instance with updated fields.
\item Demonstrates field updates using another instance.
\end{itemize}
\end{frame}

\begin{frame}[fragile]
\frametitle{Main Function: Employee Modification (3)}
\begin{lstlisting}[language=Rust]
// Modify and display another Employee instance
let mut employee4 = Employee::create_default();
println!("Initial name of employee 4: {}", employee4.full_name);
employee4.full_name = String::from("David Wilson");
println!("Updated name of employee 4: {}", employee4.full_name);
\end{lstlisting}
\begin{itemize}
\item Demonstrates modifying an existing \texttt{Employee} instance.
\item Shows how to update fields after creation.
\end{itemize}
\end{frame}


\begin{frame}[fragile]
\frametitle{Tuple Structure: Pair}
\vspace{20pt}
\begin{lstlisting}[language=Rust]
struct Pair(i32, i32);
impl Pair {
	// Method to get the larger value
	fn larger(&self) -> i32 {
		if self.0 >= self.1 { self.0 } else { self.1 }
	}
	// Method to get the smaller value
	fn smaller(&self) -> i32 {
		if self.0 < self.1 { self.0 } else { self.1 }
	}
}
\end{lstlisting}
\begin{itemize}
\item Defines a tuple structure \texttt{Pair}.
\item Methods \texttt{larger} and \texttt{smaller} return respective values.
\end{itemize}
\end{frame}

\begin{frame}[fragile]
\frametitle{Main Function: Pair Usage}
\begin{lstlisting}[language=Rust]
fn main() {
	// Create a Pair instance and display values
	let numbers = Pair(45, 23);
	println!("Numbers are: {} and {}", numbers.0, numbers.1);
	println!("Larger number: {}", numbers.larger());
	println!("Smaller number: {}", numbers.smaller());
}
\end{lstlisting}
\begin{itemize}
\item Creates a \texttt{Pair} instance with two integers.
\item Displays the values and methods' results.
\end{itemize}
\end{frame}


\begin{frame}[fragile]
\frametitle{Structure Definition: Employee}
\begin{lstlisting}[language=Rust]
struct Employee {
	nationality: String,
	full_name: String,
	years: u8,
	gender: char,
	salary: i32,
}
\end{lstlisting}
\begin{itemize}
\item Defines the structure of an \texttt{Employee}.
\item Includes personal and financial details.
\end{itemize}
\end{frame}

\begin{frame}[fragile]
\frametitle{Trait Definition: PersonalDetails}
\begin{lstlisting}[language=Rust]
trait PersonalDetails {
	fn get_details(&self) -> (&str, u8, char);
	
	fn get_nation(&self) -> &str;
}
\end{lstlisting}
\begin{itemize}
\item Defines the \texttt{PersonalDetails} trait.
\item Contains methods for retrieving details and nationality.
\end{itemize}
\end{frame}

\begin{frame}[fragile]
\frametitle{Trait Implementation for Employee: Part 1}
\begin{lstlisting}[language=Rust]
impl PersonalDetails for Employee {
fn get_details(&self) -> (&str, u8, char) {
(&self.full_name, self.years, self.gender)
}

fn get_nation(&self) -> &str {
&self.nationality
}
}
\end{lstlisting}
\begin{itemize}
\item Implements \texttt{PersonalDetails} for \texttt{Employee}.
\item Provides methods to retrieve details and nationality.
\end{itemize}
\end{frame}

\begin{frame}[fragile]
\frametitle{Trait Implementation for Learner: Part 2}
\begin{lstlisting}[language=Rust]
impl PersonalDetails for Learner {
fn get_details(&self) -> (&str, u8, char) {
(&self.student_name, self.years, self.gender)
}

fn get_nation(&self) -> &str {
&self.nation
}
}
\end{lstlisting}
\begin{itemize}
\item Implements \texttt{PersonalDetails} for \texttt{Learner}.
\item Provides methods to retrieve details and nationality.
\end{itemize}
\end{frame}


\begin{frame}[fragile]
\frametitle{Main Function: Using Traits (Part 1)}
\vspace{20pt}
\begin{lstlisting}[language=Rust]
fn main() {
	let emp1 = Employee {
		full_name: String::from("John Doe"),
		nationality: String::from("Canada"),
		years: 45, gender: 'M', salary: 55_000,
	};	
	let learner1 = Learner {
		student_name: String::from("Jane Doe"),
		years: 20, gender: 'F',
		nation: String::from("Australia"),
	};
\end{lstlisting}
\begin{itemize}
\item Demonstrates creating instances of \texttt{Employee} and \texttt{Learner}.
\item Sets up data for displaying using the \texttt{PersonalDetails} trait.
\end{itemize}
\end{frame}

\begin{frame}[fragile]
\frametitle{Main Function: Using Traits (Part 2)}
\begin{lstlisting}[language=Rust]
	println!("Employee details: {:?}", emp1.get_details());
	println!("Learner details: {:?}", learner1.get_details());
}
\end{lstlisting}
\begin{itemize}
\item Uses the \texttt{get\_details} method from the \texttt{PersonalDetails} trait.
\item Outputs details for both \texttt{Employee} and \texttt{Learner}.
\end{itemize}
\end{frame}


\begin{frame}[fragile]
\frametitle{Trait Definition: ShapeMetrics}
\begin{lstlisting}[language=Rust]
trait ShapeMetrics {
	fn compute_area(&self);
	
	fn compute_perimeter(&self) {
		println!("Perimeter calculation not provided for this shape.");
	}
}
\end{lstlisting}
\begin{itemize}
\item Defines the \texttt{ShapeMetrics} trait.
\item Includes methods for calculating area and perimeter.
\end{itemize}
\end{frame}

\begin{frame}[fragile]
\frametitle{Trait Implementation: Sphere}
\vspace{15pt}
\begin{lstlisting}[language=Rust]
impl ShapeMetrics for Sphere {
	fn compute_area(&self) {
		let area_of_sphere = 4.0 * 3.14 * (self.radius * self.radius);
		println!("The surface area of the sphere is {}", area_of_sphere);
	}
	fn compute_perimeter(&self) {
		let circumference = 2.0 * 3.14 * self.radius;
		println!("The circumference of the sphere is {}", circumference);
	}
}
\end{lstlisting}
\begin{itemize}
\item Implements \texttt{ShapeMetrics} for \texttt{Sphere}.
\item Calculates the surface area and circumference of the sphere.
\end{itemize}
\end{frame}

\begin{frame}[fragile]
\frametitle{Trait Implementation: Square}
\vspace{20pt}
\begin{lstlisting}[language=Rust]
impl ShapeMetrics for Square {
	fn compute_area(&self) {
		let area_of_square = self.side_length * self.side_length;
		println!("The area of the square is {}", area_of_square);
	}
	fn compute_perimeter(&self) {
		let perimeter_of_square = 4.0 * self.side_length;
		println!("The perimeter of the square is {}", perimeter_of_square);
	}
}
\end{lstlisting}
\begin{itemize}
\item Implements \texttt{ShapeMetrics} for \texttt{Square}.
\item Calculates the area and perimeter of the square.
\end{itemize}
\end{frame}


\begin{frame}[fragile]
\frametitle{Main Function: Shape Metrics}
\vspace{20pt}
\begin{lstlisting}[language=Rust]
fn main() {
	let sphere = Sphere { radius: 4.0 };
	let square = Square { side_length: 6.0 };
	
	sphere.compute_area();
	sphere.compute_perimeter();
	
	square.compute_area();
	square.compute_perimeter();
}
\end{lstlisting}
\begin{itemize}
\item Creates instances of \texttt{Sphere} and \texttt{Square}.
\item Demonstrates the use of methods to calculate area and perimeter.
\end{itemize}
\end{frame}

\begin{frame}[fragile]
\frametitle{Trait Definition: Statistical}
\begin{lstlisting}[language=Rust]
struct Metrics {
	values: Vec<i32>,
}

trait Statistical {
	fn average(&self) -> f32; 
	fn dispersion(&self) -> f32;  
}
\end{lstlisting}
\begin{itemize}
\item Defines a trait \texttt{Statistical}.
\item Includes methods for computing average and dispersion.
\end{itemize}
\end{frame}

\begin{frame}[fragile]
\frametitle{Trait Implementation: Average Calculation}
\begin{lstlisting}[language=Rust]
impl Statistical for Metrics {
	fn average(&self) -> f32 {
	let mut total = 0; 
	for value in self.values.iter() {
		total += *value; 
	}
	total as f32 / self.values.len() as f32
}
\end{lstlisting}
\begin{itemize}
\item Implements \texttt{average} for \texttt{Metrics}.
\item Computes the mean of the values in the dataset.
\end{itemize}
\end{frame}

\begin{frame}[fragile]
\frametitle{Trait Implementation: Dispersion Calculation}
\vspace{20pt}
\begin{lstlisting}[language=Rust]
fn dispersion(&self) -> f32 {
	let avg = self.average();  
	let mut squared_diff_sum: f32 = 0.0;
	for value in self.values.iter() {
		squared_diff_sum += (*value as f32 - avg) * (*value as f32 - avg);
	} 
		squared_diff_sum / self.values.len() as f32
	}
}
\end{lstlisting}
\begin{itemize}
\item Implements \texttt{dispersion} for \texttt{Metrics}.
\item Calculates variance based on the average value.
\end{itemize}
\end{frame}

\begin{frame}[fragile]
\frametitle{Main Function: Using the Trait}
\begin{lstlisting}[language=Rust]
fn main() {
	let dataset = Metrics {
	values: vec![4, 7, 5, 9, 8, 6, 10],
	};
	println!("The average of the dataset is {}", dataset.average()); 
	println!("The variance of the dataset is {}", dataset.dispersion()); 
}
\end{lstlisting}
\begin{itemize}
\item Creates an instance of \texttt{Metrics} with sample data.
\item Demonstrates the use of \texttt{average} and \texttt{dispersion} methods.
\end{itemize}
\end{frame}

\begin{frame}[fragile]
\frametitle{Enum Definition: Transportation}
\begin{lstlisting}[language=Rust]
enum Transportation {
	Bike, 
	Bus,
	Plane,
}
\end{lstlisting}
\begin{itemize}
\item Defines an enum `Transportation` with variants for different transport modes.
\item Each variant represents a mode of transportation: Bike, Bus, and Plane.
\end{itemize}
\end{frame}

\begin{frame}[fragile]
\frametitle{Reimbursement Calculation Method}
\vspace{20pt}
\begin{lstlisting}[language=Rust]
impl Transportation {
	fn reimbursement(&self, distance: i32) -> f32 {
		let rate = match self {
			Transportation::Bike => distance as f32 * 12.0 * 1.5,  
			Transportation::Bus => distance as f32 * 16.0 * 1.5,
			Transportation::Plane => distance as f32 * 25.0 * 1.5,
		}; 
		rate
	}
}
\end{lstlisting}
\begin{itemize}
\item Implements the `reimbursement` method to calculate travel expenses based on distance and transportation type.
\item Demonstrates how the method is used with different transportation modes.
\end{itemize}
\end{frame}

\begin{frame}[fragile]
\frametitle{Using Enums: Example 1}
\vspace{15pt}
\begin{lstlisting}[language=Rust]
fn main() {
	let traveler1 = Transportation::Bike;
	println!("The enum value for traveler1 {}", traveler1 as i32);
	let traveler2 = Transportation::Plane; 
	let traveler3 = Transportation::Bus;
	println!("Traveler 1 is entitled to a reimbursement of {}", traveler1.reimbursement(70)); 
	println!("Traveler 2 is entitled to a reimbursement of {}", traveler2.reimbursement(150)); 
	println!("Traveler 3 is entitled to a reimbursement of {}", traveler3.reimbursement(70)); 
}
\end{lstlisting}
\begin{itemize}
\item Demonstrates using \texttt{Transportation} enum.
\item Prints reimbursement for different types of transportation.
\end{itemize}
\end{frame}

\begin{frame}[fragile]
\frametitle{Enum for Transport with Associated Data}
\begin{lstlisting}[language=Rust]
enum Transport {
	Bicycle(i32), 
	Coach(i32),
	Aircraft(i32),
}
\end{lstlisting}
\begin{itemize}
\item Defines an enum `Transport` with associated data for different transport modes.
\item Each variant carries a distance value of type `i32`.
\end{itemize}
\end{frame}

\begin{frame}[fragile]
\frametitle{Calculating Reimbursement}
\vspace{15pt}
\begin{lstlisting}[language=Rust]
impl Transport {
	fn reimbursement(&self) -> f32 {
		let rate = match self {
				Transport::Bicycle(distance) => *distance as f32 * 12.0 * 1.5,  
				Transport::Coach(distance) => *distance as f32 * 16.0 * 1.5,
				Transport::Aircraft(distance) => *distance as f32 * 25.0 * 1.5,
		}; 
		rate }}
\end{lstlisting}
\begin{itemize}
\item Implements a method `reimbursement` to calculate travel reimbursement based on distance.
\item Demonstrates usage of the `Transport` enum and its `reimbursement` method.
\end{itemize}
\end{frame}


\begin{frame}[fragile]
\frametitle{Using Enums with Data: Example 2}
\vspace{15pt}
\begin{lstlisting}[language=Rust]
fn main() {
	let commuter1 = Transport::Bicycle(70); 
	let commuter2 = Transport::Aircraft(150); 
	let commuter3 = Transport::Coach(70); 
	
	println!("Commuter 1 is entitled to a reimbursement of {}", commuter1.reimbursement()); 
	println!("Commuter 2 is entitled to a reimbursement of {}", commuter2.reimbursement()); 
	println!("Commuter 3 is entitled to a reimbursement of {}", commuter3.reimbursement()); 
}
\end{lstlisting}
\begin{itemize}
\item Demonstrates using \texttt{Transport} with associated data.
\item Prints reimbursement for different transport modes.
\end{itemize}
\end{frame}

\begin{frame}[fragile]
\frametitle{Enum Definition: Mixed Data Types}
\begin{lstlisting}[language=Rust]
#[derive(Debug)]
enum Variant {
	Integer(i32), 
	Decimal(f32),
}
\end{lstlisting}
\begin{itemize}
\item Defines an enum `Variant` with two variants: `Integer` and `Decimal`.
\item Uses `\#[derive(Debug)]` to enable easy debugging output.
\end{itemize}
\end{frame}

\begin{frame}[fragile]
\frametitle{Using Enums in a Vector}
\begin{lstlisting}[language=Rust]
fn main() {
	let values = vec![Variant::Integer(42), Variant::Decimal(22.5)]; 
	println!("\n\nThe integer value is {:?}, the float value is {:?}", values[0], values[1]); 
\end{lstlisting}
\begin{itemize}
	\item Creates a vector containing different `Variant` enum instances.
	\item Demonstrates accessing and printing enum values.
\end{itemize}
\end{frame}

\begin{frame}[fragile]
\frametitle{Processing Enums with Match}
\begin{lstlisting}[language=Rust]
	for item in values.iter() { 
		match item {
			Variant::Integer(num) => println!("The value is an integer: {}", num), 
			Variant::Decimal(num) => println!("The value is a float: {}", num),  
		}
	}
}
\end{lstlisting}
\begin{itemize}
\item Uses `match` to handle different enum variants.
\item Demonstrates processing and outputting mixed data types from a vector.
\end{itemize}
\end{frame}


\begin{frame}[fragile]
\frametitle{Generic Functions: Initial Approach}
\vspace{20pt}
\begin{lstlisting}[language=Rust]
fn square_int(n: i32) -> i32 {
	n * n
}
fn square_float(n: f32) -> f32 {
	n * n
}

fn main() {
	println!("The square of the integer is {}", square_int(5));
	println!("The square of the float is {}", square_float(5.0));
}
\end{lstlisting}
\begin{itemize}
\item Two separate functions for different types.
\item Code duplication for each type.
\end{itemize}
\end{frame}

\begin{frame}[fragile]
\frametitle{Generic Function: Simplified}
\vspace{10pt}
\begin{lstlisting}[language=Rust]
fn square<T>(n: T) -> T
where T: std::ops::Mul<Output = T> + Copy {
	n * n
}

fn main() {
	println!("The square of the integer is {}", square(5));
	println!("The square of the float is {}", square(5.5));
}
\end{lstlisting}
\begin{itemize}
\item Uses generics to handle multiple types.
\item Reduces code duplication by generalizing function.
\end{itemize}
\end{frame}

\begin{frame}[fragile]
\frametitle{Generic Struct: Initial Approach}
\begin{lstlisting}[language=Rust]
struct Coordinate {
	x: i32,
	y: i32,
}

fn main() {
	let coord1 = Coordinate { x: 5, y: 5 };
	let coord2 = Coordinate { x: 1.0, y: 4.0 }; // Error: mismatched types
}
\end{lstlisting}
\begin{itemize}
\item Struct with fixed types.
\item Type mismatch error when using different types.
\end{itemize}
\end{frame}

\begin{frame}[fragile]
\frametitle{Generic Struct: Improved}
\begin{lstlisting}[language=Rust]
struct Coordinate<T> {
	x: T,
	y: T,
}

fn main() {
	let coord1 = Coordinate { x: 5, y: 5 };
	let coord2 = Coordinate { x: 1.0, y: 4.0 };
	// let coord3 = Coordinate { x: 5, y: 5.0 }; // Error: mismatched types
}
\end{lstlisting}
\begin{itemize}
\item Uses generics to allow different types.
\item Both fields must be of the same type.
\end{itemize}
\end{frame}

\begin{frame}[fragile]
\frametitle{Generic Struct: Basic Definition}
\begin{lstlisting}[language=Rust]
struct Coordinate<T, U> {
	x: T,
	y: U,
}
\end{lstlisting}
\begin{itemize}
\item Defines a generic struct `Coordinate` with two different types for its fields.
\item `T` and `U` represent different types that can be used for `x` and `y`.
\end{itemize}
\end{frame}

\begin{frame}[fragile]
\frametitle{Generic Struct: Method Implementation}
\begin{lstlisting}[language=Rust]
impl<T: std::fmt::Debug, U: std::fmt::Debug> Coordinate<T, U> {
	fn display(&self) {
		println!("The values are {:?} and {:?}", self.x, self.y);
	}
}
\end{lstlisting}
\begin{itemize}
\item Implements a method `display` for the `Coordinate` struct.
\item Uses Rust's `Debug` trait to format and print the values of `x` and `y`.
\end{itemize}
\end{frame}

\begin{frame}[fragile]
\frametitle{Generic Struct: Usage Example}
\begin{lstlisting}[language=Rust]
fn main() {
	let coord1 = Coordinate { x: 5, y: 5 };
	coord1.display();
	let coord2 = Coordinate { x: 1.0, y: 4.0 };
	let coord3 = Coordinate { x: 5, y: 5.0 };
	coord3.display();
}
\end{lstlisting}
\begin{itemize}
\item Demonstrates creating instances of `Coordinate` with different types.
\item Shows how the `display` method works with various type combinations.
\end{itemize}
\end{frame}


\begin{frame}[fragile]
\frametitle{Option Enum: Syntax}
\begin{lstlisting}[language=Rust]
enum Option<T> {
	None,
	Some(T),
}
\end{lstlisting}
\begin{itemize}
\item `Option` is an enum used for optional values.
\item Can be either `None` or `Some(value)`.
\end{itemize}
\end{frame}

\begin{frame}[fragile]
\frametitle{Example 1: Basic Usage}
\begin{lstlisting}[language=Rust]
fn main() {
	let mut health_condition: Option<String> = None;
	health_condition = Some(String::from("Hypertension"));
	
	match health_condition {
		Some(condition) => println!("You have the health condition: {}", condition),
		None => println!("You have no health conditions"),
	}
}
\end{lstlisting}
\begin{itemize}
\item Demonstrates how to use `Option` with `String`.
\item Uses `match` to handle `Some` and `None` cases.
\end{itemize}
\end{frame}

\begin{frame}[fragile]
\frametitle{Options with Various Types (Part 1)}
\begin{lstlisting}[language=Rust]
struct Individual {
	name: String,
	age: i32,
}

fn main() {
	let str_option: Option<&str> = Some("Sample Text");
	
	println!("\n The value of str_option is {:?}\n The actual value is {:?} \n\n", str_option, str_option.unwrap());
\end{lstlisting}
\begin{itemize}
	\item Shows how `Option` can wrap different types like `String`.
	\item Demonstrates unwrapping `Option` values and accessing their content.
\end{itemize}
\end{frame}

\begin{frame}[fragile]
\frametitle{Options with Various Types (Part 2)}
\begin{lstlisting}[language=Rust]
	let float_option: Option<f64> = Some(12.34);
	
	let mut sum = 20.0;
	sum += float_option.unwrap();
	println!("The total value is {}", sum);
\end{lstlisting}
\begin{itemize}
	\item Demonstrates using `Option` with floating-point numbers.
	\item Shows how to perform operations on `Option` values by unwrapping them.
\end{itemize}
\end{frame}

\begin{frame}[fragile]
\frametitle{Options with Various Types (Part 3)}
\begin{lstlisting}[language=Rust]
	let vec_option: Option<Vec<i32>> = Some(vec![1, 2, 3, 4]);
	
	let person = Individual {
		name: String::from("Alex"),
		age: 28,
	};
	
	let person_option: Option<Individual> = Some(person);
}
\end{lstlisting}
\begin{itemize}
\item Shows `Option` wrapping more complex types like `Vec` and custom structs.
\item Highlights the versatility of `Option` in handling optional values of diverse types.
\end{itemize}
\end{frame}


\begin{frame}[fragile]
\frametitle{Function Example: Using Option (Part 1)}
\vspace{15pt}
\begin{lstlisting}[language=Rust]
fn main() {
	let value = Some(4);
	
	if calculate_square(value) != None {
		println!("\n\n The result of the square is {:?} \n\n", calculate_square(Some(4)).unwrap());
	} else {
		println!("\n\n No value present \n\n");
	}
}
\end{lstlisting}
\begin{itemize}
\item This part of the code sets up a value wrapped in `Some` and checks if the square of the value is calculated.
\item It demonstrates how to handle `Option` values and use the `unwrap` method safely.
\end{itemize}
\end{frame}

\begin{frame}[fragile]
\frametitle{Function Example: Using Option (Part 2)}
\vspace{20pt}
\begin{lstlisting}[language=Rust]
fn calculate_square(opt: Option<i32>) -> Option<i32> {
	match opt {
		Some(num) => Some(num * num),  // Wrap result in Some to match return type
		None => None,
	}
}
\end{lstlisting}
\begin{itemize}
\item This part defines the \texttt{calculate\_square} function that takes an `Option<i32>` and returns an `Option<i32>`.
\item It uses pattern matching to either calculate the square of the number if present or return `None`.
\item Highlights how to work with `Option` types and manage cases where values may or may not be present.
\end{itemize}
\end{frame}


\begin{frame}[fragile]
\frametitle{Result Enum: Syntax}
\begin{lstlisting}[language=Rust]
enum Result<T, E> {
	Ok(T),
	Err(E),
}
\end{lstlisting}
\begin{itemize}
\item `Result` is an enum used for error handling.
\item Can be either `Ok(value)` for success or `Err(error)` for failure.
\end{itemize}
\end{frame}

\begin{frame}[fragile]
\frametitle{Example 1: Division Function}
\vspace{40pt}
\begin{columns}[T]
\begin{column}{0.2\textwidth}
	\begin{itemize}
		\item Demonstrates handling division by zero with `Result`.
		\item Uses `match` to differentiate between valid and error cases.
	\end{itemize}
\end{column}
\begin{column}{0.78\textwidth}
\begin{lstlisting}[language=Rust]
fn safe_divide(numerator: f64, denominator: f64) -> Result<f64, String> {
	match denominator {
		0. => Err(String::from("Error: Cannot divide by zero")),
		_ => Ok(numerator / denominator),
	}
}

fn main() {
	println!("\n\nResult: {:?}", safe_divide(18.0, 6.0));  
	println!("Result: {:?}", safe_divide(5.0, 0.0));
	println!("Result: {:?} \n\n", safe_divide(0.0, 3.0));
}
\end{lstlisting}
\end{column}
\end{columns}
\end{frame}


\begin{frame}[fragile]
\frametitle{Example 2: Vector Lookup}
\begin{lstlisting}[language=Rust]
fn main() {
	let numbers: Vec<i32> = vec![10, 20, 30, 40, 50];
	
	let result = match numbers.get(7) {
		Some(value) => Ok(value),
		None => Err("Index out of bounds"),
	};
	
	println!("Result of the lookup: {:?}", result);
}
\end{lstlisting}
\begin{itemize}
\item Shows usage of `Result` with vector indexing.
\item Handles the case where the index is out of bounds.
\end{itemize}
\end{frame}

\begin{frame}[fragile]
\frametitle{Hash Maps: Basic Operations (1)}
\vspace{10pt}
\begin{columns}[T,onlytextwidth]
\begin{column}{0.7\textwidth}
\begin{lstlisting}[language=Rust]
use std::collections::HashMap;

fn main() {
	let mut people_ages: HashMap<&str, i32> = HashMap::new();
	people_ages.insert("Alice", 30);
	people_ages.insert("Bob", 25);
	people_ages.insert("Charlie", 35);
	
	println!("Alice's age is {:?}", people_ages.get("Alice").unwrap());
\end{lstlisting}
\end{column}
\begin{column}{0.3\textwidth}
\begin{itemize}
	\item Creating and using a `HashMap` for storing key-value pairs.
	\item Methods for adding, retrieving, and checking entries.
\end{itemize}
\end{column}
\end{columns}
\end{frame}


\begin{frame}[fragile]
\frametitle{Hash Maps: Basic Operations (2)}
\vspace{20pt}
\begin{columns}[T,onlytextwidth]
\begin{column}{0.7\textwidth}
\begin{lstlisting}[language=Rust,firstnumber=10]
	if people_ages.contains_key("Alice") {
		println!("The key 'Alice' is present.");
	} else {
		println!("The key 'Alice' is not found.");
	}
	match people_ages.get("Alice") {
		Some(age) => println!("Alice's age: {}", age),
		None => println!("No entry for Alice."),
	}
	
	for (name, age) in &people_ages {
		println!("{} is {} years old", name, age);
	}
}
\end{lstlisting}
\end{column}
\begin{column}{0.3\textwidth}
	\begin{itemize}
		\item Creating and using a `HashMap` for storing key-value pairs.
		\item Methods for adding, retrieving, and checking entries.
	\end{itemize}
\end{column}
\end{columns}
\end{frame}



\begin{frame}[fragile]
\frametitle{Hash Maps: Using `entry` API}
\begin{lstlisting}[language=Rust]
fn main() {
	let mut favorite_fruits: HashMap<&str, &str> = HashMap::new();
	
	favorite_fruits.entry("Alice").or_insert("banana");
	favorite_fruits.entry("Alice").or_insert("cherry");
	
	println!("Alice's favorite fruits: {:?}", favorite_fruits);
}
\end{lstlisting}
\begin{itemize}
\item Demonstrates the `entry` API for updating or inserting values.
\item Useful for setting default values or updating existing entries.
\end{itemize}
\end{frame}

\begin{frame}[fragile]
\frametitle{Hash Maps: Frequency Counting}
\begin{lstlisting}[language=Rust]
fn main() {
	let numbers: Vec<i32> = vec![1, 2, 2, 3, 3, 3, 4, 4, 4, 4];
	let mut counts: HashMap<i32, u32> = HashMap::new();
	
	for number in &numbers {
		let counter: &mut u32 = counts.entry(*number).or_insert(0);
		*counter += 1;
	}
	
	println!("Number frequencies: {:?}", counts);
}
\end{lstlisting}
\begin{itemize}
\item Counts occurrences of elements in a vector.
\item Utilizes \texttt{entry} and \texttt{or\_insert} to manage counts.
\end{itemize}
\end{frame}

\end{document}
