\documentclass[aspectratio=169, table]{beamer}

\usepackage[utf8]{inputenc}
\usepackage{listings} 

\usetheme{Pradita}

\subtitle{IF120203-Programming Fundamentals}

\title{Session-06:\\\LARGE{Lifetimes in Rust}\\ \vspace{10pt}}
\date[Serial]{\scriptsize {PRU/SPMI/FR-BM-18/0222}}
\author[Pradita]{\small{\textbf{Alfa Yohannis}}}

\lstdefinelanguage{Rust} {
keywords={MAX, std, print, fn, let, mut, println, true, false, u8, u16, u32, u64, u128, i8, i16, i32, i64, i128, f32, f64, char, bool, if, else, for, while, loop, match, return},
basicstyle=\ttfamily\small,
keywordstyle=\color{blue}\bfseries,
ndkeywords={self, String, Option, Some, None, Result, Ok, Err},
ndkeywordstyle=\color{purple}\bfseries,
sensitive=true,
commentstyle=\color{gray},
stringstyle=\color{red},
numbers=left,
numberstyle=\tiny\color{gray},
breaklines=true,
frame=lines,
backgroundcolor=\color{lightgray!10},
tabsize=2,
comment=[l]{//},
morecomment=[s]{/*}{*/},
commentstyle=\color{gray}\ttfamily,
stringstyle=\color{purple}\ttfamily,
showstringspaces=false
}

\begin{document}

\frame{\titlepage}

\begin{frame}[fragile]
\frametitle{Invalid References Example}
\begin{lstlisting}[language=Rust]
fn main() {
	let ref_to_int: &i32;
	{
		let local_int = 42;
		ref_to_int = &local_int;
	}
	println!("The referenced value is {}", ref_to_int);
}
\end{lstlisting}
\begin{itemize}
\item Demonstrates \texttt{invalid reference} due to \texttt{scope}.
\item \texttt{local\_int} goes out of \texttt{scope}, causing \texttt{ref\_to\_int} to be \texttt{invalid}.
\end{itemize}
\end{frame}

\begin{frame}[fragile]
\frametitle{Function Returning Reference}
\begin{lstlisting}[language=Rust]
fn main() {
	let number = 20;
	let result = my_function(number);
	println!("{}", result);
}

fn my_function(num: i32) -> &i32 {
	&num
}
\end{lstlisting}
\begin{itemize}
\item Demonstrates returning a \texttt{reference} to a \texttt{local variable}.
\item Causes a \texttt{compile-time error} due to \texttt{invalid reference}.
\end{itemize}
\end{frame}

\begin{frame}[fragile]
\frametitle{Comparing References}
\begin{lstlisting}[language=Rust]
fn main() {
	let num1 = 15;
	let num2 = 30;
	let max_num = max_value(&num1, &num2);
}

fn max_value(x: &i32, y: &i32) -> &i32 {
	if x > y {
		x
	} else {
		y
	}
}
\end{lstlisting}
\begin{itemize}
\item Compares two \texttt{integer references} and returns the \texttt{greater one}.
\item Demonstrates basic \texttt{lifetime} usage in \texttt{function return}.
\end{itemize}
\end{frame}

\begin{frame}[fragile]
\frametitle{String References}
\begin{lstlisting}[language=Rust]
fn main() {
	let greeting = "Hi";
	
	let message;
	{
		let word = String::from("Rust");
		message = combine_strings(greeting, word.as_str());
	}
	println!("\n\n{} \n\n", message);
}

fn combine_strings(str1: &str, str2: &str) -> &str {
	str1
}
\end{lstlisting}
\begin{itemize}
\item Demonstrates returning one of the \texttt{string references}.
\item \texttt{combine\_strings} returns the \texttt{first string reference}.
\end{itemize}
\end{frame}

\begin{frame}[fragile]
\frametitle{Dangling Reference Example}
\begin{lstlisting}[language=Rust]
fn main() {
	let greeting = "Hello";
	let message;
	{
		let world = String::from("World");
		message = concatenate(greeting, world.as_str());
	}
	println!("{}", message);
}

fn concatenate<'a, 'b>(str1: &'a str, str2: &'b str) -> &'a str {
	str1
}
\end{lstlisting}
\begin{itemize}
\item Demonstrates \texttt{dangling\ reference} due to \texttt{scope}.
\item \texttt{world} goes out of \texttt{scope}, causing \texttt{message} to be \texttt{invalid}.
\end{itemize}
\end{frame}

\begin{frame}[fragile]
\frametitle{Function with Lifetimes}
\begin{lstlisting}[language=Rust]
fn main() {
	let num1 = 5;
	let num2 = 10;
	let result = max_value(&num1, &num2);
}

fn max_value(x: &i32, y: &i32) -> i32 {
	if x > y {
		*x
	} else {
		*y
	}
}
\end{lstlisting}
\begin{itemize}
\item Compares two \texttt{integer\ references} and returns the \texttt{larger\ value}.
\item \texttt{max\_value} function dereferences and returns the larger value.
\end{itemize}
\end{frame}

\begin{frame}[fragile]
\frametitle{Lifetime Parameters Example}
\begin{lstlisting}[language=Rust]
fn main() {
	let num1 = 5;
	let num2 = 10;
	let result = max_value(&num1, num2);
}

fn max_value<'a>(x: &'a i32, y: i32) -> &'a i32 {
	x
}
\end{lstlisting}
\begin{itemize}
\item Returns a \texttt{reference} to the \texttt{first\ parameter}.
\item Demonstrates basic usage of \texttt{lifetime\ parameters}.
\end{itemize}
\end{frame}

\begin{frame}[fragile]
\frametitle{Multiple Lifetimes Example}
\begin{lstlisting}[language=Rust]
fn main() {
	let num1 = 5;
	let num2 = 10;
	let result = max_value(&num1, &num2);
}

fn max_value<'a, 'b>(x: &'a i32, y: &'b i32) -> &'a i32 {
	if x > y {
		x
	} else {
		y
	}
}
\end{lstlisting}
\begin{itemize}
\item Compares two \texttt{integer\ references} with different \texttt{lifetimes}.
\item Returns the \texttt{larger\ reference} based on \texttt{comparison}.
\end{itemize}
\end{frame}

\begin{frame}[fragile]
\frametitle{Scoped References Example}
\begin{lstlisting}[language=Rust]
fn main() {
	let num1 = 5;
	{
		let num2 = 10;
		let result = max_value(&num1, &num2);
		println!("The larger value is {}", result);
	}
}

fn max_value<'a, 'b>(x: &'a i32, y: &'a i32) -> &'a i32 {
	if x > y {
		x
	} else {
		y
	}
}
\end{lstlisting}
\begin{itemize}
\item Compares two \texttt{integer\ references} within a \texttt{scoped\ block}.
\item Prints the \texttt{larger\ value} within the \texttt{scope}.
\end{itemize}
\end{frame}

\begin{frame}[fragile]
\frametitle{Struct with Lifetime Example}
\begin{lstlisting}[language=Rust]

struct Individual<'a> {
	name: &'a str,
	age: i32,
}

fn main() {
	let first_name = "John";
	let mut person = Individual {
		name: first_name,
		age: 40,
	};
	
	{
		let last_name = String::from("Doe");
		person.name = &last_name;
	}
	
	println!("\n\nThe name of the person is {} and their age is {}\n\n", person.name, person.age);
}
\end{lstlisting}
\begin{itemize}
\item Demonstrates struct with a \texttt{lifetime\ parameter}.
\item Shows updating the struct's field within a \texttt{scoped\ block}.
\end{itemize}
\end{frame}

\begin{frame}[fragile]
\frametitle{Processing Vector References}
\begin{lstlisting}[language=Rust]
fn main() {
	let numbers: Vec<i32> = vec![5, 8, 9, 8, 7, 5, 2];
	let returned_vec = process_vec(&numbers, &numbers);
}

fn process_vec<'a>(vec1: &'a [i32], vec2: &'a [i32]) -> &'a [i32] {
	if 3 > 5 {
		vec1
	} else {
		vec2
	}
}
\end{lstlisting}
\begin{itemize}
\item Function takes two \texttt{vector\ references} and returns one of them.
\item Demonstrates usage of \texttt{lifetime\ parameters} with vectors.
\end{itemize}
\end{frame}


\begin{frame}[fragile]
\frametitle{Basic Closure Syntax}
\begin{lstlisting}[language=Rust]
fn main() {
	let a = 5;
	let calculate_square = || println!("\n\n Square of a is {} \n\n", a * a);
	calculate_square();
}
\end{lstlisting}
\begin{itemize}
\item Simple \texttt{closure} without parameters.
\item Captures the variable \texttt{a} from the environment.
\end{itemize}
\end{frame}

\begin{frame}[fragile]
\frametitle{Closure with Parameters}
\begin{lstlisting}[language=Rust]
fn main() {
	let a = 5;
	let compute_square = |value: i32| println!("\n\n Square of {} is {} \n\n", value, value * value);
	compute_square(a);
	
	let b = 15;
	compute_square(b);
}
\end{lstlisting}
\begin{itemize}
\item Closure that takes an \texttt{input\ parameter}.
\item Reuse the same closure with different \texttt{values}.
\end{itemize}
\end{frame}

\begin{frame}[fragile]
\frametitle{Multiple Closures with Same Variable Name}
\begin{lstlisting}[language=Rust]
fn main() {
	let a = 5;
	let compute_square = |value: i32| println!("\n\n Square is {}", value * value);
	let compute_cube = |value: i32| println!("\n\n Cube is {} \n\n", value * value * value);
	compute_cube(a);
	
	let b = 15;
	compute_cube(b);
}
\end{lstlisting}
\begin{itemize}
\item Redefining \texttt{closure} with the same name.
\item Demonstrates shadowing with different \texttt{closure\ logic}.
\end{itemize}
\end{frame}

\begin{frame}[fragile]
\frametitle{Ownership and Closures}
\begin{lstlisting}[language=Rust]
fn main() {
	let display_user_info = |info: String, name: &str, age| println!("{}\n\t{}: {}", info, name, age);
	let info = String::from("User Information:");
	let (name, age) = (String::from("Alex"), 30);
	
	display_user_info(info, &name, age);
	println!("Variable after move: {}", name);
}
\end{lstlisting}
\begin{itemize}
\item Passing ownership and references to \texttt{closure}.
\item Demonstrates how variables are moved or borrowed.
\end{itemize}
\end{frame}

\begin{frame}[fragile]
\frametitle{Inferring Inputs and Outputs}
\begin{lstlisting}[language=Rust]
fn main() {
	let compute_square = |value| value * value;
	
	let a = 5;
	compute_square(a);
	
	let b = 105.5;
	compute_square(b);
}
\end{lstlisting}
\begin{itemize}
\item Inference of input and output types in \texttt{closure}.
\item Demonstrates usage with \texttt{integer\ and\ floating-point\ values}.
\end{itemize}
\end{frame}

\begin{frame}[fragile]
\frametitle{Passing Closure as Function Argument}
\begin{lstlisting}[language=Rust]
fn divide<F: Fn(f32) -> bool>(numerator: f32, denominator: f32, is_valid: F) {
	if is_valid(denominator) {
		println!("\n\n Result is {} \n\n", numerator / denominator);
	} else {
		println!("\n\n Division by zero is not allowed \n\n");
	}
}

fn main() {
	let check_denominator = |denom: f32| denom != 0.0;
	divide(5.0, 10.0, check_denominator);
	divide(54.0, 0.0, check_denominator);
}
\end{lstlisting}
\begin{itemize}
\item Function that takes a \texttt{closure\ as\ an\ argument}.
\item Demonstrates checking validity of input before performing division.
\end{itemize}
\end{frame}


\begin{frame}[fragile]
\frametitle{Basic Closure Syntax}
\begin{lstlisting}[language=Rust]
fn main() {
	let increment_1 = |x: u32| -> u32 { x + 1 };
	let increment_2 = |x| { x + 1 };
	let increment_3 = |x| x + 1;
}
\end{lstlisting}
\begin{itemize}
\item Defines three closures to increment a value.
\item Demonstrates various ways of specifying closure syntax.
\end{itemize}
\end{frame}

\begin{frame}[fragile]
\frametitle{Borrowing by Immutable Reference}
\begin{lstlisting}[language=Rust]
fn main() {
	let mut numbers = vec![1, 2, 3];
	let display_numbers = || {
		// Accessing numbers by reference.
		println!("Numbers: {:?}", numbers);
	};
	
	println!("Numbers: {:?}", numbers);
	display_numbers();
	
	numbers[1] = 15;
}
\end{lstlisting}
\begin{itemize}
\item Closure accessing \texttt{numbers} immutably.
\item Changes to \texttt{numbers} after closure invocation.
\end{itemize}
\end{frame}

\begin{frame}[fragile]
\frametitle{Borrowing by Mutable Reference}
\begin{lstlisting}[language=Rust]
fn main() {
	let mut numbers = vec![4, 5, 6];
	let mut add_to_numbers = || {
		numbers.push(35);
	};
	
	// Displaying numbers is commented out.
	// numbers[1] = 15;
	add_to_numbers();
	
	// Modifying numbers is commented out.
	// numbers[2] = 15;
}
\end{lstlisting}
\begin{itemize}
\item Mutable closure modifying \texttt{numbers}.
\item Demonstrates the effect of mutable borrowing.
\end{itemize}
\end{frame}

\begin{frame}[fragile]
\frametitle{Moving a Value into a Closure}
\begin{lstlisting}[language=Rust]
fn main() {
	let mut numbers_1 = vec![1, 2, 3];
	let handle_numbers = || {
		let numbers_2 = numbers_1;
	};
	
	handle_numbers();
	// Accessing numbers_1 after moving is not possible.
	// println!("Numbers 1 = {:?}", numbers_1);
	// println!("Numbers 2 = {:?}", numbers_2);
}
\end{lstlisting}
\begin{itemize}
\item Moving \texttt{numbers\_1} into the closure.
\item Shows that \texttt{numbers\_1} is no longer accessible.
\end{itemize}
\end{frame}


\begin{frame}[fragile]
\frametitle{Function Types: Basic Syntax}
\begin{lstlisting}[language=Rust]
fn maximum(x: i32, y: i32) -> i32 {
	if x > y {
		x
	} else {
		y
	}
}

fn minimum(x: i32, y: i32) -> i32 {
	if x < y {
		x
	} else {
		y
	}
}

fn main() {
	let mut chosen_function = maximum;
	println!("The minimum of the two values is {}", chosen_function(2, 3));
}
\end{lstlisting}
\begin{itemize}
\item Shows how to define and use functions as values.
\item Demonstrates storing and invoking a function.
\end{itemize}
\end{frame}

\begin{frame}[fragile]
\frametitle{Function Types as Parameters}
\begin{lstlisting}[language=Rust]
fn display_name(name: &str) {
	print!("The name is {}", name); 
}

fn show_details(func: fn(&str), person: &str, age: i32) {
	func(person); 
	println!(" and my age is {}", age);
}

fn main() {
	let (name, age) = (String::from("Nouman"), 40); 
	show_details(display_name, &name, age);
}
\end{lstlisting}
\begin{itemize}
\item Passing functions as arguments to other functions.
\item Function \texttt{show\_details} uses a function pointer to print details.
\end{itemize}
\end{frame}

\begin{frame}[fragile]
\frametitle{Function Types: Applying Twice}
\begin{lstlisting}[language=Rust]
fn increment(x: i32) -> i32 {
	x + 1
}

fn apply_twice(func: fn(i32) -> i32, value: i32) -> i32 {
	func(value) + func(value)
}

fn main() {
	let result = apply_twice(increment, 5);
	println!("The result is: {}", result);
}
\end{lstlisting}
\begin{itemize}
\item Demonstrates passing a function as a parameter and applying it multiple times.
\item Shows how functions can be used as first-class citizens in Rust.
\end{itemize}
\end{frame}


\begin{frame}[fragile]
\frametitle{Iterators: Basics}
\begin{lstlisting}[language=Rust]
fn main() {
	let numbers = vec![1, 2, 3, 4, 5, 6, 7];
	let mut iterator = numbers.iter();
	
	println!("The iterator: {:?}", iterator); 
	println!("{:?}", iterator.next());
	println!("{:?}", iterator.next());
	println!("{:?}", iterator.next());
	println!("{:?}", iterator.next());
	println!("{:?}", iterator.next());
	println!("{:?}", iterator.next());
	println!("{:?}", iterator.next());
	println!("{:?}", iterator.next());
}
\end{lstlisting}
\begin{itemize}
\item Introduction to iterators and their basic usage.
\item Demonstrates creating and using an iterator with `vec`.
\end{itemize}
\end{frame}

\begin{frame}[fragile]
\frametitle{Useful Functions for Iterators}
\begin{lstlisting}[language=Rust]
fn main() {
	let values: Vec<u32> = vec![0, 1, 2, 4, 5, 6, 9, 8, 7];
	
	let any_check = values.iter().any(|&x| x > 0);
	println!("The result of the any function is {}", any_check);
	
	let all_check = values.iter().all(|&x| x > 0);
	println!("The result of the all function is {}", all_check);
	
	let find_check = values.iter().find(|&&x| x > 0);
	println!("The result of the find function is {}", find_check.unwrap());
	
	let position_check = values.iter().position(|&x| x > 4);
	println!("The result of the position function is {}", position_check.unwrap());
	
	let rposition_check = values.iter().rposition(|&x| x > 4);
	println!("The result of the rposition function is {}", rposition_check.unwrap());
	
	let max_check = values.iter().max();
	println!("The result of the max function is {}", max_check.unwrap());
	
	let min_check = values.iter().min();
	println!("The result of the min function is {}", min_check.unwrap());
	
	let sum_check: u32 = values.iter().sum();
	let product_check: u32 = values.iter().product(); 
	println!("Sum and product: {} {}", sum_check, product_check);
	
	let mut reversed_iter = values.iter().rev();
	println!("The result of applying the rev function {:?}", reversed_iter.collect::<Vec<_>>());
	println!("Original vector: {:?}", values);
}
\end{lstlisting}
\begin{itemize}
\item Utilizes common iterator functions such as `any`, `all`, `find`, `position`, and `rposition`.
\item Shows how to compute maximum, minimum, sum, and product using iterators.
\item Demonstrates reversing an iterator with `rev`.
\end{itemize}
\end{frame}


\begin{frame}[fragile]
\frametitle{Iterators: Basics}
\begin{lstlisting}[language=Rust]
fn main() {
	let numbers = vec![0, 1, 2, 3, 4, 5, 6, 7];
	
	let filtered_refs = numbers.iter().filter(|&x| *x >= 5).collect::<Vec<&u32>>();
	println!("Filtered references: {:?}", filtered_refs);
	
	let cloned_numbers = numbers.clone();
	let filtered_values = cloned_numbers.into_iter().filter(|x| *x >= 5).collect::<Vec<u32>>();
	println!("Filtered values: {:?}", filtered_values);
	
	let mapped_values = cloned_numbers.iter().map(|x| 2 * *x).collect::<Vec<u32>>();
	println!("Mapped values: {:?}", mapped_values);
	
	let filtered_mapped_values = cloned_numbers.iter().map(|x| 2 * x).filter(|x| *x > 10).collect::<Vec<u32>>();
	println!("Filtered and mapped values: {:?}", filtered_mapped_values);
}
\end{lstlisting}
\begin{itemize}
\item Demonstrates basic usage of iterators with `filter` and `map` functions.
\item Shows how to collect filtered and mapped values into vectors.
\end{itemize}
\end{frame}

\begin{frame}[fragile]
\frametitle{Iterators: Example of Summing Multiples}
\begin{lstlisting}[language=Rust]
fn main() {
	let mut input = String::new();
	std::io::stdin()
	.read_line(&mut input)
	.expect("Failed to read input.");
	let limit: u32 = input.trim().parse().expect("Invalid input");
	
	let multiples = (1..limit)
	.filter(|&x| x % 3 == 0 || x % 5 == 0)
	.collect::<Vec<u32>>();
	
	println!("Multiples of 3 or 5: {:?}", multiples);
	println!("Sum of multiples: {:?}", multiples.iter().sum::<u32>());
}
\end{lstlisting}
\begin{itemize}
\item Reads an integer from user input.
\item Filters numbers less than the input that are multiples of 3 or 5.
\item Collects these numbers into a vector.
\item Prints the filtered numbers and their sum.
\end{itemize}
\end{frame}

\begin{frame}[fragile]
\frametitle{Set Operations: Union and Intersection with Different Values}
\begin{lstlisting}[language=Rust]
fn main() {
	let mut set1: Vec<u32> = vec![2, 7, 8, 11, 14];
	let mut set2: Vec<u32> = vec![2, 6, 8, 10, 13, 18, 22];
	
	let common_elements = find_intersection(&set1, &set2);
	println!("\n\n The intersection of the two sets is {:?}", common_elements);
	
	let union_set = compute_union(&mut set1, &mut set2, &common_elements);
	println!("\n\n The union of the sets is {:?}", union_set);
	
	// Alternative method for intersection
	let set1_copy = set1.clone();
	
	// Alternative way to find common elements
	let common_elements: Vec<u32> = set1.into_iter()
	.filter(|&x| set2.iter().any(|&y| y == x))
	.collect();
	println!("The common values are {:?}", common_elements);
	// This approach will consume set1
	
	println!("The uncommon values are {:?}", uncommon_elements);
}

// Function to find the intersection of two sets
fn find_intersection(set1: &Vec<u32>, set2: &Vec<u32>) -> Vec<u32> {
	let mut common_elements: Vec<u32> = Vec::new();
	
	for item in set1 {
		if set2.iter().any(|&x| x == *item) {
			common_elements.push(*item);
		}
	}
	common_elements
}

// Function to compute the union of two sets, excluding common elements
fn compute_union<'a>(set1: &'a mut Vec<u32>, set2: &'a mut Vec<u32>, common_elements: &'a Vec<u32>) -> Vec<&'a u32> {
	for item in common_elements {
		if let Some(pos1) = set1.iter().position(|&x| x == *item) {
			set1.remove(pos1);
		}
		if let Some(pos2) = set2.iter().position(|&x| x == *item) {
			set2.remove(pos2);
		}
	}
	let union_set = set1.iter()
	.chain(set2.iter())
	.chain(common_elements.iter())
	.collect::<Vec<_>>();
	union_set
}
\end{lstlisting}
\begin{itemize}
\item Initializes two vectors with different values.
\item Computes the intersection of the two vectors.
\item Calculates the union of the vectors while excluding the intersection.
\item Includes an alternative method for finding common elements.
\end{itemize}
\end{frame}

\end{document}
