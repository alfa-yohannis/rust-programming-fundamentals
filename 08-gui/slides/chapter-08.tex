\documentclass[aspectratio=169, table]{beamer}

\usepackage[utf8]{inputenc}
\usepackage{listings} 
\usepackage{color}

\usetheme{Pradita}

\subtitle{IF120203-Programming Fundamentals}
\title{Session-08:\\\LARGE{Simple GUI with FLTK and Rust\\\vspace{10pt}}}
\date[Serial]{\scriptsize {PRU/SPMI/FR-BM-18/0222}}
\author[Pradita]{\small{\textbf{Alfa Yohannis}}}

\lstdefinelanguage{Rust} {
keywords={let, mut, fn, pub, struct, impl, move, unwrap, true, false, u8, u16, u32, u64, u128, i8, i16, i32, i64, i128, f32, f64, char, bool, if, else, for, while, loop, match, return},
basicstyle=\ttfamily\small,
keywordstyle=\color{blue}\bfseries,
ndkeywords={self, String, Option, Some, None, Result, Ok, Err},
ndkeywordstyle=\color{purple}\bfseries,
sensitive=true,
commentstyle=\color{gray},
stringstyle=\color{red},
numbers=left,
numberstyle=\tiny\color{gray},
breaklines=true,
frame=lines,
backgroundcolor=\color{lightgray!10},
tabsize=2,
comment=[l]{//},
morecomment=[s]{/*}{*/},
commentstyle=\color{gray}\ttfamily,
stringstyle=\color{purple}\ttfamily,
showstringspaces=false
}

\begin{document}

\frame{\titlepage}

% Table of contents slide
\begin{frame}{Contents}
\vspace{15pt}
\begin{columns}[t]
\begin{column}{.5\textwidth}
	\tableofcontents[sections={1-12}]
\end{column}
\begin{column}{.5\textwidth}
	\tableofcontents[sections={13-24}]
\end{column}
\end{columns}
\end{frame}

\section{Simple App - Setting Up the FLTK App}
\begin{frame}[fragile]
\frametitle{Simple App - Setting Up the FLTK App}
\begin{lstlisting}[language=Rust]
use fltk::{app, button::Button, frame::Frame, input::Input, prelude::*, window::Window};

fn main() {
	let app = app::App::default();
	let (screen_width, screen_height) = app::screen_size();
	let win_width = 400;
	let win_height = 300;
\end{lstlisting}
\begin{itemize}
	\item Imports necessary modules from the \texttt{fltk} crate.
	\item Creates an instance of the FLTK application and retrieves screen dimensions.
	\item Defines the dimensions for the application window.
\end{itemize}
\end{frame}

\section{Creating the Main Window}
\begin{frame}[fragile]
\frametitle{Creating the Main Window}
\begin{lstlisting}[language=Rust]
	let mut win = Window::new(
	(screen_width as i32 - win_width) / 2,
	(screen_height as i32 - win_height) / 2,
	win_width,
	win_height,
	"Greetings App",
	);
	win.make_resizable(true);
\end{lstlisting}
\begin{itemize}
	\item Initializes the main application window centered on the screen.
	\item The window is made resizable by the user.
	\item The window title is set to "Greetings App".
\end{itemize}
\end{frame}

\section{Adding Input Fields}
\begin{frame}[fragile]
\frametitle{Adding Input Fields}
\begin{lstlisting}[language=Rust]
	let name_input = Input::new(160, 50, 200, 30, "Name:");
	let mut age_input = Input::new(160, 90, 200, 30, "Age:");
\end{lstlisting}
\begin{itemize}
	\item Creates an input field for the user's name with a label.
	\item Adds an input field for the user's age, also labeled.
\end{itemize}
\end{frame}

\section{Filtering Numeric Input for Age}
\begin{frame}[fragile]
\frametitle{Filtering Numeric Input for Age}
\vspace{15pt}
\begin{lstlisting}[language=Rust]
	// Set input to allow only numeric values for the age field
	age_input.set_callback({
		let mut age_input = age_input.clone();
		move |input| {
			let text = input.value();
			let filtered: String = text.chars().filter(|c| c.is_numeric()).collect();
			if text != filtered {
				age_input.set_value(&filtered);
			}
		}
	});
\end{lstlisting}
\begin{itemize}
	\item Restricts the age input field to accept only numeric values.
	\item The callback filters out non-numeric characters in real-time.
\end{itemize}
\end{frame}

\section{Adding a Button and Display Frame}
\begin{frame}[fragile]
\frametitle{Adding a Button and Display Frame}
\begin{lstlisting}[language=Rust]
	let mut btn = Button::new(160, 130, 80, 40, "Say Hi!");
	let mut frame = Frame::new(60, 170, 280, 40, "");
\end{lstlisting}
\begin{itemize}
	\item Creates a button labeled "Say Hi!" to trigger the greeting.
	\item Adds a frame to display the greeting message.
\end{itemize}
\end{frame}

\section{Button Callback}
\begin{frame}[fragile]
\frametitle{Button Callback}
\begin{lstlisting}[language=Rust]
	btn.set_callback(move |_| {
		let name = name_input.value();
		let age = age_input.value();
		frame.set_label(&format!("Hello, {}! I'm {} years old.", name, age));
	});
\end{lstlisting}
\begin{itemize}
	\item Defines the button's callback function to display the greeting message.
	\item Retrieves values from the name and age input fields.
	\item Formats and sets the greeting text in the frame.
\end{itemize}
\end{frame}

\section{Running the Application}
\begin{frame}[fragile]
\frametitle{Running the Application}
\begin{lstlisting}[language=Rust]
	win.end();
	win.show();
	
	app.run().unwrap();
}
\end{lstlisting}
\begin{itemize}
\item Finalizes the window and displays it to the user.
\item Runs the FLTK application, ensuring the event loop executes.
\end{itemize}
\end{frame}

\section{Calculator - Setting Up the FLTK App}
\begin{frame}[fragile]
\frametitle{Calculator - Setting Up the FLTK App}
\begin{lstlisting}[language=Rust]
use fltk::{app, button::Button, frame::Frame, prelude::*, window::Window};

fn main() {
	let app = app::App::default();
	let (screen_width, screen_height) = app::screen_size();
	let win_width = 220;
	let win_height = 340;
\end{lstlisting}
\begin{itemize}
	\item Imports necessary modules from the \texttt{fltk} crate.
	\item Creates an instance of the FLTK application and retrieves screen dimensions.
	\item Defines the dimensions for the application window.
\end{itemize}
\end{frame}

\section{Creating the Main Window}
\begin{frame}[fragile]
\frametitle{Creating the Main Window}
\begin{lstlisting}[language=Rust]
	let mut win = Window::new(
	(screen_width as i32 - win_width) / 2,
	(screen_height as i32 - win_height) / 2,
	win_width,
	win_height,
	"Simple Calculator",
	);
	win.make_resizable(true);
\end{lstlisting}
\begin{itemize}
	\item Initializes the main application window centered on the screen.
	\item The window is made resizable by the user.
	\item The window title is set to "Simple Calculator".
\end{itemize}
\end{frame}

\section{Adding the Display Frame}
\begin{frame}[fragile]
\frametitle{Adding the Display Frame}
\begin{lstlisting}[language=Rust]
	// Display frame
	let mut display = Frame::new(20, 20, 200, 40, "");
	display.set_label_size(24);
\end{lstlisting}
\begin{itemize}
	\item Adds a display frame to show the calculator's input and output.
	\item The label size of the display is set to 24 for better visibility.
\end{itemize}
\end{frame}

\section{Creating Digit Buttons}
\begin{frame}[fragile]
\frametitle{Creating Digit Buttons}
\begin{lstlisting}[language=Rust]
	// Button creation
	let mut buttons = vec![];
	
	// Digits 1-9
	for i in 1..=9 {
		let x = 20 + ((i - 1) % 3) * 60;
		let y = 80 + ((i - 1) / 3) * 60;
		let mut btn = Button::new(x, y, 50, 50, i.to_string().as_str());
	\end{lstlisting}
	\begin{itemize}
		\item Creates buttons for digits 1 through 9.
		\item The buttons are arranged in a 3x3 grid pattern.
	\end{itemize}
\end{frame}

\section{Handling Button Clicks}
\begin{frame}[fragile]
	\frametitle{Handling Button Clicks}
	\begin{lstlisting}[language=Rust]
		btn.set_callback({
			let mut display = display.clone();
			move |_| {
				let current_text = display.label();
				display.set_label(&(current_text + &i.to_string()));
			}
		});
		buttons.push(btn);
	}
\end{lstlisting}
\begin{itemize}
	\item Defines callbacks for the digit buttons.
	\item Appends the digit to the display when the corresponding button is clicked.
\end{itemize}
\end{frame}

\section{Adding Function Buttons}

\begin{frame}[fragile]
\frametitle{Adding Plus Button}
\vspace{15pt}
\begin{lstlisting}[language=Rust]
// Plus button
let mut plus_btn = Button::new(20, 260, 50, 50, "+");
plus_btn.set_callback({
	let mut display = display.clone();
	move |_| {
		let current_text = display.label();
		if !current_text.is_empty() && !current_text.ends_with('+') {
			display.set_label(&(current_text + "+"));
		}
	}
});
buttons.push(plus_btn);
\end{lstlisting}
\begin{itemize}
\item Adds the plus button to the calculator.
\item Prevents consecutive plus signs from being added.
\end{itemize}
\end{frame}

\begin{frame}[fragile]
\frametitle{Adding Zero Button}
\begin{lstlisting}[language=Rust]
// Zero button
let mut zero_btn = Button::new(80, 260, 50, 50, "0");
zero_btn.set_callback({
	let mut display = display.clone();
	move |_| {
		let current_text = display.label();
		display.set_label(&(current_text + "0"));
	}
});
buttons.push(zero_btn);
\end{lstlisting}
\begin{itemize}
\item Adds the zero button to the calculator.
\item Appends a zero to the display when clicked.
\end{itemize}
\end{frame}

\section{Adding the Equals Button}
\begin{frame}[fragile]
\frametitle{Adding the Equals Button}
\vspace{15pt}
\begin{lstlisting}[language=Rust]
// Equals button
let mut equals_btn = Button::new(140, 260, 50, 50, "=");
equals_btn.set_callback({
	let mut display = display.clone();
	move |_| {
		let current_text = display.label();
		if let Some(result) = calculate_expression(&current_text) {
			display.set_label(&result.to_string());
		}
	}
});
buttons.push(equals_btn);
\end{lstlisting}
\begin{itemize}
\item Adds the equals button to the calculator.
\item Evaluates the expression and updates the display with the result.
\end{itemize}
\end{frame}

\section{Running the Application}

\begin{frame}[fragile]
	\frametitle{Running the FLTK Application}
	\begin{lstlisting}[language=Rust]
		win.end();
		win.show();
		
		app.run().unwrap();
	\end{lstlisting}
	\begin{itemize}
		\item Finalizes and displays the window.
		\item Runs the FLTK application, entering the event loop.
	\end{itemize}
\end{frame}

\begin{frame}[fragile]
	\frametitle{Expression Evaluation Function}
	\vspace{15pt}
	\begin{lstlisting}[language=Rust]
		fn calculate_expression(expr: &str) -> Option<i32> {
			let parts: Vec<&str> = expr.split('+').collect();
			if parts.len() > 2 {
				let sum: i32 = parts
				.iter()
				.map(|s| s.parse::<i32>().unwrap())
				.sum();
				Some(sum)
			} else {
				None
			}
		}
	\end{lstlisting}
	\begin{itemize}
		\item Implements a basic expression evaluator for addition.
		\item Used to calculate the result when the equals button is clicked.
	\end{itemize}
\end{frame}


\end{document}
