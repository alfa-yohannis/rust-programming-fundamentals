\documentclass[aspectratio=169, table]{beamer}

\usepackage[utf8]{inputenc}
\usepackage{listings} 

\usetheme{Pradita}

\subtitle{IF120203-Programming Fundamentals}

\title{Session-07:\\\LARGE{Modules in Rust}\\ \vspace{10pt}}
\date[Serial]{\scriptsize {PRU/SPMI/FR-BM-18/0222}}
\author[Pradita]{\small{\textbf{Alfa Yohannis}}}

\lstdefinelanguage{Rust} {
keywords={MAX, std, print, fn, let, mut, println, true, false, u8, u16, u32, u64, u128, i8, i16, i32, i64, i128, f32, f64, char, bool, if, else, for, while, loop, match, return},
basicstyle=\ttfamily\small,
keywordstyle=\color{blue}\bfseries,
ndkeywords={self, String, Option, Some, None, Result, Ok, Err},
ndkeywordstyle=\color{purple}\bfseries,
sensitive=true,
commentstyle=\color{gray},
stringstyle=\color{red},
numbers=left,
numberstyle=\tiny\color{gray},
breaklines=true,
frame=lines,
backgroundcolor=\color{lightgray!10},
tabsize=2,
comment=[l]{//},
morecomment=[s]{/*}{*/},
commentstyle=\color{gray}\ttfamily,
stringstyle=\color{purple}\ttfamily,
showstringspaces=false
}

\begin{document}

\frame{\titlepage}

% Add table of contents slide
\begin{frame}{Contents}
\vspace{15pt}
\begin{columns}[t]
\begin{column}{.5\textwidth}
	\tableofcontents[sections={1-12}]
\end{column}
\begin{column}{.5\textwidth}
	\tableofcontents[sections={13-24}]
\end{column}
\end{columns}
\end{frame}

\section{Example 1}
\begin{frame}[fragile]
\frametitle{Module Definition: \texttt{module\_a.rs}}
\begin{lstlisting}[language=Rust]
pub fn hello() {
	println!("Hello from A"); 
}
\end{lstlisting}
\begin{itemize}
\item Defines a function \texttt{hello} in \texttt{module\_a}.
\item This function prints "Hello from A".
\end{itemize}
\end{frame}

\begin{frame}[fragile]
\frametitle{Module Definition: \texttt{module\_b.rs}}
\begin{lstlisting}[language=Rust]
pub fn hello() {
	println!("Hello from B");
}
\end{lstlisting}
\begin{itemize}
\item Defines a function \texttt{hello} in \texttt{module\_b}.
\item This function prints "Hello from B".
\end{itemize}
\end{frame}

\begin{frame}[fragile]
\frametitle{Library Module Declaration: \texttt{lib.rs}}
\begin{lstlisting}[language=Rust]
pub mod module_a; 
pub mod module_b; 
\end{lstlisting}
\begin{itemize}
\item Declares the modules \texttt{module\_a} and \texttt{module\_b}.
\item These modules are included in the library crate.
\end{itemize}
\end{frame}

\begin{frame}[fragile]
\frametitle{Main Function Usage: \texttt{main.rs}}
\begin{lstlisting}[language=Rust]
fn main() { 
	code_01::module_a::hello();
	code_01::module_b::hello(); 
}
\end{lstlisting}
\begin{itemize}
\item Calls the \texttt{hello} function from \texttt{module\_a} and \texttt{module\_b}.
\item Outputs:
\begin{itemize}
	\item "Hello from A"
	\item "Hello from B"
\end{itemize}
\end{itemize}
\end{frame}

\section{Module Definition: module\_a.rs}
\begin{frame}[fragile]
\frametitle{Module Definition: \texttt{module\_a.rs}}
\vspace{15pt}
\begin{lstlisting}[language=Rust]
pub mod arithmetic {
pub mod core_operations {
	pub fn multiply(x: &i32, y: &i32) -> i32 {
		let product = x * y;
		show_result(&product);
		product
	}
	fn show_result(product: &i32) {
		println!("The computed result is {}", product);
		crate::component_a::display_message();
	} } }
\end{lstlisting}
\begin{itemize}
\item Defines the \texttt{arithmetic} module with \texttt{core\_operations}.
\item Includes a function \texttt{multiply} and a helper function \texttt{show\_result}.
\item \texttt{show\_result} prints the result and calls \texttt{display\_message} (commented out).
\end{itemize}
\end{frame}

\section{Module Definition: module\_b.rs}

\begin{frame}[fragile]
\frametitle{Module Definition: Struct and Method}
\vspace{15pt}
\begin{lstlisting}[language=Rust]
mod animal {
	pub struct Characteristics {
		pub height: u8,
		pub species: String,
	}
	impl Characteristics {
		pub fn create(new_species: &str) -> Self {
			Self {
				species: String::from(new_species),
				height: 30,
			}
		} } }
\end{lstlisting}
\begin{itemize}
\item Defines a struct \texttt{Characteristics} within the \texttt{animal} module.
\item Includes a method \texttt{create} to instantiate \texttt{Characteristics}.
\end{itemize}
\end{frame}

\begin{frame}[fragile]
\frametitle{Module Definition: Function \texttt{create\_animals}}
\begin{lstlisting}[language=Rust]
pub fn create_animals() {
	let mut animal1 = animal::Characteristics::create("Feline");
	animal1.species = String::from("Canine");
\end{lstlisting}
\begin{itemize}
	\item Shows the usage of \texttt{Characteristics::create} in the \texttt{create\_animals} function.
	\item Demonstrates modification of an instance's fields.
\end{itemize}
\end{frame}

\begin{frame}[fragile]
\frametitle{Module Definition: Additional Instance Creation}
\begin{lstlisting}[language=Rust]
	let animal2 = animal::Characteristics {
		height: 50,
		species: String::from("Avian"),
	};
}
\end{lstlisting}
\begin{itemize}
\item Creates another instance of \texttt{Characteristics} with specified values.
\end{itemize}
\end{frame}


\section{Module Definition: module\_c.rs}
\begin{frame}[fragile]
\frametitle{Module Definition: \texttt{module\_c.rs}}
\begin{lstlisting}[language=Rust]
mod job_history {
	pub(crate) enum Employment {
		FullTime,
		PartTime,
		Freelance,
	}
}
pub fn calculate_salary() {
	let job1 = job_history::Employment::FullTime;
	let job2 = job_history::Employment::PartTime;
}
\end{lstlisting}
\begin{itemize}
\item Defines an enum \texttt{Employment} in \texttt{module\_c}.
\item Includes a function \texttt{calculate\_salary} that uses the enum.
\end{itemize}
\end{frame}

\section{Library Modules}
\begin{frame}[fragile]
\frametitle{Library Modules: \texttt{lib.rs}}
\begin{lstlisting}[language=Rust]
pub mod module_a;
pub mod module_b;
pub mod module_c;
\end{lstlisting}
\begin{itemize}
\item Defines the public modules \texttt{module\_a}, \texttt{module\_b}, and \texttt{module\_c}.
\item These modules contain functions and structs used in the main program.
\end{itemize}
\end{frame}

\section{Main Function}

\begin{frame}[fragile]
\frametitle{Main Function: Imports and Struct}
\begin{lstlisting}[language=Rust]
use code::module_a::arithmetic::core_operations::multiply;

fn main() {
	let rectangle = Rectangle {
		height: 5,
		breadth: 10,
	};
\end{lstlisting}
\begin{itemize}
	\item Imports the \texttt{multiply} function from \texttt{module\_a}.
	\item Defines a \texttt{Rectangle} struct and initializes it in \texttt{main}.
\end{itemize}
\end{frame}

\begin{frame}[fragile]
\frametitle{Main Function: Area Calculation}
\begin{lstlisting}[language=Rust]
	let rectangle_area = calculate_area(&rectangle.height, &rectangle.breadth);
}

struct Rectangle {
	height: i32,
	breadth: i32,
}
\end{lstlisting}
\begin{itemize}
\item Shows the calculation of the area using the \texttt{calculate\_area} function.
\item Defines the \texttt{Rectangle} struct.
\end{itemize}
\end{frame}

\begin{frame}[fragile]
\frametitle{Main Function: Calculate Area Function}
\begin{lstlisting}[language=Rust]
pub fn calculate_area(height: &i32, breadth: &i32) -> i32 {
	multiply(height, breadth)
}
\end{lstlisting}
\begin{itemize}
\item Includes the \texttt{calculate\_area} function which uses \texttt{multiply} to compute the area.
\end{itemize}
\end{frame}


\section{Commented Code}

\begin{frame}[fragile]
\frametitle{Commented Code: Previous Implementations}
\begin{lstlisting}[language=Rust]
// fn display_message() {
	//     println!("This is a message from the primary_module crate");
	// }

// primary_module
//         - arithmetic
//             - core_operations
//         area_of_rectangle()
//         display_message()
\end{lstlisting}
\begin{itemize}
\item Shows the previous function \texttt{display\_message} and module structure.
\end{itemize}
\end{frame}

\begin{frame}[fragile]
\frametitle{Commented Code: Alternative Implementations}
\begin{lstlisting}[language=Rust]
// pub mod arithmetic {
	//     pub mod core_operations {
		//         pub fn multiply(x: &i32, y: &i32) -> i32 {
			//             let product = x * y;
			//             show_result(&product);
			//             product
			//         }
		//         fn show_result(product: &i32) {
			//             println!("The computed result is {}", product);
			//             crate::component_a::display_message();
			//         }
		//     }
	// }
\end{lstlisting}
\begin{itemize}
\item Includes an alternative definition of \texttt{arithmetic} and \texttt{core\_operations}.
\end{itemize}
\end{frame}

\begin{frame}[fragile]
\frametitle{Commented Code: Function Definitions}
\begin{lstlisting}[language=Rust]
// pub fn area_of_rectangle(length: &i32, width: &i32) -> i32 {
	//     use arithmetic::core_operations::multiply;
	//     multiply(length, width)
	// }

// mod module_b;
// fn main() {
	//     module_b::create_animals();
	// }

// mod module_c;
// fn main() {
	//     module_c::calculate_salary();
	// }
\end{lstlisting}
\begin{itemize}
\item Defines the \texttt{area\_of\_rectangle} function and shows usage of \texttt{module\_b} and \texttt{module\_c}.
\end{itemize}
\end{frame}


\section{Cargo.toml}
\begin{frame}[fragile]
\frametitle{Adding Dependencies: \texttt{Cargo.toml}}
\begin{lstlisting}
[dependencies]
array_tool = "1.0.3"
\end{lstlisting}
\begin{itemize}
\item Add the \texttt{array\_tool} library to your dependencies in \texttt{Cargo.toml}.
\item This allows the use of additional vector manipulation functions.
\end{itemize}
\end{frame}

\section{Third-Party Library Usage}

\begin{frame}[fragile]
\frametitle{Third-Party Library: \texttt{array\_tool}}
\begin{lstlisting}[language=Rust]
// -------------------------------------------
//          Third-Party Libraries
//              - Include in Cargo.toml under the dependencies array_tool = "1.0.3"
// -------------------------------------------
\end{lstlisting}
\begin{itemize}
\item Includes third-party library \texttt{array\_tool}.
\item Add to \texttt{Cargo.toml} under dependencies.
\end{itemize}
\end{frame}

\begin{frame}[fragile]
\frametitle{Using \texttt{array\_tool}: Vector Operations}
\begin{lstlisting}[language=Rust]
use array_tool::vec::*;
fn main() {
	let list_a = vec![4, 4, 6, 8, 9, 10];
	let list_b = vec![4, 5, 6];
	
	let overlap = list_a.intersect(list_b.clone());
	println!("The overlap = {:?}", overlap);
	
	let combined_set = list_a.union(list_b.clone());
	println!("The combined set = {:?}", combined_set);
}
\end{lstlisting}
\begin{itemize}
\item Demonstrates vector operations with \texttt{array\_tool}.
\item Finds the intersection and union of two vectors.
\end{itemize}
\end{frame}

\begin{frame}[fragile]
\frametitle{Using \texttt{array\_tool}: Repeating Vectors}
\begin{lstlisting}[language=Rust]
	println!("List B repeated three times = {:?}", list_b.times(3));
}
\end{lstlisting}
\begin{itemize}
\item Shows how to repeat elements of a vector using \texttt{array\_tool}.
\end{itemize}
\end{frame}


\section{Vector Intersection}
\begin{frame}[fragile]
\frametitle{Vector Intersection}
\begin{lstlisting}[language=Rust]
fn main() {
	let list_a = vec![4, 4, 6, 8, 9, 10];
	let list_b = vec![4, 5, 6];
	
	let overlap = list_a.intersect(list_b.clone());
	println!("The overlap = {:?}", overlap);
}
\end{lstlisting}
\begin{itemize}
\item The \texttt{intersect} method finds common elements between \texttt{list\_a} and \texttt{list\_b}.
\item \texttt{list\_b} is cloned to avoid borrowing issues.
\end{itemize}
\end{frame}

\section{Vector Union}
\begin{frame}[fragile]
\frametitle{Vector Union}
\begin{lstlisting}[language=Rust]
fn main() {
	let list_a = vec![4, 4, 6, 8, 9, 10];
	let list_b = vec![4, 5, 6];
	
	let combined_set = list_a.union(list_b.clone());
	println!("The combined set = {:?}", combined_set);
}
\end{lstlisting}
\begin{itemize}
\item The \texttt{union} method combines unique elements from both vectors.
\item \texttt{list\_b} is cloned to avoid borrowing issues.
\end{itemize}
\end{frame}

\section{Repeating Elements in a Vector}
\begin{frame}[fragile]
\frametitle{Repeating Elements in a Vector}
\begin{lstlisting}[language=Rust]
fn main() {
	let list_a = vec![4, 4, 6, 8, 9, 10];
	let list_b = vec![4, 5, 6];
	
	println!("List B repeated three times = {:?}", list_b.times(3));
}
\end{lstlisting}
\begin{itemize}
\item The \texttt{times} method repeats the elements of \texttt{list\_b} three times.
\item Useful for creating patterns or repeating sequences.
\end{itemize}
\end{frame}

\section{Documentation Comments}
\begin{frame}[fragile]
\frametitle{Documentation Comments}
\begin{lstlisting}[language=Rust]
/* 
These lines are not intended for documentation
// Documentation comments begin with three slashes. Ensure your code is error-free and well-documented before publishing.

// Command to generate documentation: cargo doc --open
*/
\end{lstlisting}
\begin{itemize}
\item Regular comments using \texttt{/* */} and \texttt{//} are not included in documentation.
\item Documentation comments use \texttt{///}.
\item Generate documentation with \texttt{cargo doc --open}.
\end{itemize}
\end{frame}

\section{Module-Level Documentation}
\begin{frame}[fragile]
\frametitle{Module-Level Documentation}
\begin{lstlisting}[language=Rust]
// The following lines will be included in the documentation 
// _______________________________________________

//! # Fitness Tracker Crate
//! 
//! This crate provides utilities for basic fitness calculations and metrics.
\end{lstlisting}
\begin{itemize}
\item Use \texttt{//!} for module-level documentation.
\item Provides an overview of the crate and its purpose.
\end{itemize}
\end{frame}

\section{Function Documentation}
\begin{frame}[fragile]
\frametitle{Function Doc: \texttt{calculate\_calories}}
\begin{lstlisting}[language=Rust]
/// Calculates daily calorie needs based on BMR and activity level
/// 
/// # Examples
/// ``` 
/// let bmr = 1500.0;
/// let activity_factor = 1.2;
/// let daily_calories = fitness_tracker::calculate_calories(bmr, activity_factor);
/// assert_eq!(1800.0, daily_calories); 
/// ```  
pub fn calculate_calories(bmr: f32, activity_level: f32) -> f32 {
	bmr * activity_level
}
\end{lstlisting}
\end{frame}

\begin{frame}[fragile]
\frametitle{Discussion: \texttt{calculate\_calories}}
\begin{itemize}
\item Documents the \texttt{calculate\_calories} function.
\item Includes examples to demonstrate usage.
\end{itemize}
\end{frame}

\section{Publishing Your Crate}

\begin{frame}[fragile]
	\frametitle{Publishing Your Crate: Steps 1-4}
	\begin{lstlisting}[language=Rust]
		// Steps for publishing your crate
		// 1. Create a GitHub account and log in to crates.io.
		// 2. Navigate to the crates.io dashboard, access API tokens, and create a new token.
		// 3. Copy the token.
		// 4. In the terminal, execute cargo login followed by the token from crates.io.
	\end{lstlisting}
	\begin{itemize}
		\item Steps for creating an account and obtaining an API token.
		\item Logging in to \texttt{crates.io} using the token.
	\end{itemize}
\end{frame}

\begin{frame}[fragile]
	\frametitle{Publishing Your Crate: Steps 5-7}
	\begin{lstlisting}[language=Rust]
		// 5. Go back to crates.io, verify your email in Account Settings -> Profile -> Email -> Save.
		// 6. Ensure your Cargo.toml includes:
		// version = "0.1.1", edition = "2021",
		// authors = ["Your Name"],
		// description = "Fitness calculation utilities",
		// license = "MIT".
		// 7. Run cargo publish --allow-dirty to publish your crate.
	\end{lstlisting}
	\begin{itemize}
		\item Verifying your email on \texttt{crates.io}.
		\item Setting up the necessary fields in \texttt{Cargo.toml}.
		\item Publishing your crate using \texttt{cargo publish --allow-dirty}.
	\end{itemize}
\end{frame}

\begin{frame}[fragile]
	\frametitle{Publishing Your Crate: Steps 8-11}
	\begin{lstlisting}[language=Rust]
		// 8. To update, modify the version number in Cargo.toml and run the publish command again.
		// 9. Your crate will appear on your dashboard.
		// 10. From your dashboard, you can yank or unyank versions to control downloads.
		// 11. Your crate will also be searchable.
	\end{lstlisting}
	\begin{itemize}
		\item Updating the crate by modifying the version number and republishing.
		\item Managing your crate from the \texttt{crates.io} dashboard.
		\item Yank or unyank versions to control availability.
		\item Your crate becomes searchable on \texttt{crates.io}.
	\end{itemize}
\end{frame}


\section{Demonstrating Crate Usage}
\begin{frame}[fragile]
\frametitle{Demonstrating Crate Usage}
\begin{lstlisting}[language=Rust]
// To demonstrate usage, create a new project and include the following crate:
use my_bmi_crate_01::calculate_bmi; 
fn main() {
	println!("BMI for 70kg and 1.75m height: {}", calculate_bmi(70.0, 1.75));
	println!("Daily calories needed: {}", calculate_calories(1500.0, 1.2));
}
\end{lstlisting}
\begin{itemize}
\item Shows how to use the \texttt{calculate\_bmi} and \texttt{calculate\_calories} functions from the crate.
\item Practical example of including and using a custom crate in a new project.
\end{itemize}
\end{frame}

\end{document}
