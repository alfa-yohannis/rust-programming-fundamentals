\documentclass[aspectratio=169, table]{beamer}

\usepackage[utf8]{inputenc}
\usepackage{listings} 

\usetheme{Pradita}

\subtitle{IF120203-Programming Fundamentals}

\title{Session-09:\\\LARGE{Input-Ouput in Rust}\\ \vspace{10pt}}
\date[Serial]{\scriptsize {PRU/SPMI/FR-BM-18/0222}}
\author[Pradita]{\small{\textbf{Alfa Yohannis}}}

\lstdefinelanguage{Rust} {
keywords={::, use, const, struct, fn, let, mut, move, pub, str, i32, if, else, true, false, enum, match, impl, for, while, loop, return, mod, crate, super, Self, self, break, continue, extern, ref, static, type, unsafe, as, async, await, dyn, macro_rules},
basicstyle=\ttfamily\footnotesize,
keywordstyle=\color{blue}\bfseries,
ndkeywords={app, button, draw, enums, frame, input, prelude, table, window, Window, Button, Input, SecretInput, Table, Frame, Lazy, Mutex, new},
ndkeywordstyle=\color{purple}\bfseries,
sensitive=true,
commentstyle=\color{gray},
stringstyle=\color{red},
numbers=left,
numberstyle=\tiny\color{gray},
breaklines=true,
frame=lines,
backgroundcolor=\color{lightgray!10},
tabsize=2,
comment=[l]{//},
morecomment=[s]{/*}{*/},
commentstyle=\color{gray}\ttfamily,
stringstyle=\color{purple}\ttfamily,
showstringspaces=false
}

\begin{document}

\frame{\titlepage}

% Add table of contents slide
\begin{frame}{Contents}
	\vspace{15pt}
	\begin{columns}[t]
		\begin{column}{.5\textwidth}
			\tableofcontents[sections={1-12}]
		\end{column}
		\begin{column}{.5\textwidth}
			\tableofcontents[sections={13-24}]
		\end{column}
	\end{columns}
\end{frame}

\section{Console-based App - Introduction}
\begin{frame}
	\frametitle{Console-based App - Introduction}
	\vspace{15pt}
	This Rust application is a \textbf{console-based} program designed for user management.
	\begin{itemize}
		\item It handles user authentication, including login functionality.
		\item Users can view, add, update, and delete other users.
		\item User data is stored in and read from a CSV file.
	\end{itemize}
\end{frame}

\section{Imports and Constants}
\begin{frame}[fragile]
	\frametitle{Imports and Constants}
	\begin{lstlisting}[language=Rust]
		use std::fs::File;
		use std::io::{self, BufRead, BufReader, Write};
		use std::process;
		
		const CSV_FILE_PATH: &str = "users.csv";
	\end{lstlisting}
	\begin{itemize}
		\item Imports necessary modules for file handling, I/O operations, and process management.
		\item Defines a constant `CSV\_FILE\_PATH` for the path of the CSV file.
	\end{itemize}
\end{frame}

\section{User Structure}
\begin{frame}[fragile]
	\frametitle{User Structure}
	\begin{lstlisting}[language=Rust]
		#[derive(Clone)]
		struct User {
			username: String,
			password: String,
		}
	\end{lstlisting}
	\begin{itemize}
		\item Defines a `User` structure with `username` and `password` fields.
		\item The `Clone` trait is derived to allow cloning of `User` instances.
	\end{itemize}
\end{frame}

\section{Main Function}

\begin{frame}[fragile]
\frametitle{Main Function - Part 1}
\begin{lstlisting}[language=Rust]
fn main() {
	let mut users = load_users_from_csv();
	loop {
		println!("Login Screen");
		println!("Enter Username:");
		let username = read_input();
		println!("Enter Password:");
		let password = read_input();
		\end{lstlisting}
		\begin{itemize}
			\item The `main` function starts by loading users from the CSV file.
			\item Prompts the user to enter their username and password.
		\end{itemize}
	\end{frame}
	
	\begin{frame}[fragile]
		\vspace{15pt}
		\frametitle{Main Function - Part 2}
		\begin{lstlisting}[language=Rust]
		if users.iter()
		.any(|user| user.username == username && user.password == password)
		{
			println!("Login successful!");
			user_list_screen(&mut users);
		} else {
			println!("Login failed. Incorrect username or password.");
		}
	}
}
\end{lstlisting}
\begin{itemize}
\item Validates credentials against the loaded user list.
\item Displays a login failure message if credentials are incorrect.
\item On successful login, transitions to the user list screen.
\item Continues looping until the application is terminated.
\end{itemize}
\end{frame}


\section{User List Screen}

\begin{frame}[fragile]
	\frametitle{User List Screen - Part 1}
	\vspace{15pt}
	\begin{lstlisting}[language=Rust]
		fn user_list_screen(users: &mut Vec<User>) {
			loop {
				println!("User List Screen");
				println!("1. Add User");
				println!("2. Update User");
				println!("3. Delete User");
				println!("4. List Users");
				println!("5. Logout");
				println!("6. Exit Application");
				
				let choice = read_input();
				match choice.as_str() {
				\end{lstlisting}
				\begin{itemize}
					\item Displays a menu for user management options.
					\item Handles choices for adding, updating, deleting users, and listing users.
				\end{itemize}
			\end{frame}
			
			\begin{frame}[fragile]
			\vspace{15pt}	
				\frametitle{User List Screen - Part 2}
				\begin{lstlisting}[language=Rust]
					"1" => add_user(users),
					"2" => update_user(users),
					"3" => delete_user(users),
					"4" => list_users(users),
					"5" => break, // Return to the login screen
					"6" => {
						println!("Exiting application...");
						process::exit(0); // Terminate the process
					}
					_ => println!("Invalid choice, please try again."),
				}
			}
		}
	\end{lstlisting}
	\begin{itemize}
		\item Handles user choices to log out or exit the application.
		\item Terminates the process if the exit option is chosen.
	\end{itemize}
\end{frame}


\section{Add User Function}

\begin{frame}[fragile]
	\frametitle{Add User Function - Part 1}
	\begin{lstlisting}[language=Rust]
		fn add_user(users: &mut Vec<User>) {
			println!("Enter new username:");
			let username = read_input();
			println!("Enter new password:");
			let password = read_input();
			
			users.push(User {
				username: username.clone(),
				password: password.clone(),
			});
		\end{lstlisting}
		\begin{itemize}
			\item Prompts for and reads the new username and password.
			\item Adds the new user to the `users` vector.
		\end{itemize}
	\end{frame}
	
	\begin{frame}[fragile]
		\frametitle{Add User Function - Part 2}
		\begin{lstlisting}[language=Rust]
			persist_users_to_csv(users);
			println!("User added successfully.");
		}
	\end{lstlisting}
	\begin{itemize}
		\item Updates the CSV file with the new user list.
		\item Prints a confirmation message indicating the user was added successfully.
	\end{itemize}
\end{frame}


\section{Update User Function}

\begin{frame}[fragile]
\frametitle{Update User Function - Part 1}
\vspace{15pt}
\begin{lstlisting}[language=Rust]
fn update_user(users: &mut Vec<User>) {
	println!("Enter username to update:");
	let username = read_input();
	let mut updated = false;
	
	if let Some(pos) = users.iter().position(|u| u.username == username) {
		println!("Enter new password:");
		let new_password = read_input();
		users[pos].password = new_password;
		updated = true;
	}
\end{lstlisting}
\begin{itemize}
	\item Prompts for the username to update.
	\item Searches for the user in the `users` vector.
	\item Prompts for a new password and updates it if the user is found.
\end{itemize}
\end{frame}

\begin{frame}[fragile]
\frametitle{Update User Function - Part 2}
\begin{lstlisting}[language=Rust]
	if updated {
		persist_users_to_csv(users);
		println!("User updated successfully.");
	} else {
		println!("User not found.");
	}
}
\end{lstlisting}
\begin{itemize}
\item Saves the updated user information to the CSV file if the update was successful.
\item Prints a success or failure message based on the outcome.
\end{itemize}
\end{frame}


\section{Delete User Function}

\begin{frame}[fragile]
\frametitle{Delete User Function - Part 1}
\begin{lstlisting}[language=Rust]
fn delete_user(users: &mut Vec<User>) {
	println!("Enter username to delete:");
	let username = read_input();
	let mut deleted = false;
	
	if let Some(pos) = users.iter().position(|u| u.username == username) {
		users.remove(pos);
		deleted = true;
	}
\end{lstlisting}
\begin{itemize}
	\item Prompts for the username to delete.
	\item Searches for the user in the `users` vector.
	\item Removes the user if found.
\end{itemize}
\end{frame}

\begin{frame}[fragile]
\frametitle{Delete User Function - Part 2}
\begin{lstlisting}[language=Rust]
	if deleted {
		persist_users_to_csv(users);
		println!("User deleted successfully.");
	} else {
		println!("User not found.");
	}
}
\end{lstlisting}
\begin{itemize}
\item Updates the CSV file if the user was deleted.
\item Prints a success or failure message based on the outcome.
\end{itemize}
\end{frame}


\section{List User Function}
\begin{frame}[fragile]
	\frametitle{List Users Function}
	\begin{lstlisting}[language=Rust]
		fn list_users(users: &Vec<User>) {
			println!("Current Users:");
			for user in users.iter() {
				println!("Username: {}, Password: {}", user.username, user.password);
			}
		}
	\end{lstlisting}
	\begin{itemize}
		\item Lists all users with their usernames and passwords.
		\item Iterates over the `users` vector and prints each user's details.
	\end{itemize}
\end{frame}

\section{Read Input Function}
\begin{frame}[fragile]
	\frametitle{Read Input Function}
	\begin{lstlisting}[language=Rust]
		fn read_input() -> String {
			let mut input = String::new();
			io::stdin().read_line(&mut input).unwrap();
			input.trim().to_string()
		}
	\end{lstlisting}
	\begin{itemize}
		\item Reads a line of input from the user.
		\item Trims whitespace and returns the input as a `String`.
	\end{itemize}
\end{frame}

\section{Load User from CSV}

\begin{frame}[fragile]
\frametitle{Loading Users from CSV - Part 1}
\begin{lstlisting}[language=Rust]
fn load_users_from_csv() -> Vec<User> {
	let file = File::open(CSV_FILE_PATH).unwrap_or_else(|_| {
		let mut file = File::create(CSV_FILE_PATH).unwrap();
		writeln!(file, "username,password").unwrap();
		file
	});
	
	let reader = BufReader::new(file);
	let mut users = Vec::new();
\end{lstlisting}
\begin{itemize}
	\item Opens the CSV file or creates it if it does not exist.
	\item Creates a `BufReader` and initializes an empty `users` vector.
\end{itemize}
\end{frame}
	
\begin{frame}[fragile]
\frametitle{Loading Users from CSV - Part 2}
\vspace{15pt}
\begin{lstlisting}[language=Rust]
for line in reader.lines().skip(1) {
	if let Ok(line) = line {
		let mut parts = line.split(',');
		if let (Some(username), Some(password)) = (parts.next(), parts.next()) {
			users.push(User {
				username: username.to_string(),
				password: password.to_string(),
			}); } } }
users
}
\end{lstlisting}
\begin{itemize}
\item Reads each line after the header.
\item Splits the line into `username` and `password`, and creates `User` objects.
\item Adds the `User` objects to the `users` vector and returns it.
\end{itemize}
\end{frame}



\section{Persist Users to CSV}
\begin{frame}[fragile]
	\frametitle{Persist Users to CSV}
	\begin{lstlisting}[language=Rust]
		fn persist_users_to_csv(users: &Vec<User>) {
			let mut file = File::create(CSV_FILE_PATH).unwrap();
			writeln!(file, "username,password").unwrap(); // Write the header
			for user in users.iter() {
				writeln!(file, "{},{}", user.username, user.password).unwrap();
			}
		}
	\end{lstlisting}
	\begin{itemize}
		\item Saves the current users to the CSV file.
		\item Writes the header and user details to the file.
	\end{itemize}
\end{frame}

\section{GUI-based App - Introduction}
\begin{frame}
\frametitle{GUI-based App - Introduction}
\begin{itemize}
\item Creating a simple GUI application using FLTK in Rust.
\item The application includes a login screen and a user list management screen.
\item User data is persisted in a CSV file.
\end{itemize}
\end{frame}

\section{Dependencies}
\begin{frame}[fragile]
\frametitle{Dependencies}
\begin{lstlisting}[language=Rust]
use fltk::{
app, button::Button, draw, enums,
frame::Frame, input::{Input, SecretInput},
prelude::*, table::{self, Table},
window::Window,
};
\end{lstlisting}
\begin{itemize}
\item FLTK crate for GUI components.
\item Provides basic components like `Button`, `Input`, `Table`, and `Window`.
\end{itemize}
\end{frame}

\begin{frame}[fragile]
\frametitle{Dependencies (cont.)}
\begin{lstlisting}[language=Rust]
use lazy_static::lazy_static;
use std::fs::File;
use std::io::{BufRead, BufReader, Write};
use std::sync::Mutex;
use std::thread;
\end{lstlisting}
\begin{itemize}
\item LazyStatic for global static data.
\item Standard library modules for file I/O and threading.
\end{itemize}
\end{frame}

\section{Constants and User Struct}
\begin{frame}[fragile]
\frametitle{Constants and User Struct}
\begin{lstlisting}[language=Rust]
const DEFAULT_USERNAME: &str = "admin";
const DEFAULT_PASSWORD: &str = "1234";
const CSV_FILE_PATH: &str = "users.csv";

#[derive(Clone)]
struct User {
username: String,
password: String,
}
\end{lstlisting}
\begin{itemize}
\item Default username and password.
\item Path to the CSV file for storing user data.
\item User struct to represent user data.
\end{itemize}
\end{frame}

\section{Global Static Users}
\begin{frame}[fragile]
\frametitle{Global Static Users}
\begin{lstlisting}[language=Rust]
lazy_static! {
static ref USERS: Mutex<Vec<User>> = Mutex::new(Vec::new());
}
\end{lstlisting}
\begin{itemize}
\item Use LazyStatic to create a globally accessible list of users.
\item Mutex to ensure thread-safe access to the user list.
\end{itemize}
\end{frame}

\section{Main Function}
\begin{frame}[fragile]
\frametitle{Main Function}
\begin{lstlisting}[language=Rust]
fn main() {
load_users_from_csv();
let app = app::App::default();
let (screen_width, screen_height) = app::screen_size();

// Main window (login form)
let window_width = 400;
let window_height = 300;
\end{lstlisting}
\begin{itemize}
\item Load users from CSV file.
\item Create the main application.
\item Define window dimensions.
\end{itemize}
\end{frame}

\begin{frame}[fragile]
\frametitle{Main Function (cont.)}
\vspace{15pt}
\begin{lstlisting}[language=Rust]
let mut login_window = Window::new(
(screen_width as i32 - window_width) / 2,
(screen_height as i32 - window_height) / 2,
window_width,
window_height,
"Login Screen",
);
login_window.make_resizable(true);
let mut username_input = Input::new(150, 50, 200, 30, "Username:");
let mut password_input = SecretInput::new(150, 100, 200, 30, "Password:");
\end{lstlisting}
\begin{itemize}
\item Create and center the login window.
\item Make the window resizable.
\item Add username and password input fields.
\end{itemize}
\end{frame}

\begin{frame}[fragile]
\frametitle{Main Function (cont.)}
\vspace{15pt}
\begin{lstlisting}[language=Rust]
// Set default values
username_input.set_value(DEFAULT_USERNAME);
password_input.set_value(DEFAULT_PASSWORD);

let mut login_button = Button::new(150, 150, 80, 40, "Login");
let mut clear_button = Button::new(250, 150, 80, 40, "Clear");
let message_frame = Frame::new(150, 200, 200, 40, "");

login_window.end();
login_window.show();
\end{lstlisting}
\begin{itemize}
\item Set default values for inputs.
\item Add login and clear buttons.
\item Add a message frame for displaying messages.
\item Show the login window.
\end{itemize}
\end{frame}

\begin{frame}[fragile]
\frametitle{Main Function (cont.)}
\begin{lstlisting}[language=Rust]
login_button.set_callback({
	let username_input = username_input.clone();
	let password_input = password_input.clone();
	let mut message_frame = message_frame.clone();
	let mut login_window = login_window.clone();
	move |_| {
		let username = username_input.value();
		let password = password_input.value();
	\end{lstlisting}
	\begin{itemize}
		\item Set callback for login button.
		\item Clone input fields and message frame for use in callback.
		\item Get values from input fields on button click.
	\end{itemize}
\end{frame}

\begin{frame}[fragile]
	\frametitle{Main Function (cont.)}
	\vspace{15pt}
	\begin{lstlisting}[language=Rust]
		let users = USERS.lock().unwrap();
		if users.iter().any(|user| user.username == username && user.password == password) {
			login_window.hide();
			let mut user_list_window = Window::new(
			(screen_width as i32 - window_width) / 2,
			(screen_height as i32 - window_height) / 2,
			window_width,
			window_height,
			"User List", );
			user_list_window.make_resizable(true);
		\end{lstlisting}
		\begin{itemize}
			\item Check if username and password are valid.
			\item Hide login window on successful login.
			\item Create and show user list window.
		\end{itemize}
	\end{frame}
	
	\begin{frame}[fragile]
		\frametitle{Main Function (cont.)}
		\begin{lstlisting}[language=Rust]
			let mut user_table = Table::new(10, 10, 380, 280, "");
			user_table.set_rows(users.len() as i32);
			user_table.set_cols(2);
			user_table.set_col_header(true);
			user_table.set_col_width(0, 180);
			user_table.set_col_width(1, 180);
		\end{lstlisting}
		\begin{itemize}
			\item Create user table in user list window.
			\item Set table dimensions and headers.
		\end{itemize}
	\end{frame}
	
	\begin{frame}[fragile]
		\frametitle{Main Function (cont.)}
		\begin{lstlisting}[language=Rust]
			user_table.set_cell_draw(move |t, ctx, row, col, x, y, w, h| {
				if ctx == table::TableContext::Cell {
					let data = match col {
						0 => users[row as usize].username.clone(),
						1 => users[row as usize].password.clone(),
						_ => String::new(),
					};
					draw_data(&data, x, y, w, h, t.is_selected(row, col));
				} 
				else if ctx == table::TableContext::ColHeader {
				}
			});
		\end{lstlisting}
		\begin{itemize}
			\item Define cell draw function to display user data.
			\item Handle column header context in cell draw function.
		\end{itemize}
	\end{frame}
	
	
	\begin{frame}[fragile]
		\frametitle{Main Function (cont.)}
		\begin{lstlisting}[language=Rust]
			let header = match col {
				0 => "Username",
				1 => "Password",
				_ => "",
			};
			draw_header(header, x, y, w, h);
		}
	});
	user_table.set_selection(0, 0);
	user_table.select_all_rows(true);
	user_table.end();
\end{lstlisting}
\begin{itemize}
	\item Draw headers for table columns.
	\item Set initial selection for table.
\end{itemize}
\end{frame}

\begin{frame}[fragile]
\frametitle{Main Function (cont.)}
\begin{lstlisting}[language=Rust]
	user_list_window.end();
	user_list_window.show();
} else {
	message_frame.set_label("Invalid username or password");
}
});
\end{lstlisting}
\begin{itemize}
\item Show user list window on successful login.
\item Display error message on failed login.
\end{itemize}
\end{frame}

\begin{frame}[fragile]
\frametitle{Main Function (cont.)}
\vspace{15pt}
\begin{lstlisting}[language=Rust]
clear_button.set_callback({
let username_input = username_input.clone();
let password_input = password_input.clone();
let mut message_frame = message_frame.clone();
move |_| {
	username_input.set_value(DEFAULT_USERNAME);
	password_input.set_value(DEFAULT_PASSWORD);
	message_frame.set_label("");
}
});
app.run().unwrap();
\end{lstlisting}
\begin{itemize}
\item Set callback for clear button.
\item Reset input fields and clear message on button click.
\item Run the application.
\end{itemize}
\end{frame}

c
\begin{frame}[fragile]
\frametitle{Utility Functions}
\begin{lstlisting}[language=Rust]
fn load_users_from_csv() {
let file = File::open(CSV_FILE_PATH).expect("Failed to open CSV file");
let reader = BufReader::new(file);
let mut users = USERS.lock().unwrap();
for line in reader.lines() {
	let line = line.expect("Failed to read line");
	let parts: Vec<&str> = line.split(',').collect();
\end{lstlisting}
\begin{itemize}
	\item Load user data from CSV file.
	\item Use BufReader for efficient reading.
	\item Lock the user list for thread-safe access.
\end{itemize}
\end{frame}

\begin{frame}[fragile]
\frametitle{Utility Functions (cont.)}
\begin{lstlisting}[language=Rust]
	if parts.len() == 2 {
		let user = User {
			username: parts[0].to_string(),
			password: parts[1].to_string(),
		};
		users.push(user);
	}
}
}


\end{lstlisting}
\begin{itemize}
\item Parse CSV lines into user data.
\item Append parsed users to the global list.

\end{itemize}
\end{frame}

\begin{frame}[fragile]
\frametitle{Utility Functions (cont.)}
\begin{lstlisting}[language=Rust]
fn save_users_to_csv() {
let file = File::create(CSV_FILE_PATH).expect("Failed to create CSV file");
let mut writer = std::io::BufWriter::new(file);
let users = USERS.lock().unwrap();
for user in users.iter() {
	writeln!(writer, "{},{}", user.username, user.password)
	.expect("Failed to write to CSV file");
}
\end{lstlisting}
\begin{itemize}
\item Save users to CSV file using BufWriter.
\item Write user data to CSV file.
\item Ensure data is written securely.
\end{itemize}
\end{frame}

\begin{frame}[fragile]
\frametitle{Utility Functions (cont.)}
\begin{lstlisting}[language=Rust]
fn draw_header(header: &str, x: i32, y: i32, w: i32, h: i32) {
	draw::draw_box(enums::FrameType::FlatBox, x, y, w, h, enums::Color::Dark3);
	draw::set_draw_color(enums::Color::Black);
	draw::draw_text2(header, x, y, w, h, enums::Align::Center);
}
\end{lstlisting}
\begin{itemize}
\item Draw table headers with specified styles.
\item Enhance the visual appearance of headers.
\end{itemize}
\end{frame}


\begin{frame}[fragile]
\frametitle{Utility Functions (cont.)}
\begin{lstlisting}[language=Rust]
fn draw_data(data: &str, x: i32, y: i32, w: i32, h: i32, selected: bool) {
	if selected {
		draw::set_draw_color(enums::Color::Light2);
		draw::draw_rectf(x, y, w, h);
	}
	draw::set_draw_color(enums::Color::Black);
	draw::draw_text2(data, x, y, w, h, enums::Align::Center);
}
\end{lstlisting}
\begin{itemize}
\item Draw table data cells with specified styles.
\item Highlight selected cells.
\end{itemize}
\end{frame}

\end{document}
