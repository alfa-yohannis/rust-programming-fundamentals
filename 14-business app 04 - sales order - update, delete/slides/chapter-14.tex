\documentclass[aspectratio=169, table]{beamer}

\usepackage[utf8]{inputenc}
\usepackage{listings} 

\usetheme{Pradita}

\subtitle{IF120203-Programming Fundamentals}

\title{Session-12:\\\LARGE{Item Master Data Form:\\Update and Delete}}
\date[Serial]{\scriptsize {PRU/SPMI/FR-BM-18/0222}}
\author[Pradita]{\small{\textbf{Alfa Yohannis}}}


\lstdefinelanguage{Rust} {
keywords={::, use, const, struct, fn, let, mut, move, pub, str, i32, if, else, true, false, enum, match, impl, for, while, loop, return, mod, crate, super, Self, self, break, continue, extern, ref, static, type, unsafe, as, async, await, dyn, macro_rules},
basicstyle=\ttfamily\footnotesize,
keywordstyle=\color{blue}\bfseries,
ndkeywords={Result, Option, Ok, None, Item, Vec, Error, Some, app, button, draw, enums, frame, input, prelude, table, window, Window, Button, Input, SecretInput, Table, Frame, Lazy, Mutex, new},
ndkeywordstyle=\color{purple}\bfseries,
sensitive=true,
commentstyle=\color{gray},
stringstyle=\color{red},
numbers=left,
numberstyle=\tiny\color{gray},
breaklines=true,
frame=lines,
backgroundcolor=\color{lightgray!10},
tabsize=2,
comment=[l]{//},
morecomment=[s]{/*}{*/},
commentstyle=\color{gray}\ttfamily,
string=[s]{'}{'},
morestring=[s]{"}{"},
stringstyle=\color{teal}\ttfamily,
showstringspaces=false
}



\begin{document}

\frame{\titlepage}

% Add table of contents slide
\begin{frame}{Contents}
\vspace{15pt}
\begin{columns}[t]
\begin{column}{.5\textwidth}
\tableofcontents[sections={1-3}]
\end{column}
\begin{column}{.5\textwidth}
\tableofcontents[sections={4-6}]
\end{column}
\end{columns}
\end{frame}


\section{Introduction}
\begin{frame}{Introduction}
In this session, we will explore how to integrate FLTK with PostgreSQL in a Rust application. We will cover:
\begin{itemize}
\item Database setup
\item Main application setup
\item Table configuration
\item Button callbacks
\end{itemize}
\end{frame}

\section{Database Setup}
\subsection{Imports and Setup}
\begin{frame}[fragile]{Imports and Setup}
\vspace{15pt}
\begin{lstlisting}[language=Rust]
#[macro_use]
extern crate lazy_static;

use async_std::task;
use business_app::item::Item;
use fltk::{
app, button::Button, draw, enums, input::Input,
prelude::*, table::{self, Table}, window::Window,
};
use std::sync::Mutex;
use tokio_postgres::{Client, NoTls};
\end{lstlisting}

\begin{itemize}
\item Imports necessary crates and modules.
\item Sets up \texttt{lazy\_static} for static variables.
\item Defines \texttt{Item} from \texttt{busines\_app}.
\end{itemize}
\end{frame}

\subsection{Database Client}
\begin{frame}[fragile]{Database Client Part 1}
\begin{lstlisting}[language=Rust]
lazy_static! {
static ref DB_CLIENT: Mutex<Client> = Mutex::new(task::block_on(create_db_client()));
static ref ITEMS: Mutex<Vec<Item>> = Mutex::new(task::block_on(load_items_from_db(
&DB_CLIENT.lock().unwrap()
)));
}
\end{lstlisting}

\begin{itemize}
\item Defines global \texttt{DB\_CLIENT} and \texttt{ITEMS} using \texttt{lazy\_static}.
\item \texttt{task::block\_on} ensures asynchronous operations are awaited.
\end{itemize}
\end{frame}

\begin{frame}[fragile]{Database Client Part 2}
\vspace{15pt}
\begin{lstlisting}[language=Rust]
async fn create_db_client() -> Client {
let (client, connection) = tokio_postgres::connect("host=localhost user=postgres password=1234 dbname=session10", NoTls, ).await.unwrap();

tokio::spawn(async move {
	if let Err(e) = connection.await {
		eprintln!("Connection error: {}", e);
	}
});

client
}
\end{lstlisting}

\begin{itemize}
\item \texttt{create\_db\_client} establishes a PostgreSQL connection asynchronously.
\item \texttt{tokio::spawn} manages the connection in the background.
\end{itemize}
\end{frame}

\begin{frame}[fragile]{Database Client Part 3}
\begin{lstlisting}[language=Rust]
async fn load_items_from_db(client: &Client) -> Vec<Item> {
business_app::item::list(client)
.await
.unwrap_or_else(|_| Vec::new())
}
\end{lstlisting}

\begin{itemize}
\item \texttt{load\_items\_from\_db} retrieves items from the database.
\item Returns an empty vector if the database operation fails.
\end{itemize}
\end{frame}

\subsection{Database Operations}
\begin{frame}[fragile]{Database Operations}
\begin{lstlisting}[language=Rust]
async fn persist_item_to_db(client: &Client, item: &Item) {
business_app::item::add(client, item).await.unwrap();
}

async fn delete_item_from_db(client: &Client, code: &str) {
business_app::item::delete(client, code).await.unwrap();
}

async fn update_item_in_db(client: &Client, item: &Item) {
business_app::item::update(client, item).await.unwrap();
}
\end{lstlisting}

\begin{itemize}
\item Defines functions for database operations: add, delete, and update items.
\item Each function handles asynchronous operations and unwraps results.
\end{itemize}
\end{frame}

\section{Main Application}
\subsection{Setup}
\begin{frame}[fragile]{Main Application Setup Part 1}
\vspace{15pt}
\begin{lstlisting}[language=Rust]
#[tokio::main]
async fn main() {
let app = app::App::default();
let (screen_width, screen_height) = app::screen_size();
let window_width = 600;
let window_height = 400;
let mut window = Window::new(
(screen_width as i32 - window_width) / 2,
(screen_height as i32 - window_height) / 2,
window_width, window_height, "Item List",
);
\end{lstlisting}

\begin{itemize}
\item Initializes the FLTK application.
\item Creates and centers a window on the screen.
\item Sets up window dimensions and title.
\end{itemize}
\end{frame}

\begin{frame}[fragile]{Main Application Setup Part 2}
\vspace{15pt}
\begin{lstlisting}[language=Rust]
window.make_resizable(true);

let edit_code_input = Input::new(50, 50, 140, 30, "Code:");
let edit_name_input = Input::new(50, 100, 140, 30, "Name:");
let edit_currency_input = Input::new(50, 150, 140, 30, "Currency:");
let edit_price_input = Input::new(250, 50, 140, 30, "Price:");
let edit_quantity_input = Input::new(250, 100, 140, 30, "Quantity:");
let edit_unit_input = Input::new(250, 150, 140, 30, "Unit:");

let mut add_button = Button::new(50, 200, 80, 40, "Add");
let mut update_button = Button::new(150, 200, 80, 40, "Update");
let mut delete_button = Button::new(250, 200, 80, 40, "Delete");
\end{lstlisting}

\begin{itemize}
\item Makes the window resizable, sets up input fields and buttons, and configures their positions and sizes.
\end{itemize}
\end{frame}

\subsection{Table Configuration}
\begin{frame}[fragile]{Table Configuration Part 1}
\begin{lstlisting}[language=Rust]
let mut item_table = Table::new(50, 250, 500, 130, "");

// Use the global ITEMS vector to set the number of rows in the table
item_table.set_rows(ITEMS.lock().unwrap().len() as i32);
item_table.set_cols(6); // Code, Name, Currency, Price, Quantity, Unit
item_table.set_col_header(true);
item_table.set_row_header(true);
item_table.set_col_width_all(100); // Set a default column width for all columns
\end{lstlisting}

\begin{itemize}
\item Initializes the \texttt{Table} widget.
\item Configures the number of rows and columns.
\item Sets column and row headers.
\end{itemize}
\end{frame}

\begin{frame}[fragile]{Table Configuration Part 2}
\begin{lstlisting}[language=Rust]
item_table.set_col_width(0, 100);
item_table.set_col_width(1, 150);
item_table.set_col_resize(true);
item_table.set_col_resize_min(50);
item_table.end();
\end{lstlisting}

\begin{itemize}
\item Configures column widths and resizing properties.
\item Finalizes the table setup with \texttt{end()}.
\end{itemize}
\end{frame}

\subsection{Table Drawing}
\begin{frame}[fragile]{Table Drawing Part 1}
\begin{lstlisting}[language=Rust]
item_table.draw_cell({
	move |t, ctx, row, col, x, y, w, h| match ctx {
		table::TableContext::StartPage => 
		draw::set_font(enums::Font::Helvetica, 14),
		table::TableContext::ColHeader => {
			let headers = ["Code", "Name", "Currency", 
			"Price", "Quantity", "Unit"];
			draw_header(headers[col as usize], x, y, w, h);
		}
		table::TableContext::RowHeader => 
		draw_header(&format!("{}", row + 1), x, y, w, h),
	\end{lstlisting}
	
	\begin{itemize}
		\item Sets the font for the table content.
		\item Draws column headers based on the header array.
	\end{itemize}
\end{frame}

\begin{frame}[fragile]{Table Drawing Part 2}
	\vspace{15pt}
	\begin{lstlisting}[language=Rust]
		table::TableContext::Cell => {
			let item = &ITEMS.lock().unwrap()[row as usize];
			let data = match col {
				0 => &item.code,  1 => &item.name,  2 => &item.currency, 
				3 => &item.price.to_string(),  4 => &item.quantity.to_string(),
				5 => &item.unit.as_deref().unwrap_or(""),  _ => "",
			};
			draw_cell_data(data, x, y, w, h);
		}
		_ => (),
	}
});
\end{lstlisting}

\begin{itemize}
\item Retrieves and displays item data based on column index.
\item Manages drawing of cell content.
\end{itemize}
\end{frame}


\section{Button Callbacks}
\subsection{Add Button}
\begin{frame}[fragile]{Add Button Callback Part 1}
\begin{lstlisting}[language=Rust]
add_button.set_callback({
	let mut item_table = item_table.clone();
	let mut edit_code_input = edit_code_input.clone();
	let mut edit_name_input = edit_name_input.clone();
	let mut edit_currency_input = edit_currency_input.clone();
	let mut edit_price_input = edit_price_input.clone();
	let mut edit_quantity_input = edit_quantity_input.clone();
	let mut edit_unit_input = edit_unit_input.clone();
\end{lstlisting}

\begin{itemize}
	\item Defines the callback for the "Add" button.
	\item Creates a new \texttt{Item} based on user input.
\end{itemize}
\end{frame}

\begin{frame}[fragile]{Add Button Callback Part 2}
\vspace{15pt}
\begin{lstlisting}[language=Rust]
	move |_| {
		let new_item = Item {
			code: edit_code_input.value(),
			name: edit_name_input.value(),
			currency: edit_currency_input.value(),
			price: edit_price_input.value().parse().unwrap_or(0.0),
			quantity: edit_quantity_input.value().parse().unwrap_or(0.0),
			unit: if edit_unit_input.value().is_empty() {
				None
			} else {
				Some(edit_unit_input.value())
			},
		};
	\end{lstlisting}
	
	\begin{itemize}
		\item Defines the callback for the "Add" button.
		\item Creates a new \texttt{Item} based on user input.
	\end{itemize}
\end{frame}

\begin{frame}[fragile]{Add Button Callback Part 3}
	\vspace{15pt}
	\begin{lstlisting}[language=Rust]
		// Persist the new item to the database
		task::block_on(persist_item_to_db(
		&DB_CLIENT.lock().unwrap(),
		&new_item
		));
		// Update the table
		let mut items = ITEMS.lock().unwrap();
		items.push(new_item);
		item_table.set_rows(items.len() as i32);
		item_table.redraw();
	}
});
\end{lstlisting}

\begin{itemize}
\item Adds the new item to the database and updates the table.
\item Refreshes the item list and redraws the table.
\end{itemize}
\end{frame}

\subsection{Update Button}
\begin{frame}[fragile]{Update Button Callback Part 1}
\vspace{15pt}
\begin{lstlisting}[language=Rust]
update_button.set_callback({
	let mut item_table = item_table.clone();
	let mut edit_code_input = edit_code_input.clone();
	let mut edit_name_input = edit_name_input.clone();
	let mut edit_currency_input = edit_currency_input.clone();
	let mut edit_price_input = edit_price_input.clone();
	let mut edit_quantity_input = edit_quantity_input.clone();
	let mut edit_unit_input = edit_unit_input.clone();
\end{lstlisting}
\begin{itemize}
	\item Defines the callback for the "Update" button.
	\item Prepares the updated item for database and table update.
\end{itemize}
\end{frame}

\begin{frame}[fragile]{Update Button Callback Part 2}
\vspace{15pt}
\begin{lstlisting}[language=Rust]
	move |_| {
		let updated_item = Item {
			code: edit_code_input.value(),
			name: edit_name_input.value(),
			currency: edit_currency_input.value(),
			price: edit_price_input.value().parse().unwrap_or(0.0),
			quantity: edit_quantity_input.value().parse().unwrap_or(0.0),
			unit: if edit_unit_input.value().is_empty() {
				None
			} else {
				Some(edit_unit_input.value())
			},
		};
	\end{lstlisting}
	\begin{itemize}
		\item Defines the callback for the "Update" button.
		\item Prepares the updated item for database and table update.
	\end{itemize}
\end{frame}

\begin{frame}[fragile]{Update Button Callback Part 3}
	\vspace{15pt}
	\begin{lstlisting}[language=Rust]
		// Update the item in the database
		task::block_on(update_item_in_db(&DB_CLIENT.lock().unwrap(), &updated_item));
		
		// Update the table
		let mut items = ITEMS.lock().unwrap();
		let pos = items.iter().position(|x| x.code == updated_item.code);
		if let Some(pos) = pos {
			items[pos] = updated_item;
			item_table.redraw();
		}
	}
});
\end{lstlisting}
\begin{itemize}
\item Updates the item in the \texttt{ITEMS} vector and refreshes the table.
\item Ensures the table reflects the changes immediately.
\end{itemize}
\end{frame}

\subsection{Delete Button}
\begin{frame}[fragile]{Delete Button Callback Part 1}
\begin{lstlisting}[language=Rust]
delete_button.set_callback({
	let mut item_table = item_table.clone();
	let mut edit_code_input = edit_code_input.clone();
	move |_| {
		let code_to_delete = edit_code_input.value();
	\end{lstlisting}
	
	\begin{itemize}
		\item Sets up the callback for the "Delete" button.
		\item Clones the \texttt{item\_table} and \texttt{edit\_code\_input} to use within the callback.
	\end{itemize}
\end{frame}

\begin{frame}[fragile]{Delete Button Callback Part 2}
	\begin{lstlisting}[language=Rust]
		// Delete the item from the database
		task::block_on(delete_item_from_db(
		&DB_CLIENT.lock().unwrap(),
		&code_to_delete
		));
	\end{lstlisting}
	
	\begin{itemize}
		\item Deletes the item from the database asynchronously.
		\item Utilizes \texttt{task::block\_on} to handle asynchronous database operations in a synchronous context.
	\end{itemize}
\end{frame}

\begin{frame}[fragile]{Delete Button Callback Part 3}
	\begin{lstlisting}[language=Rust]
		// Update the table
		let mut items = ITEMS.lock().unwrap();
		let pos = items.iter().position(|x| x.code == code_to_delete);
		if let Some(pos) = pos {
			items.remove(pos);
			item_table.set_rows(items.len() as i32);
			item_table.redraw();
		}
	}
});
\end{lstlisting}

\begin{itemize}
\item Updates the \texttt{ITEMS} vector by removing the deleted item.
\item Adjusts the number of rows in the \texttt{item\_table} and triggers a redraw.
\end{itemize}
\end{frame}

\begin{frame}[fragile]{Delete Button Callback Part 4}
\begin{lstlisting}[language=Rust]
window.end();
window.show();
app.run().unwrap();
\end{lstlisting}

\begin{itemize}
\item Finalizes the window setup and displays it.
\item Starts the FLTK application event loop.
\end{itemize}
\end{frame}

\section{Utility Functions}
\begin{frame}[fragile]
\frametitle{Utility Functions}
\begin{lstlisting}[language=Rust]
fn draw_header(txt: &str, x: i32, y: i32, w: i32, h: i32) {
	draw::draw_box(enums::FrameType::FlatBox, 
	x, y, w, h,
	enums::Color::from_rgb(180, 180, 180), 
	);
	draw::set_draw_color(enums::Color::Black);
	draw::draw_text2(txt, x, y, w, h, enums::Align::Center);
}
\end{lstlisting}
\begin{itemize}
\item Draw table headers with specified styles.
\item Enhance the visual appearance of headers.
\end{itemize}
\end{frame}

\begin{frame}[fragile]
\frametitle{Utility Functions (cont.)}
\vspace{15pt}
\begin{lstlisting}[language=Rust]
fn draw_data(txt: &str, x: i32, y: i32, w: i32, h: i32, selected: bool) {
	draw::set_draw_color(if selected {
		enums::Color::from_rgb(200, 200, 255)
	} else {
		enums::Color::White
	});
	// Draw the cell background
	draw::draw_rectf(x, y, w, h);
	// Draw the cell borders with a black color
	draw::set_draw_color(enums::Color::Black);
	draw::draw_rect(x, y, w, h);
	draw::draw_text2(txt, x, y, w, h, enums::Align::Center);
}
\end{lstlisting}
\begin{itemize}
\item Draw table data cells with specified styles.
\item Highlight selected cells.
\end{itemize}
\end{frame}

\end{document}
